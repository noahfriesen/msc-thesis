\chapter{Yangians}

\section{Simple Lie algebras}

\subsection{Structure}

To begin, we will review some basic structure and representation theory of a simple Lie algebra, following \cite{humphreys_introduction_1972}.
This will allow us to more easily see the many similarities that arise in the Yangian setting, detailed in the next section.

Let $\g$ be a finite-dimensional simple Lie algebra over $\C$, with Cartan matrix $(a_{ij})_{i,j\in\I}$.
Fix minimal symmetrizing integers $d_i\in\{1,2,3\}$ so that $d_ia_{ij}=d_ja_{ji}$.
Let $\{\alpha_i\}_{i\in\I}$ be a basis of simple roots relative to a Cartan subalgebra $\h\subset\g$, and define a nondegenerate symmetric bilinear form on $\h^{*}$ by $(\alpha_i,\alpha_j)=d_ia_{ij}$.

The following theorem gives a presentation of $\g$ in terms of so-called \emph{Chevalley--Serre} generators and relations.
Note that $\delta_{ij}$ denotes the Kronecker delta function, which is $1$ if $i=j$ and $0$ otherwise, and $\ad:\g\to\End(\g)$ is defined by $\ad(x)(y)=[x,y]$ for $x,y\in\g$.

\begin{theorem}[Serre]\label{T:Serre}
    $\g$ is generated by the elements $\{e_i,f_i,h_i\}_{i\in\I}$ subject to the following relations:
    \begin{enumerate}
        \item $[h_i,h_j] = 0$,
        \item $[e_i,f_j] = \delta_{ij}h_i$,
        \item $[h_i,e_j] = a_{ij}e_j, \quad [h_i,f_j] = -a_{ij}f_j$,
        \item $\ad(e_i)^{1-a_{ij}}(e_j) = \ad(f_i)^{1-a_{ij}}(f_j) = 0 \quad\text{if } i\neq j$.
    \end{enumerate}
\end{theorem}

This presentation is especially useful for studying the representation theory of $\g$.
Before discussing representations, we recall the \emph{universal enveloping algebra} $\Ug$, which is the associative algebra generated by elements of $\g$ in which the commutator corresponds to the Lie bracket of $\g$.
More precisely, $\Ug$ is the quotient of the tensor algebra
\[T(\g) := \C \oplus \g \oplus (\g\otimes\g) \oplus \cdots\]
by the two-sided ideal generated by the elements
\[x\otimes y - y\otimes x - [x,y]\]
for all $x,y\in\g$.
The following theorem gives us a description of $\Ug$ in terms of $\g$ itself, which has many useful consequences.

\begin{theorem}[Poincar\'e--Birkhoff--Witt]\label{T:PBW-g}
    If $x_1,\cdots,x_n$ is an ordered basis for $\g$, then $\{x_1^{k_1}\cdots x_n^{k_n} : k_i\in\N\}$ is a basis for $\Ug$.
\end{theorem}

Recall that $\g$ has a \emph{triangular decomposition}:
\[\g \cong \n^-\oplus\h\oplus\n^+,\]
where $\n^-$ and $\n^+$ are the subalgebras of $\g$ generated by the elements $f_i$ and $e_i$ for all $i\in\I$, respectively.
A consequence of the above theorem is that $\Ug$ also has a triangular decomposition: there is a vector space isomorphism
\[\Ug \cong U(\n^-)\otimes U(\h) \otimes U(\n^+).\]


\subsection{Representation theory}

A \emph{representation} of $\g$ is a vector space $V$ together with a linear map $\phi:\g\to\End(V)$ satisfying
\[\phi([x,y]) = \phi(x)\phi(y)-\phi(y)\phi(x)\]
for all $x,y\in\g$.
In other words, the map $\phi$ is a Lie algebra homomorphism from $\g$ to $\mfgl(V)$, where the latter is the endomorphism space $\End(V)$ with Lie bracket given by the commutator of endomorphisms.
Equivalently, a representation of $\g$ is a vector space $V$ together with a bilinear ``action'' map $\cdot:\g\times V\to V$ satisfying
\[[x,y]\cdot v = x\cdot(y\cdot v) - y\cdot(x\cdot v)\]
for all $x,y\in\g$ and $v\in V$.
From this definition, we see that representations of $\g$ correspond to modules over $\Ug$ and vice versa:
\[(xy)\cdot v = x\cdot (y\cdot v).\]

We would like to classify the finite-dimensional irreducible representations of $\g$.
In order to do so, we first consider a class of representations known as \emph{highest-weight} representations.

\begin{definition}\label{D:hw-g}
    $V$ is a \emph{highest-weight representation} with \emph{highest weight} $\lambda\in\h^*$ if there exists $v\in V$ such that for all $i\in\I$, the following hold:
    \begin{enumerate}
        \item $e_i\cdot v = 0$,
        \item $h_i\cdot v = \lambda(h_i)v$,
        \item $\Ug\cdot v = V$.
    \end{enumerate}
\end{definition}

The second condition in the above definition is equivalent to saying that the highest weight vector $v$ belongs to the \emph{weight space} of weight $\lambda$, which is defined as follows:
\[V_\lambda := \{v\in V : h\cdot v = \lambda(h)v \quad \forall h\in\h\}.\]
The sum of weight spaces $V_\mu$ over all $\mu\in\h^*$ is always a direct sum, and is always a subrepresentation of $V$.
In the case where $V$ is finite-dimensional, this subrepresentation is equal to $V$.

Another important thing to note is that the action of $\g$ permutes the weight spaces of $V$: for all $\mu\in\h^*$ and $i\in\I$, the generators $e_i$ and $f_i$ map $V_\mu$ into $V_{\mu+\alpha_i}$ and $V_{\mu-\alpha_i}$, respectively.
Thus we can think of the generators $e_i$ and $f_i$ as ``raising operators'' and ``lowering operators,'' respectively.
From this, along with the PBW theorem (Theorem \ref{T:PBW-g}), it follows that the weights of $V$ (that is, the linear functionals $\mu\in\h^*$ for which $V_\mu$ is nonempty) are all of the form
\[\mu = \lambda - \sum_{i\in\I} k_i\alpha_i\]
for some $k_i\in\N$.
This provides motivation for using the term ``highest weight'' in the above definition.

Next, we recall the definition of the \emph{Verma module} $M(\lambda)$ of highest weight $\lambda\in\h^*$, which is in some sense the ``largest'' highest-weight representation of $\g$ of a given highest weight.
To construct the Verma module, we make $\Ug$ into a highest-weight representation of $\g$ by taking a quotient in order to impose only the conditions required by Definition \ref{D:hw-g}.
Hence $M(\lambda)$ is the quotient of $\Ug$ by the left ideal generated by the elements $e_i$ and $h_i-\lambda(h_i)1$ for all $i\in\I$.
This representation has a unique maximal subrepresentation, and therefore a unique irreducible quotient representation that we will denote $L(\lambda)$.

The following theorem (known as the \emph{theorem of the highest weight}) completely classifies the finite-dimensional irreducible representations of $\g$ up to isomorphism.

\begin{theorem}\label{T:hw-g}
    \begin{enumerate}
        \item Every finite-dimensional irreducible representation of $\g$ is isomorphic to $L(\lambda)$ for some $\lambda\in\h^*$.
        \item $L(\lambda)$ is finite-dimensional if and only if $\lambda(h_i)\in\N$ for all $i\in\I$.
    \end{enumerate}
\end{theorem}

Finally, we give a couple of examples to better illustrate some of the concepts above.

\begin{example}\label{E:sl2-ad}
    Consider the special linear Lie algebra $\mfsl_2$, which has $1\times 1$ Cartan matrix $[2]$.
    By Theorem \ref{T:Serre}, $\mfsl_2$ is generated by $\{e,f,h\}$ subject to the relations
    \[[e,f]=h, \qquad [h,e]=2e, \qquad [h,f]=-2f.\]
    The universal enveloping algebra $U(\mfsl_2)$ is thus the associative algebra generated by the symbols $\{e,f,h\}$ subject to the relations
    \[ef-fe=h, \qquad he-eh=2e, \qquad hf-fh=-2f.\]
    Since $(f,h,e)$ is an ordered basis of $\mfsl_2$, it follows that a basis for $U(\mfsl_2)$ is given by the monomials $f^{k_1}h^{k_2}e^{k_3}$ where each $k_i\in\N$.

    Consider the \emph{adjoint representation} where $V=\mfsl_2$ and $\phi(x)=\ad(x)$ for all $x\in\mfsl_2$, i.e., where the action of $\mfsl_2$ on itself is given by $x\cdot v=[x,v]$ for all $x,v\in\mfsl_2$.
    This is a highest-weight representation where the highest weight $\lambda\in\h^*$ is given by $\lambda(h)=2$:
    \begin{enumerate}
        \item $e\cdot e = 0$ by definition,
        \item $h\cdot e = 2e$ by the Chevalley--Serre relations,
        \item $e$, $f\cdot e = -h$, and $f^2\cdot e = -2f$ span $\mfsl_2$, hence $e$ generates $\mfsl_2$ as a representation.
    \end{enumerate}
    There are three weights of this representation: the linear functionals that send $h$ to $2$, $0$, and $-2$.
    This representation is finite-dimensional and irreducible, and hence isomorphic to the quotient module $L(\lambda)$ of the Verma module $M(\lambda)$ by Theorem \ref{T:hw-g}.
    We also have that $\lambda(h)\in\N$, as we would expect due to the second part of Theorem \ref{T:hw-g}.
\end{example}

Abusing notation, we will use $L(n)$ to denote the unique irreducible representation $L(\lambda)$ of $\mfsl_2$ where $\lambda(h)=n$.
Then the adjoint representation in the example above is $L(2)$.

\begin{example}\label{E:sl2-C2}
    Consider now the \emph{natural representation} of $\mfsl_2$ as the space of $2\times 2$ trace-zero matrices with basis
    \[e = \bmat{0 & 1 \\ 0 & 0}, \qquad f=\bmat{0 & 0 \\ 1 & 0}, \qquad h=\bmat{1 & 0 \\ 0 & -1}.\]
    This gives an action of $\mfsl_2$ on $\C^2$, and it is not hard to see that $v=(1,0)$ is a highest weight vector of weight $1$.
    Further, $f\cdot v = (0,1)$ is a vector of weight $-1$, and these two vectors $v$ and $f\cdot v$ span $\C^2$, so we see that this is the unique irreducible representation $L(1)$.
\end{example}


\section{Yangians}

\subsection{Algebra structure}

We now review the definition of the Yangian associated to a simple Lie algebra $\g$, and some of its properties.

\begin{definition}\label{D:Y}
    Fix some nonzero complex number $\hbar\in\C^\times$.
    The \emph{Yangian} $\Yhg$ is the unital associative algebra over $\C$ with generators $\{\xi_{i,r}, x^\pm_{i,r}\}_{i\in\I,r\in\N}$ subject to the following relations:
    \begin{enumerate}
        \item $[\xi_{i,r},\xi_{j,s}] = 0$,
        \item $[x^+_{i,r},x^-_{j,s}] = \delta_{ij}\xi_{i,r+s}$,
        \item $[\xi_{i,0},x^\pm_{j,s}] = \pm d_ia_{ij}x^\pm_{j,s}$,
        \item $[\xi_{i,r+1},x^\pm_{j,s}]-[\xi_{i,r},x^\pm_{j,s+1}] = \pm\hbar\dfrac{d_ia_{ij}}{2}(\xi_{i,r}x^\pm_{j,s}+x^\pm_{j,s}\xi_{i,r})$,
        \item $[x^\pm_{i,r+1},x^\pm_{j,s}]-[x^\pm_{i,r},x^\pm_{j,s+1}] = \pm\hbar\dfrac{d_ia_{ij}}{2}(x^\pm_{i,r}x^\pm_{j,s}+x^\pm_{j,s}x^\pm_{i,r})$,
        \item $\displaystyle\sum\limits_{\pi\in S_m}[x^\pm_{i,r_{\pi(1)}}, [x^\pm_{i,r_{\pi(2)}}, [\cdots[x^\pm_{i,r_{\pi(m)}}, x^\pm_{j,s}]\cdots]]]=0 \quad\text{if } i\neq j$.
    \end{enumerate}
    where in the final relation $m=1-a_{ij}$ and $S_m$ denotes the symmetric group of degree $m$.
\end{definition}

Comparing these relations to the Chevalley--Serre relations of Theorem \ref{T:Serre}, we see that the first three are very similar.
Indeed, the generators $\xi_{i,r}$ behave similarly to the Cartan generators $h_i$ of $\g$, and the generators $x^+_{i,r}$ and $x^-_{i,r}$ can be though of as raising and lowering operators like the generators $e_i$ and $f_i$ of $\g$, respectively.
There is an analogue of the PBW theorem (Theorem \ref{T:PBW-g}) for the Yangian, which states that monomials in these generators (in any order) give a basis for $\Yhg$.
One consequence of this theorem is that the Yangian has a triangular decomposition: there is a vector space isomorphism
\[\Yhg \cong \Yhg[-]\otimes\Yhg[0]\otimes\Yhg[+]\]
where $\Yhg[\pm]$ and $\Yhg[0]$ are the subalgebras generated by the elements $x^\pm_{i,r}$ and $\xi_{i,r}$ for all $i\in\I$ and $r\in\N$, respectively.
As another consequence, there is an injective algebra homomorphism $\Ug\inj\Yhg$ given by
\[e_i\mapsto d_i^{-1/2}x^+_{i,0}, \qquad f_i\mapsto d_i^{-1/2}x^-_{i,0}, \qquad h_i\mapsto d_i^{-1/2}\xi_{i,0}\]
thus we can view $\Ug$ (and also $\g$ itself) as being a subalgebra of $\Yhg$.

The Yangian also has the structure of a filtered algebra: let the generators $x^\pm_{i,r}$ and $\xi_{i,r}$ have degree $r$ for all $i\in\I$ and $r\in\N$.
This gives us a sequence of vector subspaces
\[\{0\}\subset F_0\subseteq F_1\subseteq\cdots\subseteq \Yhg\]
whose union is $\Yhg$ and that have the property that $F_m\cdot F_n\subseteq F_{m+n}$ for all $m,n\in\N$.
As a generalization of the embedding of $\Ug$ into $\Yhg$, there is an isomorphism of graded algebras
\[U(\g[t]) \cong \gr(\Yhg) := \bigoplus_{r\in\N}F_r/F_{r-1}\]
given by
\[e_it^r\mapsto d_i^{-1/2}\overline{x^+_{i,r}}, \qquad f_it^r\mapsto d_i^{-1/2}\overline{x^-_{i,r}}, \qquad h_it^r\mapsto d_i^{-1/2}\overline{\xi_{i,r}}\]
where for each generator $y\in\{x^\pm_i,\xi_i\}$, $\overline{y_r}$ denotes the image of $y_r$ in the quotient $F_r/F_{r-1}$, and $\g[t] := \g\otimes\C[t]$ is the \emph{current algebra} with Lie bracket given by
\[[x\otimes t^m, y\otimes t^n] = [x,y]\otimes t^{m+n}\]
for all $x,y\in\g$ and $m,n\in\N$.


\subsection{Hopf algebra structure}\label{ssec:Y-Hopf}

The Yangian is not only a unital associative algebra, but also a Hopf algebra.
In order to write down the Hopf algebra structure, we first note that $\Yhg$ is generated by the elements $x^\pm_{i,0}$, $\xi_{i,0}$, and $t_{i,1} := \xi_{i,1}-\frac{\hbar}{2}\xi_{i,0}^2$, for all $i\in\I$.
Indeed, we can recover the generators $x^\pm_{i,r}$ and $\xi_{i,r}$ for all $r\in\N$ inductively using the defining relations of the Yangian.
The counit $\varepsilon:\Yhg\to\C$, coproduct $\Delta:\Yhg\to\Yhg\otimes\Yhg$, and antipode $S:\Yhg\to\Yhg$ are thus determined by the Hopf algebra structure on $\Ug$: for all $y\in\{x^\pm_{i,0},\xi_{i,0} : i\in\I\}$, we have
\begin{gather*}
    \varepsilon(y) = 0, \\
    \Delta(y) = y\otimes 1 + 1\otimes y, \\
    S(y) = -y.
\end{gather*}
For the remaining generators $t_{i,1}$, the coproduct and antipode have an additional term: choose any $x^\pm_\alpha\in\g_{\pm\alpha}$ satisfying $(x^+_\alpha,x^-_\alpha)=1$, then for all $i\in\I$, we have
\begin{gather*}
    \varepsilon(t_{i,1}) = 0, \\
    \Delta(t_{i,1}) = t_{i,1}\otimes 1 + 1\otimes t_{i,1} - \hbar\sum_{\alpha\in\Phi^+}(\alpha_i,\alpha) x^-_\alpha \otimes x^+_\alpha, \\
    S(t_{i,1}) = -t_{i,1} - \hbar\sum_{\alpha\in\Phi^+}(\alpha_i,\alpha) x^-_\alpha x^+_\alpha,
\end{gather*}
where $\Phi^+$ is the set of positive roots of $\g$.

We can package the generators of $\Yhg$ into \emph{generating series} in a formal variable $u^{-1}$ as follows:
\begin{equation}\label{eqn:gen-series}
    \xi_i(u) := 1+\hbar\sum_{r\geq 0}\xi_{i,r}u^{-r-1}, \qquad x^\pm_i(u) := \hbar\sum_{r\geq 0}x^\pm_{i,r}u^{-r-1}
\end{equation}
for all $i\in\I$.
For each $a\in\C$, there is a Hopf algebra automorphism $\tau_a$ of $\Yhg$, called a \emph{shift automorphism}, uniquely determined by
\[\tau_a(y(u)) = y(u-a)\]
for all $y(u)\in\{\xi_i(u),x^\pm_i(u) : i\in\I\}$.
Replacing $a$ with a formal variable $z$ gives us an injective algebra homomorphism $\tau_z:\Yhg\inj\Yhg[][z]$.
This homomorphism was used by Drinfeld in \cite{drinfeld_hopf_1985} to define the \emph{universal $R$-matrix} $\rmat(z)$ of the Yangian, whose properties are captured in the following theorem.

\begin{theorem}\label{T:R}
    There is a unique element $\rmat(z)\in 1+z^{-1}\Yhg^{\otimes 2}\db{z^{-1}}$ satisfying the following identity in $\Yhg^{\otimes 2}\dparen{z^{-1}}$ for all $y\in\Yhg$:
    \begin{equation}\label{eqn:R-intertwiner}
        (\tau_z\otimes{\id})\Delta^{\mathrm{op}}(y) = \rmat(z) \cdot (\tau_z\otimes{\id})\Delta(y) \cdot \rmat(z)^{-1}
    \end{equation}
    and the following identities in $\Yhg^{\otimes 3}\db{z^{-1}}$:
    \begin{gather*}
        (\Delta\otimes{\id})(\rmat(z)) = \rmat_{13}(z)\rmat_{23}(z), \\
        ({\id}\otimes\Delta)(\rmat(z)) = \rmat_{13}(z)\rmat_{12}(z).
    \end{gather*}
    Moreover, $\rmat(z)^{-1} = \rmat_{21}(-z)$, and for all $a,b\in\C$:
    \[(\tau_a\otimes\tau_b)\rmat(z) = \rmat(z+a-b).\]
\end{theorem}

A proof of the above theorem was recently given in \cite{gautam_meromorphic_2021}, in which the universal $R$-matrix was reconstructed from its \emph{Gauss decomposition}:
\[\rmat(z) = \rmat^+(z)\rmat^0(z)\rmat^-(z)\]
where $\rmat^+(z) := \rmat^-_{21}(-z)^{-1}$, and $\rmat^-(z)\in(\Yhg[-]\otimes\Yhg[+])\db{z^{-1}}$ is of the form
\[\rmat^-(z) = \sum_{\beta\in Q^+}\rmat_\beta^-(z) \qquad\text{with}\qquad \rmat_\beta^-(z) \in(\Yhg[-]_{-\beta}\otimes\Yhg[+]_\beta)\db{z^{-1}}\]
where $Q^+$ is the positive cone of the root lattice of $\g$.
The components $\rmat_\beta^-(z)$ were constructed recursively in the height of the root $\beta$, and $\rmat_0^-(z)=1$; see \cite[\S 4.2]{gautam_meromorphic_2021}.
The ``diagonal factor'' $\rmat^0(z)$ is defined as the unique series in $1+z^{-1}\Yhg[0]^{\otimes 2}\db{z^{-1}}$ satisfying a certain formal difference equation; see \cite[\S6]{gautam_meromorphic_2021}.

Finally, one can use the universal $R$-matrix to define the \emph{deformed Drinfeld coproduct} on $\Yhg$ (see \cite[\S 3]{gautam_meromorphic_2021}), whose restriction to $\Yhg[0]$ allows us to put a Hopf algebra structure on $\Yhg[0]$ as follows:
\begin{gather*}
    \Delta_D(\xi_i(u)) = \xi_i(u)\otimes\xi_i(u), \\
    \varepsilon_D(\xi_i(u)) = 1, \\
    S_D(\xi_i(u)) = \xi_i(u)^{-1}.
\end{gather*}
We will call this the \emph{Drinfeld Hopf algebra structure} on $\Yhg[0]$.


\subsection{Alternate generators}\label{ssec:alt-gen}

Recall the generating series $\xi_i(u)$ and $x^\pm_i(u)$ defined above (\ref{eqn:gen-series}).
In \cite{gerasimov_class_2005}, Gerasimov et al. used these series to define another set of generating series for the Yangian which will prove useful to us in later sections.
In particular, there is a unique tuple of formal series $(A_j(u))_{j\in\I}\in (1+u^{-1}\Yhg[0]\db{u^{-1}})$ that satisfy the following relation for all $i\in\I$:
\[\xi_i(u) = \frac{\prod\limits_{j\neq i}\prod\limits_{r=1}^{-a_{ji}}A_j\bigl(u-\frac{\hbar d_j}{2}(2r-a_{ji})\bigr)}{A_i(u)A_i(u-\hbar d_i)},\]
and the coefficients of these series generate the subalgebra $\Yhg[0]$.

Using the series above, we can now define the following set of series:
\begin{gather*}
    B_i(u) = d_i^{1/2}A_i(u)x^+_i(u), \\
    C_i(u) = d_i^{1/2}x^-_i(u)A_i(u), \\
    D_i(u) = A_i(u)\xi_i(u) + C_i(u)A_i(u)^{-1}B_i(u).
\end{gather*}
The coefficients of $A_i(u)$, $B_i(u)$, and $C_i(u)$ generate $\Yhg$ as an algebra.
The following proposition \cite[Prop. 2.1]{gerasimov_class_2005} contains the commutation relations for these series that we will need.

\begin{proposition}\label{P:GKLO}
    The series $A_i(u)$, $B_i(u)$, $C_i(u)$, and $D_i(u)$ satisfy the following relations:
    \begin{enumerate}
        \item For all $i,j\in\I$, we have
        \[[A_i(u),A_j(v)] = [B_i(u),B_i(v)] = [C_i(u),C_i(v)] = 0.\]
        \item For all $i,j\in\I$ with $i\neq j$, we have
        \[[A_i(u),B_j(v)] = [A_i(u),C_j(v)] = [B_i(u),C_j(v)] = 0.\]
        \item For all $i,j\in\I$, we have
        \begin{align*}
            (u-v)[A_i(u),B_i(v)] &= d_i\hbar(B_i(u)A_i(v) - B_i(v)A_i(u)), \\
            (u-v)[A_i(u),C_i(v)] &= d_i\hbar(A_i(u)C_i(v) - A_i(v)C_i(u)), \\
            (u-v)[A_i(u),D_i(v)] &= d_i\hbar(B_i(u)C_i(v) - B_i(v)C_i(u)), \\
            (u-v)[B_i(u),C_i(v)] &= d_i\hbar(A_i(u)D_i(v) - A_i(v)D_i(u)), \\
            (u-v)[C_i(u),D_i(v)] &= d_i\hbar(D_i(u)C_i(v) - D_i(v)C_i(u)).
        \end{align*}
    \end{enumerate}
\end{proposition}

Now notice that the generator $e_i$ of $\g$, via the embedding of $\Ug$ into $\Yhg$, can be seen as the coefficient of $u^{-1}$ in the series $B_i(u)$.
Similarly, the generator $f_i$ is the coefficient of $u^{-1}$ in $C_i(u)$.
We can take the coefficient of $u^{-1}$ on both sides of the relations in the above proposition in order to obtain commutation relations of these generators with the other generating series.
Thus the above proposition encodes information about the adjoint action of $\g$ on $\Yhg$, which is captured in the following corollary.

\begin{corollary}\label{C:GKLO}
    The series $A_i(u)$, $B_i(u)$, $C_i(u)$, and $D_i(u)$ satisfy the following relations:
    \begin{enumerate}
        \item For all $i,j\in\I$ with $i\neq j$, we have
        \[[e_i,A_j(u)] = [f_i,A_j(u)] = 0.\]
        \item For all $i\in\I$, we have
        \begin{gather*}
            [e_i,A_i(u)] = B_i(u), \qquad [e_i,B_i(u)] = 0, \\
            [e_i,C_i(u)] = D_i(u)-A_i(u), \qquad [e_i,D_i(u)] = -B_i(u).
        \end{gather*}
        \item For all $i\in\I$, we have
        \begin{gather*}
            [f_i,A_i(u)] = -C_i(u), \qquad [f_i,B_i(u)] = A_i(u)-D_i(u), \\
            [f_i,C_i(u)] = 0, \qquad [f_i,D_i(u)] = C_i(u).
        \end{gather*}
    \end{enumerate}
\end{corollary}


\subsection{Representation theory}

We now review some basic representation theory of the Yangian, following \cite{chari_guide_1995}.
A \emph{representation} of $\Yhg$ is a module for $\Yhg$.
Similar to the case of a simple Lie algebra, we can define highest-weight representations for $\Yhg$ as follows.

\begin{definition}\label{D:hw-Y}
    Let $\ul=\{\lambda_{i,r}\in\C\}_{i\in\I,r\in\N}$ be a collection of complex numbers.
    A representation $V$ of $\Yhg$ is a \emph{highest-weight representation} with \emph{highest weight} $\ul$ if there exists $v\in V$ such that for all $i\in\I$ and $r\in\N$, the following hold:
    \begin{enumerate}
        \item $x^+_{i,r}\cdot v = 0$,
        \item $\xi_{i,r}\cdot v = \lambda_{i,r}v$,
        \item $\Yhg\cdot v = V$.
    \end{enumerate}
\end{definition}

We can define the \emph{Verma module} $M(\ul)$ for the Yangian by taking the quotient of $\Yhg$ by the left ideal generated by the elements $x^+_{i,r}$ and $\xi_{i,r}-\lambda_{i,r}1$ for all $i\in\I$ and $r\in\N$.
Again the Verma module has a unique maximal submodule and therefore a unique irreducible quotient module that we will denote $L(\ul)$.
The following theorem provides an analogue of the theorem of the highest weight (Theorem \ref{T:hw-g}) for the Yangian.

\begin{theorem}\label{T:hw-Y}
    \begin{enumerate}
        \item Every finite-dimensional irreducible representation of $\Yhg$ is isomorphic to $L(\ul)$ for some $\ul$.
        \item $L(\ul)$ is finite-dimensional if and only if there exist a tuple of monic polynomials (called \emph{Drinfeld polynomials}) $P_i(u)\in\C[u]$ such that
        \[\frac{P_i(u+d_i\hbar)}{P_i(u)} = 1+\hbar\sum_{r\geq 0}\lambda_{i,r}u^{-r-1}\]
        for all $i\in\I$, where we take the Laurent expansion of the left-hand side of the above equation about $u=\infty$.
    \end{enumerate}
\end{theorem}

We will denote the representation of $\Yhg$ with Drinfeld polynomials $\uP = (P_i(u))_{i\in\I}$ by $L(\uP)$.
Finally, we give an example to illustrate the above theorem.

\begin{example}\label{E:Y(sl2)}
    Consider $Y_\hbar(\mfsl_2)$.
    This algebra is generated by $\{\xi_r,x^\pm_r\}_{r\in\N}$, where we have dropped the first subscript $i$ since in this case $\I$ is a singleton set.
    Following \cite[\S 12.1]{chari_guide_1995}, we can obtain representations of $Y_\hbar(\mfsl_2)$ by extending the finite-dimensional irreducible representations $L(n)$ of $\mfsl_2$.
    Every such representation has a basis $\{v_0,\dots,v_n\}$ on which the action of $\mfsl_2$ is given by
    \[e\cdot v_m = (n-m+1)v_{m-1}, \qquad f\cdot v_m = (m+1)v_{m+1}, \qquad h\cdot v_m = (n-2m)v_m\]
    for each $m=0,1,\dots,n$, where we set $v_{-1}=v_{n+1}=0$.
    One can prove that for every $a\in\C$, we get a representation $L(n)_a$ of $Y_\hbar(\mfsl_2)$ (called an \emph{evaluation representation}) where the action on the same basis as above is given by
    \begin{align*}
        x^+_r\cdot v_m &= \left(a+\frac{1}{2}n-m+\frac{1}{2}\right)^r(n-m+1)v_{m-1}, \\
        x^-_r\cdot v_m &= \left(a+\frac{1}{2}n-m-\frac{1}{2}\right)^r(m+1)v_{m+1}, \\
        \xi_r\cdot v_m &= \left(\left(a+\frac{1}{2}n-m-\frac{1}{2}\right)^r(n-m)(m+1) -\left(a+\frac{1}{2}n-m+\frac{1}{2}\right)^r(n-m+1)m\right) v_m.
    \end{align*}

    Recall the natural representation $L(1)=\C^2$ of Example \ref{E:sl2-C2}, with basis $v_0=(1,0)$ and $v_1=(0,1)$.
    Setting $n=1$ and $m=0,1$ in the formulas above, we get a representation $L(1)_a$ of $Y_\hbar(\mfsl_2)$ where
    \begin{align*}
        \begin{split}
            x^+_r\cdot v_0 &= 0, \\
            x^-_r\cdot v_0 &= a^rv_1, \\
            \xi_r\cdot v_0 &= a^rv_0,
        \end{split}
        \begin{split}
            x^+_r\cdot v_1 &= a^rv_0, \\
            x^-_r\cdot v_1 &= 0, \\
            \xi_r\cdot v_1 &= -a^rv_1.
        \end{split}
    \end{align*}
    hence the representation map on each Yangian generator is given explicitly by:
    \[x^+_r\mapsto\bmat{0 & a^r \\ 0 & 0}, \qquad x^-_r\mapsto\bmat{0 & 0 \\ a^r & 0}, \qquad \xi_r\mapsto\bmat{a^r & 0 \\ 0 & -a^r}.\]
    and we see that in the case of $r=0$ we recover the action of $\mfsl_2$ on $\C^2$ that we started with.
    Here, $v_0$ is again a highest weight vector and so by the above theorem $L(1)_a$ must be the unique finite-dimensional irreducible representation $L(\ul)$ where $\ul$ is defined by $\lambda_r=a^r$ for all $r\in\N$.
    By the second part of the theorem, this representation has a Drinfeld polynomial since it is finite-dimensional.
    We will show that $P(u)=u-a$ is the polynomial we are looking for: we have
    \[\frac{P(u+\hbar)}{P(u)} = \frac{u+\hbar-a}{u-a} = 1+\hbar\frac{1}{u-a}.\]
    Now to expand $\frac{1}{u-a}$ about $u=\infty$, we replace $u$ with another variable $1/x$ and expand about $x=0$:
    \[\frac{1}{\frac{1}{x}-a} = \frac{x}{1-ax} = x\sum_{r\geq 0}(ax)^r = \sum_{r\geq 0}a^rx^{r+1} = \sum_{r\geq 0}a^ru^{-r-1}.\]
    Thus $P(u)$ satisfies the equation in the theorem.
\end{example}
