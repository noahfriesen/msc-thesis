\chapter{Yangians}

\section{Simple Lie algebras}

\subsection{Structure}

To begin, we will review some basic structure and representation theory of a simple Lie algebra, following \cite{humphreys_introduction_1972}.
This will allow us to more easily see the many similarities that arise in the Yangian setting, detailed in the next section.

Let $\g$ be a finite-dimensional simple Lie algebra over $\C$, with Cartan matrix $(a_{ij})_{i,j\in\I}$.
Fix minimal symmetrizing integers $d_i\in\{1,2,3\}$ so that $d_ia_{ij}=d_ja_{ji}$.
Let $\{\alpha_i\}_{i\in\I}$ be a basis of simple roots relative to a Cartan subalgebra $\h\subset\g$, and define a nondegenerate symmetric bilinear form on $\h^{*}$ by $(\alpha_i,\alpha_j)=d_ia_{ij}$.

The following theorem gives a presentation of $\g$ in terms of so-called \emph{Chevalley--Serre} generators and relations.
Note that $\delta_{ij}$ denotes the Kronecker delta function, which is $1$ if $i=j$ and $0$ otherwise, and $\ad:\g\to\End(\g)$ is defined by $\ad(x)(y)=[x,y]$ for $x,y\in\g$.

\begin{theorem}[Serre]\label{T:Serre}
    $\g$ is generated by the elements $\{e_i,f_i,h_i\}_{i\in\I}$ subject to the following relations:
    \begin{enumerate}
        \item $[h_i,h_j] = 0$,
        \item $[e_i,f_j] = \delta_{ij}h_i$,
        \item $[h_i,e_j] = a_{ij}e_j, \quad [h_i,f_j] = -a_{ij}f_j$,
        \item $\ad(e_i)^{1-a_{ij}}(e_j) = \ad(f_i)^{1-a_{ij}}(f_j) = 0 \quad\text{if } i\neq j$.
    \end{enumerate}
\end{theorem}

This presentation is especially useful for studying the representation theory of $\g$.
Before discussing representations, we recall the \emph{universal enveloping algebra} $\Ug$, which is the associative algebra generated by elements of $\g$ in which the commutator corresponds to the Lie bracket of $\g$.
More precisely, $\Ug$ is the quotient of the tensor algebra
\[T(\g) := \C \oplus \g \oplus (\g\otimes\g) \oplus \cdots\]
by the two-sided ideal generated by the elements
\[x\otimes y - y\otimes x - [x,y]\]
for all $x,y\in\g$.
The following theorem gives us a description of $\Ug$ in terms of $\g$ itself, which has many useful consequences.

\begin{theorem}[Poincar\'e--Birkhoff--Witt]\label{T:PBW-g}
    If $x_1,\cdots,x_n$ is an ordered basis for $\g$, then $\{x_1^{k_1}\cdots x_n^{k_n} : k_i\in\N\}$ is a basis for $\Ug$.
\end{theorem}

Recall that $\g$ has a \emph{triangular decomposition}:
\[\g \cong \n^-\oplus\h\oplus\n^+,\]
where $\n^-$ and $\n^+$ are the subalgebras of $\g$ generated by the elements $f_i$ and $e_i$ for all $i\in\I$, respectively.
A consequence of the above theorem is that $\Ug$ also has a triangular decomposition: there is a vector space isomorphism
\[\Ug \cong U(\n^-)\oplus U(\h) \oplus U(\n^+).\]


\subsection{Representation theory}

A \emph{representation} of $\g$ is a vector space $V$ together with a linear map $\phi:\g\to\End(V)$ satisfying
\[\phi([x,y]) = \phi(x)\phi(y)-\phi(y)\phi(x)\]
for all $x,y\in\g$.
In other words, the map $\phi$ is a Lie algebra homomorphism from $\g$ to $\mfgl(V)$, where the latter is the endomorphism space $\End(V)$ with Lie bracket given by the commutator of endomorphisms.
Equivalently, a representation of $\g$ is a vector space $V$ together with a bilinear ``action'' map $\cdot:\g\times V\to V$ satisfying
\[[x,y]\cdot v = x\cdot(y\cdot v) - y\cdot(x\cdot v)\]
for all $x,y\in\g$ and $v\in V$.
From this definition, we see that representations of $\g$ correspond to modules over $\Ug$ and vice versa:
\[(xy)\cdot v = x\cdot (y\cdot v).\]

We would like to classify the finite-dimensional irreducible representations of $\g$.
In order to do so, we first consider a class of representations known as \emph{highest-weight} representations.

\begin{definition}\label{D:hw-g}
    $V$ is a \emph{highest-weight representation} with \emph{highest weight} $\lambda\in\h^*$ if there exists $v\in V$ such that for all $i\in\I$, the following hold:
    \begin{enumerate}
        \item $e_i\cdot v = 0$,
        \item $h_i\cdot v = \lambda(h_i)v$,
        \item $\Ug\cdot v = V$.
    \end{enumerate}
\end{definition}

The second condition in the above definition is equivalent to saying that the highest weight vector $v$ belongs to the \emph{weight space} of weight $\lambda$, which is defined as follows:
\[V_\lambda := \{v\in V : h\cdot v = \lambda(h)v \quad \forall h\in\h\}.\]
The sum of weight spaces $V_\mu$ over all $\mu\in\h^*$ is always a direct sum, and is always a subrepresentation of $V$.
In the case where $V$ is finite-dimensional, this subrepresentation is equal to $V$.

Another important thing to note is that the action of $\g$ permutes the weight spaces of $V$: for all $\mu\in\h^*$ and $i\in\I$, the generators $e_i$ and $f_i$ map $V_\mu$ into $V_{\mu+\alpha_i}$ and $V_{\mu-\alpha_i}$, respectively.
Thus we can think of the generators $e_i$ and $f_i$ as ``raising operators'' and ``lowering operators,'' respectively.
From this, along with the PBW theorem (Theorem \ref{T:PBW-g}), it follows that the weights of $V$ (that is, the linear functionals $\mu\in\h^*$ for which $V_\mu$ is nonempty) are all of the form
\[\mu = \lambda - \sum_{i\in\I} k_i\alpha_i\]
for some $k_i\in\N$.
This provides motivation for using the term ``highest weight'' in the above definition.

Next, we recall the definition of the \emph{Verma module} $M(\lambda)$ of highest weight $\lambda\in\h^*$, which is in some sense the ``largest'' highest-weight representation of $\g$ of a given highest weight.
To construct the Verma module, we make $\Ug$ into a highest-weight representation of $\g$ by taking a quotient in order to impose only the conditions required by Definition \ref{D:hw-g}.
Hence $M(\lambda)$ is the quotient of $\Ug$ by the left ideal generated by the elements $e_i$ and $h_i-\lambda(h_i)1$ for all $i\in\I$.
This representation has a unique maximal subrepresentation, and therefore a unique irreducible quotient representation that we will denote $L(\lambda)$.

The following theorem (known as the \emph{theorem of the highest weight}) completely classifies the finite-dimensional irreducible representations of $\g$ up to isomorphism.

\begin{theorem}\label{T:hw-g}
    \begin{enumerate}
        \item Every finite-dimensional irreducible representation of $\g$ is isomorphic to $L(\lambda)$ for some $\lambda\in\h^*$.
        \item $L(\lambda)$ is finite-dimensional if and only if $\lambda(h_i)\in\N$ for all $i\in\I$.
    \end{enumerate}
\end{theorem}

Finally, we give an example to better illustrate some of the concepts above.

\begin{example}\label{E:sl2}
    Consider the special linear Lie algebra $\mfsl_2$, which has $1\times 1$ Cartan matrix $[2]$.
    By Theorem \ref{T:Serre}, $\mfsl_2$ is generated by $\{e,h,f\}$ subject to the relations
    \[[e,f]=h, \quad [h,e]=2e, \quad [h,f]=-2f.\]
    The universal enveloping algebra $U(\mfsl_2)$ is thus the associative algebra generated by the symbols $\{e,h,f\}$ subject to the relations
    \[ef-fe=h, \quad he-eh=2e, \quad hf-fh=-2f.\]
    Since $(f,h,e)$ is an ordered basis of $\mfsl_2$, it follows that a basis for $U(\mfsl_2)$ is given by the monomials $f^{k_1}h^{k_2}e^{k_3}$ where each $k_i\in\N$.

    Consider the \emph{adjoint representation} where $V=\mfsl_2$ and $\phi(x)=\ad(x)$ for all $x\in\mfsl_2$, i.e., where the action of $\mfsl_2$ on itself is given by $x\cdot v=[x,v]$ for all $x,v\in\mfsl_2$.
    This is a highest-weight representation where the highest weight $\lambda\in\h^*$ is given by $\lambda(h)=2$:
    \begin{enumerate}
        \item $e\cdot e = 0$ by definition,
        \item $h\cdot e = 2e$ by the Chevalley--Serre relations,
        \item $e$, $f\cdot e = -h$, and $f^2\cdot e = -2f$ span $\mfsl_2$, hence $e$ generates $\mfsl_2$ as a representation.
    \end{enumerate}
    There are three weights of this representation: the linear functionals that send $h$ to $2$, $0$, and $-2$.
    This representation is finite-dimensional and irreducible, and hence isomorphic to the quotient module $L(\lambda)$ of the Verma module $M(\lambda)$ by Theorem \ref{T:hw-g}.
    We also have that $\lambda(h)\in\N$, as we would expect due to the second part of Theorem \ref{T:hw-g}.
\end{example}


\section{Yangians}

\subsection{Structure}

We give the generators and relations of the Yangian associated to a simple Lie algebra.

\subsection{Representation theory}

We review the classification of the irreducible representations of the Yangian by Drinfeld polynomials.
