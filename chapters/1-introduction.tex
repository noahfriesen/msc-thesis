\chapter{Introduction}

\section{Background}

In the 1960s, the \emph{quantum Yang--Baxter equation} was first introduced in the theory of integrable systems.
The equation emerged from attempts to solve two-dimensional lattice models, in which atoms are arranged in a rectangular grid, and the energy of each atom depends on the bonds joining it to its four neighbouring atoms.
In such a model, the quantum Yang--Baxter equation provides a condition that guarantees the existence of operators that allow the system to be solved.
The equation itself is as follows:
\[\rmat_{12}(u-v)\rmat_{13}(u)\rmat_{23}(v) = \rmat_{23}(v)\rmat_{13}(u)\rmat_{12}(u-v),\]
where $\rmat$ is a matrix that describes the possible configurations of bonds between atoms, and $u$ and $v$ are external parameters on which the energies of atoms may also depend, such as magnetic fields \cite[\S 7.5]{chari_guide_1995}.

Physicist R. J. Baxter used the quantum Yang--Baxter equation in his famous paper \cite{baxter_partition_1972} to solve a lattice model known as the eight-vertex model.
It was also in this paper that he first encountered a remarkable class of polynomials now known as \emph{Baxter polynomials}.
These polynomials encode fundamental information about physical models, and have many applications to areas of mathematics including geometry and representation theory.
In this thesis, we are concerned with certain Baxter polynomials obtained from mathematical objects called \emph{Yangians}.

In the 1980s, Fields medalist V. Drinfeld introduced Yangians as a uniform context in which to study another special case of the above lattice models.
In particular, they could be used to generate solutions of the quantum Yang--Baxter equation.
Many interesting and surprising phenomena appearing in the physical models could also be explained algebraically using the structure of these objects \cite{drinfeld_hopf_1985}.
Since then, the study of Yangians has become an important topic in both mathematics and physics.

More recently, the representation theory of Yangians has been a rich and active area of research.
There are various symmetries that occur in this setting, which have been studied previously by V. Chari and Y. Tan.
From the same data used to construct a Yangian, one may also construct another object called a \emph{braid group}; these symmetries appear in an action of this group on certain functions \cite{chari_braid_2002,tan_braid_2015}.
This thesis provides a direct algebraic explanation of where these symmetries come from: an action of the braid group on a certain part of the Yangian itself.


\section{Braid groups and Yangians}

Let us now turn to describing the content of this thesis in more detail.
Given a simple Lie algebra $\g$ over $\C$, we may consider its associated braid group $\Bg$.
The generators of this group satisfy the same braid relations as those of the Weyl group $\Wg$, but the generators of $\Bg$ are all of infinite order.
When $\g = \mfsl_n$, the braid group $\Bg$ coincides with Artin's braid group on $n$ strands, and one can visualize elements of this group quite well using diagrams of physical braids.
In this case, imposing the additional relation that the generators of the group have order $2$ gives us the symmetric group $S_n$, which is the Weyl group of $\mfsl_n$.

Much like the Weyl group, the braid group plays an important role in the representation theory of $\g$.
More precisely, it is well known that the generator $\braid_i$ of $\Bg$ acts on integrable representations $V$ of $\g$ as
\[\exp(e_i)\exp(-f_i)\exp(e_i)\]
where $e_i$ and $f_i$ are the usual Chevalley generators of $\g$ and $i\in\I$ indexes the Cartan matrix (or equivalently, the nodes of the Dynkin diagram) of $\g$.
The integrability of $V$ is needed so that $e_i$ and $f_i$ act nilpotently on every vector of $V$, so that the exponentials above truncate after finitely many terms.
An important property of this braid group action is that it sends a vector in $V$ of weight $\lambda$ to one of weight $w(\lambda)$ for some $w$ in the Weyl group.
Moreover, this action actually preserves the dimension of the weight spaces, and so all of the weight spaces with weight of the form $w(\lambda)$ are isomorphic.
We make use of this property in Chapter \ref{chap:weights} to prove some key lemmas.

For our purposes, the most important integrable representation of $\g$ is an infinite-dimensional Hopf algebra called the \emph{Yangian}, denoted $\Yhg$.
This algebra is a type of affine quantum group, and is a deformation of the universal enveloping algebra of the current algebra $\g[t]$.
Yangians originally arose in the theory of integrable systems, and their representation theory has been studied by many since the seminal paper \cite{drinfeld_hopf_1985} of V. Drinfeld.
They also have many applications to the areas of geometry and mathematical physics, and their name honours physicist C. N. Yang.

In Chapter \ref{chap:yangians}, after reviewing some basic structure and representation theory of $\g$, we recall a presentation for the Yangian known as \emph{Drinfeld's new presentation}.
The generators and relations therein are very reminiscent of the Chevalley--Serre presentation of $\g$ itself.
Indeed, using this presentation it is easy to see that there is a natural inclusion of $\g$ and its enveloping algebra into $\Yhg$ given by sending the generators of the former to the degree-zero generators of the latter.
Because of this inclusion, we get an adjoint action of $\g$ on $\Yhg$ which is integrable, and so there is an action of the braid group $\Bg$ on $\Yhg$.
This action has appeared in a number of contexts; see for example \cite{guay_coproduct_2018, kodera_braid_2019, weekes_highest_2016}.
Also note that because of this, we may regard any representation of $\Yhg$ as a representation of $\g$.

In a later section of Chapter \ref{chap:yangians}, we introduce an alternative set of generating series for the Yangian given by A. Gerasimov et al. in \cite{gerasimov_class_2005}.
These generating series play a crucial role in our results, as they allow us to compute explicit formulas for the action of the braid group on the Yangian.
We are then able to compute formulas for our modified braid group action described in the section below, and ultimately use these formulas to find a new factorization of the so-called \emph{Baxter polynomials}, giving us our main results.


\section{Modified braid group action}

One of the main motivations for this research was to develop a framework for better understanding an action of the braid group $\Bg$ on tuples of rational functions defined in the work \cite{tan_braid_2015} of Y. Tan.
Drawing inspiration from the results of V. Chari for quantum loop algebras given in \cite{chari_braid_2002}, Tan defined operators directly and proved that they indeed give an action of $\Bg$.
In Chapters \ref{chap:braidgroup} and \ref{chap:weights}, we recover Tan's action starting with the action of $\Bg$ on the Yangian $\Yhg$ described in the previous section, and in doing so prove that Tan's action computes the eigenvalues of certain generators of $\Yhg$ on the \emph{extremal weight spaces} of a representation of $\Yhg$---weight spaces that lie in the Weyl group orbit of the highest weight space.

We now describe the key results of Chapter \ref{chap:braidgroup}.
Much like the Lie algebra $\g$, the Yangian $\Yhg$ has a triangular decomposition: one can partition the generators into raising, lowering, and Cartan operators and consider the spaces generated by each of these.
We denote by $\Yhg[0]$ the commutative subalgebra of the Yangian generated by the Cartan generators $\xi_{i,r}$ for all $i\in\I$ and $r\in\N$.
Note that this subalgebra deforms the universal enveloping algebra of the current algebra $\h[t]$ where $\h$ is a Cartan subalgebra of $\g$.
We may package the generators of $\Yhg[0]$ into series $\xi_i(u)$ in a formal variable $u^{-1}$ so that the generator $\xi_{i,r}$ is the coefficient of $u^{-r-1}$; there is a unique Hopf algebra structure on $\Yhg[0]$ such that each of these series is grouplike.
Using properties of the triangular decomposition of $\Yhg$, we show that there is a natural linear projection
\[\Pi:\Yhg\to\Yhg[0]\]
and we use this projection to introduce what we call the \emph{modified braid group operators}:
\[\mbraid_i := \Pi\circ\braid_i\big|_{\Yhg[0]}.\]
By the weight-permuting property of the braid group action on representations of $\g$ described in the previous section, the braid group action maps elements of $\Yhg[0]$ to elements of $\Yhg$ which are sums of monomials containing the same number of raising and lowering operators.
Then applying the projection $\Pi$ discards terms that are not in $\Yhg[0]$, so the above operators are endomorphisms of $\Yhg[0]$.
In fact, they have many remarkable properties which we summarize in the theorem below.
This theorem constitutes the first main result of the thesis.

\begin{theorem}\label{T:intro-main1}
    The modified braid group operators $\mbraid_i$ have the following properties:
    \begin{enumerate}
        \item They are Hopf algebra automorphisms of $\Yhg[0]$ that satisfy the braid relations, i.e., they define an action of $\Bg$ on $\Yhg[0]$.
        \item They are uniquely determined by the following formulas, for each $j\in\I$:
        \[\mbraid_i(\xi_j(u)) = \xi_j(u)\prod_{k=0}^{\abs{a_{ij}}-1} \xi_i\left(u-\frac{\hbar d_i}{2}(\abs{a_{ij}}-2k)\right)^{(-1)^{\delta_{ij}}}\]
        where $a_{ij}$ and $d_i$ are the entries of the Cartan matrix and the symmetrizing integers of $\g$, respectively.
        \item The diagonal factor $\rmat^0(z)\in \Yhg[0]^{\otimes 2}\db{z^{-1}}$ of the universal $R$-matrix of $\Yhg$ is $\Bg$-invariant:
        \[(\mbraid_i\otimes\mbraid_i)(\rmat^0(z)) = \rmat^0(z).\]
    \end{enumerate}
\end{theorem}

Let us now turn to the results of Chapter \ref{chap:weights}, in which we recover Tan's braid group action.
Using the modified braid group operators above, we define an action of $\Bg$ on the linear dual of $\Yhg[0]$.
The group of algebra homomorphisms $\Yhg[0]\to\C$ is a subrepresentation, and this group is isomorphic to the group $(1+u^{-1}\C\db{u^{-1}})^\I$ of $\I$-tuples of formal series in $u^{-1}$ with constant coefficient $1$.
To see this, notice that we can define an algebra homomorphism $\Yhg[0]\to\C$ by choosing a complex number for each generator $\xi_{i,r}$ to be mapped to; applying this homomorphism to the generating series $\xi_i(u)$ gives the desired isomorphism.
The formula for this dual action of $\Bg$ on a tuple of series $(\lambda_i(u))_{i\in\I}$ is the same as the formula in the second part of the theorem above, replacing $\xi$ with $\lambda$.
It is also the same formula as the one for the action of $\Bg$ on the group $(\C(u)^\times)^\I$ of $\I$-tuples of rational functions defined by Tan.

In Section \ref{sec:extending}, we reconcile the underlying representation spaces of our dual braid group action and Tan's.
We do this by extending our dual action to the group $M^\I$ of $\I$-tuples of monic Laurent series using the theory of formal additive difference equations.
We then further extend to the group $(\C\dparen{u^{-1}}^\times)^\I$ of $\I$-tuples of arbitrary Laurent series by taking the product of $M^\I$ and the group $(\C^\times)^\I$ on which there is a natural action of the Weyl group $\Wg$.
The group of rational functions on which Tan's action is defined is then recovered as a subrepresentation of this largest space.

In Section \ref{sec:extremal-weights}, we show that the dual braid group action determines the weights of the \emph{extremal vectors} of a representation of the Yangian.
Finite-dimensional irreducible representations of $\Yhg$ are classified in a way very similar to those of $\g$ in that they are determined by highest weights: for such a representation $V$ there is a \emph{highest weight vector} $v\in V$ that is killed by the raising operators and is an eigenvector for the Cartan operators, say $\xi_{i,r}\cdot v = \lambda_{i,r}v$ for some $\lambda_{i,r}\in\C$.
Packaging these eigenvalues into series as we did for the generators themselves and collecting them into a tuple, we let $\ul = (\lambda_i(u))_{i\in\I}$ denote the highest weight of $V$.
Recall that we can regard $V$ as a representation of $\g$, and doing so we define the extremal vectors of $V$ to be those of $\g$-weight $w(\lambda)$ for $w\in\Wg$, where $\lambda$ is the $\g$-weight of the highest weight vector $v$.
Note that because of the aforementioned dimension-preserving property of the braid group action on weight spaces, each extremal weight space is one-dimensional.
The main result of this section tells us that the eigenvalue of the generating series $\xi_i(u)$ (i.e., the $\Yhg$-weight) on the extremal vector of $\g$-weight $w(\lambda)$ is given by the dual braid group action on the $\Yhg$-highest weight.
This eigenvalue is $\braid_w(\ul)_i$, where $\braid_w\in\Bg$ is defined by taking any reduced expression for $w\in\Wg$ and replacing each Weyl group generator with its corresponding braid group generator, and the subscript $i$ denotes taking the $i$th component of the tuple.

Something to note (and that is explained in detail in \cite[\S 6]{friesen_braid_2024}) is that the results of this section also apply to the parallel case of the quantum loop algebra $\UqLg$.
In this setting, there is an action of $\Bg$ on $\UqLg$ via operators given in \cite{lusztig_introduction_2010} by G. Lusztig, and we can modify these operators as above to obtain an action on a commutative subalgebra $\UqLg[0]$.
Dualizing this modified action, we recover an action of $\Bg$ studied by Chari in \cite{chari_braid_2002}.
We can then prove the analogue of Theorem \ref{T:intro-main1} in this setting, and further we can show that this modified action is compatible with the homomorphism from $\UqLg$ to a completion of $\Yhg$ that was given in \cite{gautam_yangians_2013}.
By this, we mean that modifying the action on $\UqLg$ then applying the homomorphism gives us the same action on $\Yhg$ that we obtain by modifying the action on $\Yhg$.


\section{Baxter polynomials and cyclicity}

The other main motivation for studying these braid group actions arises from the following fact: the tensor product of two finite-dimensional irreducible representations of the Yangian will almost always be cyclic, i.e., highest weight.
This has been studied extensively and has applications in many areas; see for example \cite{chari_yangians_1996, guay_local_2015, molev_yangians_2007, nazarov_irreducibility_2002, akasaka_finite_1997}.
In the literature, there are two sufficient conditions for this cyclicity property.
The first of these was established by Tan in \cite{tan_braid_2015} using a construction that appeared earlier for the quantum loop algebra in \cite{chari_braid_2002}, and involves the braid group action on rational functions described in the previous section.
The second condition was established by S. Gautam and C. Wendlandt in \cite{gautam_poles_2023} and involves the \emph{Baxter polynomials}.
It was conjectured in the latter paper that these two conditions are identical, and we prove in Chapter \ref{chap:baxter} by giving a new factorization of the Baxter polynomials that this is indeed the case.

Let us describe the first cyclicity condition of \cite{tan_braid_2015}.
Finite-dimensional irreducible representations of $\Yhg$ are parametrized by $\I$-tuples of monic polynomials $\uP=(P_i(u))_{i\in\I}$ called \emph{Drinfeld polynomials} which encode the highest weight.
We denote by $L(\uP)$ the representation of $\Yhg$ with Drinfeld polynomials $\uP$.
Choose any reduced expression $w_0 = s_{j_1}\cdots s_{j_p}$ for the longest element of the Weyl group $\Wg$, and let $w_r\in\Wg$ be the element of the Weyl group obtained by removing $r$ generators from the left of this reduced expression.
Then the representation $L(\uP)\otimes L(\uQ)$ is cyclic if
\[\zeros(Q_{j_r}(u+\hbar d_{j_r}))\subset\C\setminus\zeros(\braid_{w_r}(\uP)_{j_r})\]
for each $r$, where $\zeros(p(u))$ denotes the set of zeros of a polynomial $p(u)$.
I.e., the tensor product is cyclic as long as certain Drinfeld polynomials of the right tensor leg and certain polynomials arising from the braid group action on the Drinfeld polynomials of the left tensor leg do not have zeros in common.

Now we will describe the second cyclicity condition of \cite{gautam_poles_2023}.
Similar to how we defined the Cartan generating series $\xi_i(u)$ of $\Yhg$ in the previous section, we may do the same for the raising operators $x^+_i(u)$ and the lowering operators $x^-_i(u)$.
Each of these series acts on a representation $L(\uP)$ as the expansion at $u=\infty$ of some $\End(L(\uP))$-valued rational function in $u$.
If we denote the set of poles of these three rational functions by $\sigma_i(L(\uP))$, then the representation $L(\uP)\otimes L(\uQ)$ is cyclic if
\[\zeros(Q_i(u+\hbar d_i))\subset\C\setminus\sigma_i(L(\uP))\]
for all $i\in\I$.
I.e., the tensor product is cyclic as long as the zeros of the Drinfeld polynomials of the right tensor leg are not poles of the corresponding generating series of the left tensor leg.
Moreover, the tensor product is irreducible if the same condition also holds with $\uP$ and $\uQ$ interchanged.

It was proven in \cite{gautam_poles_2023} that the set of poles $\sigma_i(L(\uP))$ exactly coincides with the set of roots of a distinguished polynomial $\baxter_{i,L(\uP)}^\g(u)$ called the \emph{Baxter polynomial}.
These polynomials first appeared in generality in the work \cite{frenkel_baxters_2015} of E. Frenkel and D. Hernandez, though they have been studied in more specialized contexts for a long period of time.
Their name honours physicist R. J. Baxter, who first encountered polynomials of this type in his paper \cite{baxter_partition_1972}.
The motivation behind Baxter's original work, and much of the literature on Baxter polynomials, is based in mathematical physics, as these polynomials encode information about certain models describing physical phenomena.
For us, $\baxter_{i,L(\uP)}^\g(u)$ arises as the eigenvalue of a certain \emph{abelian transfer operator} on the lowest weight vector of $L(\uP)$.
In \cite{gautam_poles_2023}, a formula for this polynomial was given in terms of the Drinfeld polynomials $\uP$ and a matrix known as the \emph{quantum Cartan matrix} of $\g$.

In Section \ref{sec:baxter-extremal}, we use the results of Chapters \ref{chap:braidgroup} and \ref{chap:weights} to obtain a factorization of the Baxter polynomials in terms of polynomials arising from the action of the braid group on the Drinfeld polynomials.
Because of the discussion above---namely that the zeros of the Baxter polynomials are exactly the poles of the generating series---this factorization implies that the two cyclicity criteria for tensor products are identical.
The theorem below summarizes these findings and provides the second main result of the thesis.

\begin{theorem}\label{T:intro-main2}
    The Baxter polynomial $\baxter_{i,L(\uP)}^\g(u)$ admits the following factorization:
    \[\baxter_{i,L(\uP)}^\g(u) = \prod_{r:j_r=i}\braid_{w_r}(\uP)_i.\]
    Consequently, the sufficient conditions for the cyclicity of any tensor product $L(\uP)\otimes L(\uQ)$ obtained in \cite{tan_braid_2015} and \cite{gautam_poles_2023} are identical.
\end{theorem}

Something to note is that the above theorem actually provides a factorization for the Baxter polynomial associated to any extremal weight, not just the lowest weight.
Instead of the longest element $w_0$ in the above construction, we may take a reduced expression for any element $w\in\Wg$ and define the elements $w_r$ in the same way.

Lastly, in Section \ref{sec:baxter-example}, we explicitly compute Baxter polynomials for fundamental representations in the case where $\g=\mfsl_n$.
There are many symmetries which make the computation nicer in this case, and this section provides a good example of how one might make use of the main results of the thesis.
