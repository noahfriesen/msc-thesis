\chapter{Braid groups}

\section{Braid group action}

In this section, we review the action of the braid group of a simple Lie algebra $\g$ on its integrable representations, and prove some important properties.

The \emph{braid group} of $\g$, denoted $\Bg$, is the group with generators $\braid_i$ for $i\in\I$ subject to the following defining relations, called the \emph{braid relations}: for all $i,j\in\I$ with $i\neq j$, we have
\[\underbrace{\braid_i\braid_j\braid_i\cdots}_{m_{ij}\text{ factors}} = \underbrace{\braid_j\braid_i\braid_j\cdots}_{m_{ij}\text{ factors}}\]
where the number of factors $m_{ij}=m_{ji}$ on each side of the above equality is defined according to the entries of the Cartan matrix:
\[\begin{array}{c|cccc}
    a_{ij}a_{ji} & 0 & 1 & 2 & 3 \\
    \hline
    m_{ij} & 2 & 3 & 4 & 6
\end{array}\]

Recall that the Weyl group $\Wg$ associated to $\g$ is generated by the simple reflections $s_i$ for $i\in\I$, which satisfy the same relations as those of the braid group generators as well as the relation $s_i^2={\id}$ for all $i\in\I$.
Thus there is a surjective map $\Bg\surj\Wg$ defined by $\braid_i\mapsto s_i$, with kernel generated by the elements $\braid_i^2$ for all $i\in\I$.
This surjection has a section (i.e., a right inverse) $\Wg\to\Bg$ where $w\mapsto\braid_w$, defined by taking a reduced expression $w=s_{i_1}\cdots s_{i_\ell}$ and setting $\braid_w:=\braid_{i_1}\cdots\braid_{i_\ell}$.
It turns out that this is independent of the choice of reduced expression for $w$, and these elements satisfy the property $\braid_{vw}=\braid_v\braid_w$ whenever the lengths of the reduced expressions for $v$ and $w$ add: $\ell(vw)=\ell(v)+\ell(w)$.
% TODO: cite Lusztig here

As mentioned above, there is an action of the braid group on integrable representations of $\g$, which are defined below.

\begin{definition}\label{D:integrable}
    A representation $(V,\phi)$ of $\g$ is \emph{integrable} if the following two conditions hold:
    \begin{enumerate}
        \item $V$ is a direct sum of weight spaces $V_\mu$ where $\mu(h_i)\in\Z$ for all $i\in\I$,
        \item $\phi(e_i)$ and $\phi(f_i)$ are \emph{locally nilpotent} endomorphisms of $V$: for each $v\in V$ there exists a finite-dimensional subspace $W$ of $V$ such that $v\in W$, $W$ is invariant under $\phi(e_i)$ and $\phi(f_i)$, and the restrictions of $\phi(e_i)$ and $\phi(f_i)$ to $W$ are nilpotent.
    \end{enumerate}
\end{definition}

Given such a representation $(V,\phi)$, we can define the following operators for each $i\in\I$:
\[\braid_i^V := \exp(\phi(e_i))\exp(\phi(-f_i))\exp(\phi(e_i)).\]
Since $\phi(e_i)$ and $\phi(-f_i)$ are locally nilpotent endomorphisms of $V$, it follows by the argument in \cite[\S21.2]{humphreys_introduction_1972} that each $\braid_i^V$ is an automorphism of $V$.
One can check that these operators satisfy the braid relations, so the assignment $\braid_i\mapsto\braid_i^V$ gives us an action of the braid group on $V$.
% TODO: prove this, or give an example (maybe adjoint rep of sl3)

Recall that there is an action of the Weyl group on $\h^*$ given by
\[s_i(\alpha_j) = \alpha_j - a_{ij}\alpha_i\]
for all $i,j\in\I$.
The following proposition shows that the braid group action on $V$ intertwines the weight spaces of $V$ via the Weyl group action on the weights:

\begin{proposition}\label{P:tau-wt-space}
    For all $i\in\I$ and $\mu\in\h^*$, we have $\braid_i^V(V_\mu)=V_{s_i(\mu)}$.
\end{proposition}
\begin{proof}
    We use the argument shown in \cite[\S1.3]{kumar_kac-moody_2002}.
    Fix $v\in V_\mu$, then for any $h\in\h$ such that $\alpha_i(h)=0$, we have
    \[h\cdot \braid_i^V(v) = \mu(h)\braid_i^V(v).\]
    By the Serre relations and some properties of nilpotent derivations (namely using the Leibniz rule), we have
    \[(\braid_i^V)^{-1}\phi(h_i)\braid_i^V = -\phi(h_i)\]
    as endomorphisms of $V$.
    From this, it follows that
    \begin{align*}
        h_i\cdot\braid_i^V(v) &= -\braid_i^V(h_i\cdot v) \\
        &= -\braid_i^V(\mu(h_i)v) \\
        &= -\mu(h_i)\braid_i^V(v)
    \end{align*}
    which is sufficient to show that $\braid_i^V(v)$ is a vector of weight $s_i(\mu)$.
    One can similarly show that $(\braid_i^V)^{-1}(v)$ is also a vector of weight $s_i(\mu)$, and since $s_i^2={\id}$, this completes the proof.
\end{proof}

The following proposition shows another important property of these braid group operators.

\begin{proposition}\label{P:tau-tensor}
    If $(V,\phi_V)$ and $(W,\phi_W)$ are both integral representations of $\g$, then $\braid_i^{V\otimes W} = \braid_i^V\otimes\braid_i^W$ for all $i\in\I$.
\end{proposition}
\begin{proof}
    Let $\phi=\phi_V\otimes\phi_W$, so for all $x\in\g$ we have
    \[\phi(x) = \phi_V(x)\otimes{\id} + {\id}\otimes\phi_W(x).\]
    Notice that the two terms of the right-hand side of the above equation commute, since
    \[(\phi_V(x)\otimes{\id})(\id\otimes\phi_W(x)) = ({\id}\otimes\phi_W(x))(\phi_V(x)\otimes{\id}) = \phi_V(x)\otimes\phi_W(x).\]
    So we can take advantage of the fact that $\exp(A+B)=\exp(A)\exp(B)$ for any two commuting operators $A$ and $B$.
    Also using the fact that $\exp(A\otimes{\id})=\exp(A)\otimes{\id}$ for any operator $A$, we have:
    \begin{align*}
        \exp(\phi(x)) &= \exp(\phi_V(x)\otimes{\id}+{\id}\otimes\phi_W(x)) \\
        &= \exp(\phi_V(x)\otimes{\id})\exp({\id}\otimes\phi_W(x)) \\
        &= (\exp(\phi_V(x))\otimes{\id})({\id}\otimes\exp(\phi_W(x))) \\
        &= \exp(\phi_V(x))\otimes\exp(\phi_W(x)).
    \end{align*}
    Now applying the definition of the braid group operators as products of exponentials of $e_i$ and $-f_i$ completes the proof.
\end{proof}

\begin{corollary}\label{C:tau-alg}
    Let $A$ be an associative algebra such that there is an injective algebra homomorphism $\Ug\inj A$, and let $V$ be a module for $A$.
    If the actions of $\g$ on $A$ and $V$ are both integrable, then
    \[\braid_i^V(a\cdot v) = \braid_i^A(a)\cdot\braid_i^V(v)\]
    for all $a\in A$ and $v\in V$.
\end{corollary}

This corollary is especially useful because the Yangian is such an associative algebra $A$.


\section{Modified braid group action}

Here we define modified braid group operators that act on the Cartan part of the Yangian.

\section{Action on the universal \texorpdfstring{$R$}{R}-matrix}

Here we show how the braid group acts on the universal $R$-matrix and its $R^0$ part.
