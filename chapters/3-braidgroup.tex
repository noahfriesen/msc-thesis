\chapter{Braid groups}

\section{Braid group operators}

In this section, we review the action of the braid group of a simple Lie algebra $\g$ on its integrable representations, and prove some important properties.

The \emph{braid group} of $\g$, denoted $\Bg$, is the group with generators $\braid_i$ for $i\in\I$ subject to the following defining relations, called the \emph{braid relations}: for all $i,j\in\I$ with $i\neq j$, we have
\[\underbrace{\braid_i\braid_j\braid_i\cdots}_{m_{ij}\text{ factors}} = \underbrace{\braid_j\braid_i\braid_j\cdots}_{m_{ij}\text{ factors}}\]
where the number of factors $m_{ij}=m_{ji}$ on each side of the above equality is defined according to the entries of the Cartan matrix:
\[\begin{array}{c|cccc}
    a_{ij}a_{ji} & 0 & 1 & 2 & 3 \\
    \hline
    m_{ij} & 2 & 3 & 4 & 6
\end{array}\]

Recall that the Weyl group $\Wg$ associated to $\g$ is generated by the simple reflections $s_i$ for $i\in\I$, which satisfy the same relations as those of the braid group generators as well as the relation $s_i^2={\id}$ for all $i\in\I$.
Thus there is a surjective map $\Bg\surj\Wg$ defined by $\braid_i\mapsto s_i$, with kernel generated by the elements $\braid_i^2$ for all $i\in\I$.
This surjection has a section (i.e., a right inverse) $\Wg\to\Bg$ where $w\mapsto\braid_w$, defined by taking a reduced expression $w=s_{i_1}\cdots s_{i_\ell}$ and setting $\braid_w:=\braid_{i_1}\cdots\braid_{i_\ell}$.
It turns out that this is independent of the choice of reduced expression for $w$, and these elements satisfy the property $\braid_{vw}=\braid_v\braid_w$ whenever the lengths of the reduced expressions for $v$ and $w$ add: $\ell(vw)=\ell(v)+\ell(w)$; see \cite[\S2.1.2]{lusztig_introduction_2010}.

As mentioned above, there is an action of the braid group on integrable representations of $\g$, which are defined below.

\begin{definition}\label{D:integrable}
    A representation $(V,\phi)$ of $\g$ is \emph{integrable} if the following two conditions hold:
    \begin{enumerate}
        \item $V$ is a direct sum of weight spaces $V_\mu$ where $\mu(h_i)\in\Z$ for all $i\in\I$,
        \item $\phi(e_i)$ and $\phi(f_i)$ are \emph{locally nilpotent} endomorphisms of $V$ for all $i\in\I$: for each $v\in V$ there exists some $n\in\N$ such that $\phi(e_i)^n(v)=\phi(f_i)^n(v)=0$.
    \end{enumerate}
\end{definition}

Given such a representation $(V,\phi)$, we can define the following operators for each $i\in\I$:
\[\braid_i^V := \exp(\phi(e_i))\exp(\phi(-f_i))\exp(\phi(e_i)).\]
Since $\phi(e_i)$ and $\phi(-f_i)$ are locally nilpotent endomorphisms of $V$, it follows by the argument in \cite[\S21.2]{humphreys_introduction_1972} that each $\braid_i^V$ is an automorphism of $V$.
One can check that these operators satisfy the braid relations, so the assignment $\braid_i\mapsto\braid_i^V$ gives us an action of the braid group on $V$.
% TODO: prove this, or give an example (maybe adjoint rep of sl3)

Recall that there is an action of the Weyl group on $\h^*$ given by
\[s_i(\alpha_j) = \alpha_j - a_{ij}\alpha_i\]
for all $i,j\in\I$.
The following proposition shows that the braid group action on $V$ intertwines the weight spaces of $V$ via the Weyl group action on the weights:

\begin{proposition}\label{P:tau-wt-space}
    For all $i\in\I$ and $\mu\in\h^*$, we have $\braid_i^V(V_\mu)=V_{s_i(\mu)}$.
\end{proposition}
\begin{proof}
    We use the argument shown in \cite[\S1.3]{kumar_kac-moody_2002}.
    Fix $v\in V_\mu$, then for any $h\in\h$ such that $\alpha_i(h)=0$, we have
    \[h\cdot \braid_i^V(v) = \mu(h)\braid_i^V(v).\]
    By the Serre relations and some properties of nilpotent derivations (namely using the Leibniz rule), we have
    \[(\braid_i^V)^{-1}\phi(h_i)\braid_i^V = -\phi(h_i)\]
    as endomorphisms of $V$.
    From this, it follows that
    \begin{align*}
        h_i\cdot\braid_i^V(v) &= -\braid_i^V(h_i\cdot v) \\
        &= -\braid_i^V(\mu(h_i)v) \\
        &= -\mu(h_i)\braid_i^V(v)
    \end{align*}
    which is sufficient to show that $\braid_i^V(v)$ is a vector of weight $s_i(\mu)$.
    One can similarly show that $(\braid_i^V)^{-1}(v)$ is also a vector of weight $s_i(\mu)$, and since $s_i^2={\id}$, this completes the proof.
\end{proof}

The following proposition shows another important property of these braid group operators.

\begin{proposition}\label{P:tau-tensor}
    If $(V,\phi_V)$ and $(W,\phi_W)$ are both integrable representations of $\g$, then $\braid_i^{V\otimes W} = \braid_i^V\otimes\braid_i^W$ for all $i\in\I$.
\end{proposition}
\begin{proof}
    Let $\phi=\phi_V\otimes\phi_W$, so for all $x\in\g$ we have
    \[\phi(x) = \phi_V(x)\otimes{\id} + {\id}\otimes\phi_W(x).\]
    Notice that the two terms of the right-hand side of the above equation commute, since
    \[(\phi_V(x)\otimes{\id})(\id\otimes\phi_W(x)) = ({\id}\otimes\phi_W(x))(\phi_V(x)\otimes{\id}) = \phi_V(x)\otimes\phi_W(x).\]
    So we can take advantage of the fact that $\exp(A+B)=\exp(A)\exp(B)$ for any two commuting operators $A$ and $B$.
    Also using the fact that $\exp(A\otimes{\id})=\exp(A)\otimes{\id}$ for any operator $A$, we have:
    \begin{align*}
        \exp(\phi(x)) &= \exp(\phi_V(x)\otimes{\id}+{\id}\otimes\phi_W(x)) \\
        &= \exp(\phi_V(x)\otimes{\id})\exp({\id}\otimes\phi_W(x)) \\
        &= (\exp(\phi_V(x))\otimes{\id})({\id}\otimes\exp(\phi_W(x))) \\
        &= \exp(\phi_V(x))\otimes\exp(\phi_W(x)).
    \end{align*}
    Now applying the definition of the braid group operators as products of exponentials of $e_i$ and $-f_i$ completes the proof.
\end{proof}

\begin{corollary}\label{C:tau-alg}
    Let $A$ be an associative algebra such that there is an injective algebra homomorphism $\Ug\inj A$, and let $V$ be a module for $A$.
    If the actions of $\g$ on $A$ and $V$ are both integrable, then
    \[\braid_i^V(a\cdot v) = \braid_i^A(a)\cdot\braid_i^V(v)\]
    for all $a\in A$ and $v\in V$.
\end{corollary}

This corollary is especially useful because the Yangian is such an associative algebra $A$.


\section{Modified braid group operators}

In this section, we will define an action of the braid group on the Cartan part $\Yhg[0]$ of the Yangian by modifying the operators defined in the previous section.

Recall the triangular decomposition of the Yangian:
\[\Yhg \cong \Yhg[-]\otimes\Yhg[0]\otimes\Yhg[+],\]
and the fact that the Yangian is a Hopf algebra with counit $\varepsilon$ that maps all generators to zero.
Now define a map $\varepsilon^+:={\id}\otimes{\id}\otimes\varepsilon$.
Choosing a PBW basis for the Yangian that corresponds to the above triangular decomposition, we see that this map sends any basis monomial containing raising operators (i.e., elements of $\Yhg[+]$) to zero, and acts as the identity on the other basis monomials.
As mentioned at the end of the previous section, it is well-known that the adjoint action of $\g$ on $\Yhg$ defines an integrable representation of $\g$.
For simplicity, we will denote by $\braid_i$ the algebra automorphisms $\braid_i^{\Yhg}$ defined in the previous section, for $i\in\I$.

\begin{definition}\label{D:mbraid}
    The \emph{modified braid group operators} are the maps defined as follows, for $i\in\I$:
    \[\mbraid_i = \varepsilon^+ \circ \braid_i\Big|_{\Yhg[0]}.\]
\end{definition}

\begin{lemma}\label{L:mbraid-hom}
    Each $\mbraid_i$ is an algebra endomorphism of $\Yhg[0]$.
\end{lemma}
\begin{proof}
    Let $\Yhg_0$ be the zero-weight space of $\Yhg$ as a representation of $\g$.
    Elements of $\Yhg$ belong to this weight space, but so do elements of with the same number of raising and lowering operators such as $x^-_{i,r}\xi_{i,r}x^+_{i,r}$, which means that $\Yhg[0]\subset\Yhg_0$.
    By Proposition \ref{P:tau-wt-space}, this means that $\braid_i$ maps $\Yhg[0]$ into $\Yhg_0$.
    Next, $\varepsilon^+$ maps monomials with operators to zero, so it maps $\Yhg_0$ into $\Yhg[0]$ and hence $\mbraid_i:\Yhg[0]\to\Yhg[0]$.

    For all $a\in\Yhg[0]$, let
    \[\braid_i(a) = a_0+a_+\]
    where $a_0\in\Yhg[0]$ and $a_+\in\Yhg_0\setminus\Yhg[0]$.
    Then for $a,b\in\Yhg[0]$, we have
    \begin{equation}\label{eqn:TaTb}
        \braid_i(a)\braid_i(b) = a_0b_0 + a_0b_+ + a_+b_0 + a_+b_+.
    \end{equation}
    Immediately, we see that applying $\varepsilon^+$ to the above will map the second and fourth terms of the right-hand side to zero, since they have raising operators on the right.
    Using the defining relations of the Yangian will allow us to write the third term of the right-hand side above as an element of $\Yhg[0]\otimes\Yhg[+]$ as well: recall that
    \[[\xi_{i,r+1},x^+_{j,s}]-[\xi_{i,r},x^+_{j,s+1}]-\hbar\frac{d_ia_{ij}}{2}(\xi_{i,r}x^+_{j,s}+x^+_{j,s}\xi_{i,r})\]
    for all $i,j\in\I$ and $r,s\in\N$.
    Expanding, we have
    \[x^+_{j,s}\xi_{i,r+1}=\xi_{i,r+1}x^+_{j,s} -\xi_{i,r}x^+_{j,s+1} -\hbar\frac{d_ia_{ij}}{2}\xi_{i,r}x^+_{j,s} +x^+_{j,s+1}\xi_{i,r} -\hbar\frac{d_ia_{ij}}{2}x^+_{j,s}\xi_{i,r}.\]
    The first three terms of the right-hand side above are in $\Yhg[0]\otimes\Yhg[+]$, and the other two terms are still in $\Yhg[+]\otimes\Yhg[0]$ but have a lower subscript $r$, which means that we can repeatedly apply this relation to these terms and this process will terminate.
    Thus all but the first term of the right-hand side of equation (\ref{eqn:TaTb}) will be mapped to zero by $\varepsilon^+$, so we have
    \[\mbraid_i(ab) = \varepsilon^+(\braid_i(ab)) = \varepsilon^+(\braid_i(a)\braid_i(b)) = a_0b_0 = \mbraid_i(a)\mbraid_i(b)\]
    which completes the proof.
\end{proof}

Further, one can show using a slightly different construction and properties of graded algebras that the operators $\mbraid_i$ are invertible and are coalgebra homomorphisms that commute with the antipode, and are therefore Hopf algebra automorphisms of $\Yhg[0]$.

Now recall that every $w\in\Wg$ has a lift $\braid_w\in\Bg$, and therefore a corresponding automorphism of $\Yhg$ which we will also denote by $\braid_w$.
The following lemma was proved in \cite[Thm. 5.3.19]{weekes_highest_2016}, and gives an analogue of this lifting for our modified operators.

\begin{lemma}\label{L:Tw}
    Let $w=s_{i_1}\cdots s_{i_\ell}$ be a reduced expression. Then
    \[\mbraid_w := \varepsilon^+\circ\braid_w\Big|_{\Yhg[0]} = \mbraid_{i_1}\cdots\mbraid_{i_\ell}.\]
\end{lemma}

As a consequence of this lemma, it follows that the operators $\mbraid_i$ satisfy the braid relations, and therefore define an action of the braid group on $\Yhg[0]$.
The results of this section are summarized in the following theorem.

\begin{theorem}\label{T:mbraid-action}
    The assignment $\braid_i\mapsto\mbraid_i$ for all $i\in\I$ defines an action of the braid group $\Bg$ on $\Yhg[0]$ by Hopf algebra automorphisms.
\end{theorem}


\section{Action on generating series of \texorpdfstring{$\Yhg[0]$}{Y0}}

In this section, we will compute the action of the modified braid group operators $\mbraid_i$ on $\Yhg[0]$ explicitly, using the generating series $(A_j(u))_{j\in\I}$ from \cite{gerasimov_class_2005}.
For the definition of these series and the other series used below, see Section \ref{ssec:alt-gen}.
The following proposition gives us explicit formulas for both the regular and modified braid group actions.
As a remark, the first equation of the following proposition was established for simply laced (i.e., type A, D, and E) Lie algebras in \cite[Lem. 5.3.16]{weekes_highest_2016}.

\begin{proposition}\label{P:tau-a}
    For all $i,j\in\I$, we have
    \[\braid_i(A_j(u)) =
    \begin{cases}
        D_i(u) & \text{if}\quad i=j, \\
        A_j(u) & \text{if}\quad i\neq j.
    \end{cases}\]
    Consequently, we have
    \[\mbraid_i(A_j(u)) = A_j(u)\xi_i(u)^{\delta_{ij}}.\]
\end{proposition}
\begin{proof}
    By the relations in the first part of Corollary \ref{C:GKLO}, if $i\neq j$ then $\mbraid_i(A_j(u)) = A_j(u)$.
    Using the relations in the second and third part of the same corollary, along with the fact that exponentials of linear maps are themselves linear maps, we obtain
    \begin{align*}
        \braid_i(A_i(u)) &= \exp(\ad(e_i))\exp(-\ad(f_i))\exp(\ad(e_i))(A_i(u)) \\
        &= \exp(\ad(e_i))\exp(-\ad(f_i))(A_i(u) + B_i(u)) \\
        &= \exp(\ad(e_i))(A_i(u) + B_i(u) - (A_i(u) - C_i(u) - D_i(u)) - C_i(u)) \\
        &= \exp(\ad(e_i))(B_i(u) + D_i(u)) \\
        &= B_i(u) + D_i(u) - B_i(u) \\
        &= D_i(u),
    \end{align*}
    which proves the first equality of the proposition.
    Now using the definition of $D_i(u)$, we have
    \[\varepsilon^+(D_i(u)) = \varepsilon^+(A_i(u)\xi_i(u) + C_i(u)A_i(u)^{-1}B_i(u)) = A_i(u)\xi_i(u)\]
    and so the second equality of the proposition follows from the definition of $\mbraid_i$.
\end{proof}

Using the defining relation of the generating series $A_j(u)$, along with the fact that the series $A_j(u)$ and $\xi_j(u)$ commute for all $j\in\I$, we obtain the following corollary, giving us an explicit formula for the modified braid group action on the more familiar generating series $\xi_j(u)$.

\begin{corollary}\label{C:tau-xi}
    For all $i,j\in\I$, we have
    \[\mbraid_i(\xi_j(u)) = \xi_j(u)\prod_{k=0}^{\abs{a_{ij}}-1} \xi_i\left(u-\frac{\hbar d_i}{2}(\abs{a_{ij}-2k})\right)^{(-1)^{\delta_{ij}}}.\]
\end{corollary}

Finally, we give an example to illustrate these explicit formulas more concretely.

\begin{example}\label{E:sl3-mbraid-action}
    Consider the Lie algebra $\mfsl_3$, which has $2\times 2$ Cartan matrix given by $a_{11}=a_{22}=2$ and $a_{12}=a_{21}=-1$.
    Notice that the Cartan matrix is symmetric, so the symmetrizing integers are $d_1=d_2=1$.
    By definition, for $i\neq j$, we have
    \[\xi_i(u) = \frac{A_i(u-\frac{\hbar}{2})}{A_j(u)A_j(u-\frac{\hbar}{2})}.\]
    Using Corollary \ref{C:tau-xi} above, we find that the modified braid group action is given by
    \[\mbraid_i(\xi_i(u)) = \frac{1}{\xi_i(u-\hbar)}, \qquad
    \mbraid_i(\xi_j(u)) = \xi_j(u)\xi_i(u-\tfrac{\hbar}{2}),\]
    for all $i\neq j$.
    Using these formulas, it is easy to check that these braid group operators satisfy the braid relation for $\mfsl_3$:
    \[\mbraid_1\mbraid_2\mbraid_1 = \mbraid_2\mbraid_1\mbraid_2.\]
\end{example}


\section{Action on the universal \texorpdfstring{$R$}{R}-matrix}

In this section, we show how the braid group operators and their modified versions interact with the universal $R$-matrix of the Yangian.
