\chapter{Dual braid group action and weights}

\section{Dualizing the braid group action}

In this section, we dualize the modified action of the braid group $\Bg$ on the Cartan part $\Yhg[0]$ of the Yangian that we constructed in the previous chapter.
In doing so, we obtain an action of $\Bg$ by automorphisms on the group $(1+u^{-1}\C\db{u^{-1}})^\I$ of $\I$-tuples of formal series in $u^{-1}$ with constant coefficient $1$.

Let $\vartheta$ be the group antiautomorphism of $\Bg$ uniquely determined by $\vartheta(\braid_i)=\braid_i$ for all $i\in\I$.
Notice that if $w = s_{i_1}\cdots s_{i_\ell}$ is a reduced expression for $w\in\Wg$, then $w^{-1} = s_{i_\ell}\cdots s_{i_1}$, so it follows that $\vartheta(\braid_w)=\braid_{w^{-1}}$ for any $w\in\Wg$.
We can use this antiautomorphism to define an action of $\Bg$ on the linear dual $\Yhg[0]^*$ as follows: for $\sigma\in\Bg$, $y\in\Yhg[0]$, and $f\in\Yhg[0]^*$, let
\[\sigma(f)(y) = f(\vartheta(\sigma)\cdot y).\]
This action is uniquely determined by the requirement that $\braid_i$ operates as the transpose of $\mbraid_i$, i.e., for all $i\in\I$ and $f\in\Yhg[0]^*$, we have
\[\braid_i(f) = \mbraid_i^*(f) = f\circ\mbraid_i.\]

Recall the Drinfeld Hopf algebra structure on $\Yhg[0]$ from Section \ref{ssec:Y-Hopf}.
This induces a commutative algebra structure on $\Yhg[0]^*$ where the unit is given by the counit $\varepsilon_D$ and the product is given by $\Delta_D^*$, the transpose of the coproduct.
Since each modified braid group operator $\mbraid_i$ is a coalgebra homomorphism, $\Bg$ acts on $\Yhg[0]^*$ by algebra automorphisms.
Since each $\mbraid_i$ is an algebra automorphism, the space of algebra homomorphisms $\Hom_{\text{Alg}}(\Yhg[0],\C)$ is a subrepresentation of $\Yhg[0]^*$.
This space is a subgroup of the group of units in $\Yhg[0]^*$, and we have an isomorphism of groups
\[\Hom_{\text{Alg}}(\Yhg[0],\C)\to(1+u^{-1}\C\db{u^{-1}})^\I, \qquad f\mapsto(f(\xi_i(u)))_{i\in\I}.\]
The following proposition gives a formula for the action of $\Bg$ on this group.

\begin{proposition}\label{P:Tan-formula}
    If $\ul = (\lambda_i(u))_{i\in\I}\in (1+u^{-1}\C\db{u^{-1}})^\I$, then the for all $i,j\in\I$, the $i$th component of $\braid_j(\ul)$ is given by
    \[\braid_j(\ul)_i = \lambda_i(u)\prod_{k=0}^{\abs{a_{ji}}-1}\lambda_j\left(u-\frac{\hbar d_j}{2}(\abs{a_{ji}}-2k)\right)^{(-1)^{\delta_{ij}}}.\]
\end{proposition}
\begin{proof}
    By the discussion above, we have
    \[\braid_j(\ul)_i = \braid_j(\ul)(\xi_i(u)) = \ul(\mbraid_j(\xi_i(u)))\]
    where we view $\ul$ as an algebra homomorphism in $\Yhg[0]^*$ via the above isomorphism.
    Then using the formula for $\mbraid_j(\xi_i(u))$ given in Corollary \ref{C:tau-xi}, we see that the claimed formula holds.
\end{proof}

The formula for the action of $\Bg$ on $(1+u^{-1}\C\db{u^{-1}})^\I$ given in the above proposition is the same as the formula for the action of $\Bg$ on the group $(\C(u)^\times)^\I$ of $\I$-tuples of rational functions defined in \cite[Prop. 3.1]{tan_braid_2015}.
This is why we have used the antiautomorphism $\vartheta$ to define the dual action rather than the usual antiautomorphism $\sigma\mapsto\sigma^{-1}$ of $\Bg$.


\section{Extending the representation space}

In this section, we will extend our dual braid group action to a larger space in order to reconcile our action and Tan's (see the end of the previous section).

Let $M$ be the subgroup of $\C\dparen{u^{-1}}^\times$ consisting of monic Laurent series in $u^{-1}$:
\[M := \bigcup_{k\in\Z}u^{k}(1+u^{-1}\C\db{u^{-1}}).\]
Define the \emph{degree} of $\lambda(u)\in M$ to be the unique integer $k$ for which $u^{-k}\lambda(u)\in 1+u^{-1}\C\db{u^{-1}}$.

\begin{lemma}\label{L:diff-eqn}
    Let $\lambda(u) = 1+\hbar\sum_{r\geq 0}\lambda_ru^{-r-1}$ be a series in $1+u^{-1}\C\db{u^{-1}}$ and let $d\in\C^\times$.
    Then the formal difference equation
    \[\frac{\mu(u+\hbar d)}{\mu(u)} = \lambda(u)\]
    has a solution in $M$ if and only if $\lambda_0\in d\Z$.
    In this case, $\mu(u)$ is unique and has degree $\lambda_0/d$.
\end{lemma}
\begin{proof}
    Suppose $\mu(u)$ has degree $k$, and write $\mu(u)=u^k\dot{\mu}(u)$ where $\dot{\mu}\in 1+u^{-1}\C\db{u^{-1}}$.
    Then the difference equation becomes
    \[\left(1+\hbar\frac{d}{u}\right)^k \frac{\dot{\mu}(u+\hbar d)}{\dot{\mu}(u)} = \lambda(u).\]
    Taking the formal logarithm of both sides, we obtain
    \[k\log(1+\hbar\frac{d}{u}) + b(u+\hbar d) - b(u) = \log(\lambda(u)),\]
    where
    \[b(u) = \sum_{r\geq 0}b_ru^{-r-1} := \log(\dot{\mu}(u)).\]
    Taking the coefficient of $u^{-1}$ on both sides, we see that $k\hbar d = \hbar\lambda_0$.
    Thus if a solution exists, we must have $\lambda_0\in d\Z$ and $k = \lambda_0/d$.

    Conversely, suppose that $\lambda_0\in d\Z$.
    Take $k = \lambda/d$, then we can solve the above equation for $b(u)$ recursively by expressing $b_{r}$ in terms of $b_s$ for $s<r$ and the coefficients of $k\log(1+\hbar\frac{d}{u})$ and $\log(\lambda(u))$.
    This allows us to construct the solution $\mu(u)$.
\end{proof}

Let $(\omega_i)_{i\in\I}$ denote the \emph{fundamental weights} of $\g$, which form a basis of $\h^*$ dual to the basis $(h_i)_{i\in\I}$ of $\h$, i.e., $\omega_i(h_j) = \delta_{ij}$ for all $i,j\in\I$.
Let $\Lambda := \bigoplus_{i\in\I}\Z\omega_i$ denote the \emph{weight lattice} of $\g$.
Recall from Section \ref{sec:braid} that there is an action of $\Wg$ on $\h^*$, so we may view $\Lambda$ as a representation of $\Bg$ through this action.
We define the \emph{degree} of an element of $M^\I$ via the following group homomorphism:
\[\deg:M^\I\to\Lambda, \qquad (\lambda_i(u))_{i\in\I}\mapsto\sum_{i\in\I}\deg(\lambda_i(u))\omega_i.\]
Next, for each $a\in\C$ let $\q^a$ be the group automorphism of $M$ given by the translation $\lambda(u)\mapsto\lambda(u+a\frac{\hbar}{2})$.
Similarly, given a diagonal matrix $A = (a_i)_{i\in\I}$, let $\q^A$ be the group automorphism of $M^\I$ given by
\[\q^A(\lambda(u))_{i\in\I} = (\q^{a_i}\lambda_i(u))_{i\in\I} = (\lambda_i(u+a_i\tfrac{\hbar}{2}))_{i\in\I}.\]
We will only be interested in the case where $A = 2D$, where $D = (d_i)_{i\in\I}$ is the diagonal matrix of symmetrizing integers for $\g$.
Hence the map $\q^{2D}$ is given by $\lambda_i(u)\mapsto\lambda_i(u+\hbar d_i)$ for each $i\in\I$.
Also note that for all $\um\in M^\I$, the element $(\q^{2D}\um)\um^{-1}$ belongs to $(1+u^{-1}\C\db{u^{-1}})^\I$, so we may apply our dual braid group action to it.

\begin{proposition}\label{P:Bg-action-M}
    For all $\sigma\in\Bg$ and $\um = (\mu_i(u))_{i\in\I} \in M^\I$, there exists a unique element $\sigma(\mu)\in M^\I$ satisfying
    \[\sigma\left(\frac{\q^{2D}\um}{\um}\right) = \frac{\q^{2D}\sigma(\um)}{\sigma(\um)}.\]
    This defines an action of $\Bg$ on $M^\I$ by group automorphisms for which $1+u^{-1}\C\db{u^{-1}}^\I$ is a subrepresentation and $\deg:M^\I\to\Lambda$ is a module homomorphism:
    \[\deg\sigma(\um) = \sigma(\deg\um).\]
\end{proposition}
\begin{proof}
    Let $\ul = (\lambda_i(u))_{i\in\I}$ denote the element $(\q^{2D}\um)\um^{-1}$, and write $\lambda_i(u) = 1+\hbar\sum_{r\geq 0}\lambda_{i,r}u^{-r-1}$.
    There is an isomorphism
    \[\h^*\to\C^\I, \qquad f\mapsto (f(h_i))_{i\in\I} = (f, \alpha_i)_{i\in\I},\]
    so we have an action of $\Bg$ on $\C^\I$.
    Then
    \[\sigma(\ul) = 1 + \hbar\sigma\cdot(\lambda_{i,0})u^{-1} + O(u^{-2}),\]
    and using the action of $\Bg$ on $\C^\I$, we have
    \[\sigma\cdot(\lambda_{i,0})_{i\in\I} = \left(\sum_{j\in\I}d_j^{-1}\lambda_{j,0}(\alpha_i,\sigma(\omega_j))\right)_{i\in\I}\]
    By the previous lemma, we have $\lambda_{j,0} = d_j\deg\mu_j(u)$ for all $j\in\I$, so the above is equal to $(\alpha_i,\sigma(\deg\um))_{i\in\I}$.
    Since the action of $\Bg$ on $\h^*$ preserves $\Lambda$, we have $(\alpha_i,\sigma(\deg\um))\in d_i\Z$ for all $i\in\I$.
    Thus by the previous lemma, $\sigma(\um)$ exists, is unique, and has degree $\sigma(\deg\um)$.
\end{proof}

By considering the unique simply connected Lie group with Lie algebra $\g$, one can show that there is an action of $\Wg$ by group automorphisms on $(\C^\times)^\I$ given explicitly for all $i,j\in\I$ and $\lambda_i\in\C^\times$ by
\[s_j(\lambda_i)_{i\in\I} = (\lambda_i\lambda_j^{-a_{ji}})_{i\in\I},\]
hence we get an action of $\Bg$ on this group as well.
Then using the action given in the above proposition, we have an action of $\Bg$ on the group $(\C^\times)^\I\times M^\I$.
There is an isomorphism from this group to $(\C\dparen{u^{-1}}^\times)^\I$ given by component-wise multiplication:
\[((\lambda_i)_{i\in\I}, (\mu_i(u))_{i\in\I}) \mapsto (\lambda_i\mu_i(u))_{i\in\I},\]
hence $\Bg$ acts on $(\C\dparen{u^{-1}}^\times)^\I$ by group automorphisms via the formulas of Proposition \ref{P:Tan-formula}.
The space $(\C(u)^\times)^\I$ is a subrepresentation, so we have successfully extended our dual braid group action so that the underlying space agrees with that of \cite[Prop. 3.1]{tan_braid_2015}.


\section{Weights of extremal vectors}

In this section, we will show that the braid group action of the previous section determines the weights of the \emph{extremal vectors} of a representation of the Yangian.

Let $V$ be a finite-dimensional irreducible representation of $\Yhg$ with highest weight $\ul$, and let $\lambda\in\h^*$ be the highest weight of $V$ as a representation of $\g$.
The \emph{extremal vectors} are those which lie in the Weyl group orbit of the highest weight vector, i.e., they are elements of $V_{w(\lambda)}$ for $w\in\Wg$.
Recall that the highest weight space is one-dimensional, so by Proposition \ref{P:tau-wt-space}, each extremal weight space is also one-dimensional.
The following proposition is the main result of this section, and provides a strengthening of \cite[Prop. 4.5]{tan_braid_2015}.

\begin{proposition}\label{P:extremal-weight}
    Let $V$ be as above.
    For all $i\in\I$ and $w\in\Wg$, the eigenvalue of $\xi_i(u)$ on the extremal weight space $V_{w(\lambda)}$ is $\braid_w(\ul)_i$.
    In particular, if $V=L(\uP)$ then this eigenvalue is equal to
    \[\frac{\q^{2d_i}\braid_w(\uP)_i}{\braid_w(\uP)_i}.\]
\end{proposition}
\begin{proof}
    Let $v$ be a highest weight vector of $V$.
    Using the definition of our dual braid group action from the first section of this chapter, we have
    \[\braid_w(\ul)_i = \ul(\mbraid_{w^{-1}}(\xi_i(u))),\]
    so for the first part of the proposition, it suffices to prove that the eigenvalue of $\xi_i(u)$ on the extremal vector $\Omega_w := \braid_{w^{-1}}(v)$ is the same as that of $\mbraid_w(\xi_i(u))$ on the highest weight vector $v$.
    Let $\mu_{i,w}(u)$ denote this former eigenvalue, so that
    \[\xi_i(u)\cdot\Omega_w = \mu_{i,w}(u)\Omega_w.\]
    Applying the (unmodified) braid group operator $\braid^V_w$ to both sides of the above equation, for the right-hand side we obtain $\mu_{i,w}(u)v$.
    For the left-hand side, we have
    \[\braid^V_w(\xi_i(u)\cdot\Omega_w) = \braid^{\Yhg}_w(\xi_i(u))\cdot\braid^V_w(\Omega_w) = \braid^{\Yhg}_w(\xi_i(u))\cdot v\]
    where in the first equality we have used Corollary \ref{C:tau-alg}.
    Now $\braid^{\Yhg}_w$ sends $\xi_i(u)$ to something in the zero weight space $\Yhg_0$ but not necessarily something in $\Yhg[0]$.
    But any terms not in $\Yhg[0]$ have raising operators on the right and thus kill the highest weight vector $v$, so in fact we have
    \[\braid^{\Yhg}_w(\xi_i(u))\cdot v = \mbraid_w(\xi_i(u))\cdot v\]
    as desired, which completes the proof of the first part of the proposition.

    The second part follows from the first part, and that since $\ul = (\q^{2D}\uP)\uP^{-1}$, by Proposition \ref{P:Bg-action-M} we have $\braid_w(\ul) = (\q^{2D}\braid_w(\uP))\braid_w(\uP)^{-1}$.
\end{proof}

Lastly, we prove a corollary of the above proposition which further relates our work to that of Tan; cf. \cite[Prop. 4.5]{tan_braid_2015}.

\begin{corollary}\label{C:Tan-monic}
    Let $\uP = (P_i(u))_{i\in\I}$ be a tuple of Drinfeld polynomials and let $V = L(\uP)$.
    For all $i\in\I$ and $w\in\Wg$, define $P_{i,w}(u)\in M$ by
    \[P_{i,w}(u) :=
    \begin{cases}
        \braid_w(\uP)_i & \text{if } w^{-1}(\alpha_i)\in\Phi^+ \\
        \braid_w(\uP)_i^{-1} & \text{if } w^{-1}(\alpha_i)\notin\Phi^+
    \end{cases}\]
    Then $P_{i,w}(u)$ is a monic polynomial in $u$.
\end{corollary}
\begin{proof}
    By \cite[Remark 2.2]{chari_fundamental_1991}, there is a monic polynomial $P^+_{i,w}(u)$ such that the eigenvalue of $\xi_i(u)$ on the extremal weight space $V_{w(\lambda)}$ is given by
    \[\frac{P^+_{i,w}(u+\hbar d_i)^{f_i(w)}}{P^+_{i,w}(u)^{f_i(w)}},\]
    where $f_i(w)$ is $1$ if $w\in\Phi^+$ and $-1$ otherwise.
    By the above proposition and the uniqueness assertion of Lemma \ref{L:diff-eqn}, we have
    \[P^+_{i,w}(u)^{f_i(w)} = \braid_w(\uP)_i = P_{i,w}(u)^{f_i(w)}\]
    hence $P_{i,w}(u)$ coincides with the monic polynomial $P^+_{i,w}(u)$.
\end{proof}