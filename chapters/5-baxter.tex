\chapter{Baxter polynomials and cyclicity}\label{chap:baxter}

A representation of $\Yhg$ is said to be \emph{cyclic} if it is generated by a single vector, i.e., if it is a highest-weight representation.
In this chapter, we prove a conjecture from \cite[\S 7.4]{gautam_poles_2023} which states that the two sufficient conditions for the cyclicity and irreducibility of any tensor product $L(\uP)\otimes L(\uQ)$ obtained in \cite{gautam_poles_2023} and \cite{tan_braid_2015} are identical.


\section{Baxter polynomials and poles}\label{sec:baxter-poles}

Let $V$ be a finite-dimensional highest-weight representation of $\Yhg$, and let $\lambda\in\h^*$ be the highest weight of $V$ as a representation of $\g$.
Recall the series $A_i(u)$ of Section \ref{ssec:alt-gen}; for each $i\in\I$ let $\lambda_i^A(u)\in 1+u^{-1}\C\db{u^{-1}}$ denote the eigenvalue of $A_i(u)$ on the highest weight space $V_\lambda$.
We then introduce the normalized operator
\[A_i^V(u) := \lambda_i^A(u)^{-1}A_i(u)\big|_V \in\End(V)\db{u^{-1}},\]
which acts by the identity on the highest weight space.
Then by \cite[Thm. 4.4]{gautam_poles_2023} (see also \cite[Cor. 4.7]{gautam_poles_2023} together with \cite[Prop. 5.7, 5.8]{hernandez_shifted_2022}), there is a unique monic polynomial $T_i(u)\in\End(V)[u]$ that satisfies
\begin{equation}\label{eqn:transfer-op}
    T_i(u+\hbar d_i) = A_i^V(u)T_i(u).
\end{equation}
We note that $T_i(u)$ can be recovered as $\lambda_i^T(u)^{-1}\mathscr{T}_i(u)$, where $\mathscr{T}_i(u)$ is the \emph{$i$th abelianized transfer operator} introduced in \cite[\S 4.3]{gautam_poles_2023} and $\lambda_i^T(u)$ is the eigenvalue of $\mathscr{T}_i(u)$ on the highest weight space; see \cite[Remark 5.1]{friesen_braid_2024} for details.
The eigenvalues of the operator $T_i(u)$ are called the \emph{(specialized) Baxter polynomials} associated to $V$, and were first introduced in generality in \cite{frenkel_baxters_2015}.

The generating series $\xi_i(u)$ and $x^\pm_i(u)$ of $\Yhg$ introduced in Section \ref{ssec:Y-Hopf} operate on $V$ as the expansions at infinity of $\End(V)$-valued rational functions of $u$; see \cite[Prop. 3.6]{gautam_yangians_2016}.
We define the \emph{$i$th set of poles} of $V$ to be the joint set of poles of these operators:
\[\sigma_i(V) := \{\text{Poles of } \xi_i(u)\big|_V,\ x^\pm_i(u)\big|_V \in\End(V)(u)\} \subset\C.\]
The Baxter polynomials are related to the poles of $V$ in the following way: let $\zeros(P(u))$ denote the zeros of any polynomial $P(u)$, let $\mathcal{Z}_i(V)$ denote the zeros of all eigenvalues of $T_i(u)$, and let $\baxterV(u)$ denote the eigenvalue of $T_i(u)$ on the lowest weight space $V_{w_0(\lambda)}$ where $w_0\in\Wg$ is the longest element.
Then by \cite[Thm. 4.4]{gautam_poles_2023}, for all $i\in\I$ we have
\[\sigma_i(V) = \mathcal{Z}_i(V) = \zeros(\baxterV(u)).\]

In the case where $V$ is irreducible, the polynomials $\baxterV(u)$ (and hence the poles of $V$) were computed explicitly in \cite[Thm. 5.2]{gautam_poles_2023}: for all $i\in\I$,
\[\baxterV(u) = \prod_{j\in\I}\prod_{b=d_i}^{2\kappa-d_i} P_j\left(u-(b-d_j)\frac{\hbar}{2}\right)^{v_{ij}^{(b)}}\]
where $\uP=(P_j(u))_{j\in\I}$ is the tuple of Drinfeld polynomials associated to $V\cong L(\uP)$, $\kappa$ is $1/4$ times the eigenvalue of the Casimir element $C\in\Ug$ on the adjoint representation of $\g$ as in Example \ref{E:dual-braid-sl3}, and the integers $v_{ij}^{(b)}$ are obtained from the \emph{quantum Cartan matrix} $(v_{ij}(z))_{i,j\in\I}$; see \cite[\S 5.2]{friesen_braid_2024} for details.


\section{Baxter polynomials associated to extremal weights}\label{sec:baxter-extremal}

Now let $V$ be irreducible, so $V=L(\uP)$ for some Drinfeld polynomials $\uP=(P_i(u))_{i\in\I}$, and the highest weight of $V$ as a representation of $\g$ is $\lambda = \sum_{i\in\I}\deg(P_i)\fund_i$.
For any $w\in\Wg$, let $\baxterP(u)\in\C[u]$ denote the eigenvalue of $T_i(u)$ on the extremal weight space $V_{w(\lambda)}$.
In particular, if $w$ is the longest element $w_0$, then $\baxterP(u) = \baxterV(u)$.

The following theorem is the main result of this chapter.
It gives a factorization of the Baxter polynomial $\baxterP(u)$ in terms of the braid group action defined in the previous chapter on the Drinfeld polynomials.

\begin{theorem}\label{T:baxter}
    Let $w = s_{j_1}s_{j_2}\cdots s_{j_p}$ be a reduced expression for $w\in\Wg$.
    For each $1\leq r\leq p$, set $w_r := s_{j_{r+1}}\cdots s_{j_p}$, where $w_p = {\id}$.
    Then
    \[\baxterP(u) = \prod_{r:j_r=i} \braid_{w_r}(\uP)_i.\]
    Moreover, $\braid_{w_r}(\uP)_{j_r}$ is a monic polynomial in $u$ for each $r$.
\end{theorem}
\begin{proof}
    First, since $s_{j_p}\cdots s_{j_{r+1}}s_{j_r}$ is a reduced expression, it follows that $w_r^{-1}(\alpha_{j_r}) = s_{j_p}\cdots s_{j_{r+1}}(\alpha_{j_r}) \in\Phi^+$.
    Then by Corollary \ref{C:Tan-monic}, $\braid_{w_r}(\uP)_{j_r}$ is a monic polynomial in $u$ for each $r$.

    By the defining equation \ref{eqn:transfer-op} of $T_i(u)$, the eigenvalue of the normalized operator $A_i^V(u) = \lambda_i^A(u)^{-1}A_i(u)$ on the extremal weight space $V_{w(\lambda)}$ is given by
    \[\frac{\baxterP(u+\hbar d_i)}{\baxterP(u)}\]
    hence the eigenvalue of $A_i(u)$ on $V_{w(\lambda)}$ is $\lambda_i^A(u)$ times the above.
    On the other hand, by Proposition \ref{P:extremal-weight} the eigenvalue of $A_i(u)$ on $V_{w(\lambda)}$ is $\braid_w(\ul)(A_i(u))$, where $\ul$ is the highest weight of $V$.
    Then by the uniqueness assertion of Lemma \ref{L:diff-eqn}, it suffices to prove that
    \[\braid_w(\ul)(A_i(u))
    = \lambda_i^A(u)\prod_{r:j_r=i}\braid_{w_r}(\ul)_i
    = \lambda_i^A(u)\prod_{r:j_r=i}\frac{\q^{2d_i}\braid_{w_r}(\uP)_i}{\braid_{w_r}(\uP)_i},\]
    where the second equality follows from the fact that the action of $\braid_{w_r}$ is an automorphism and $\ul = (\q^{2D}\uP)\uP^{-1}$.
    Using the definition of the dual braid group action from the beginning of the previous chapter, the first equality is equivalent to
    \[\mbraid_{w^{-1}}(A_i(u)) = A_i(u)\prod_{r:j_r=i}\mbraid_{w_r^{-1}}(\xi_i(u)).\]
    To prove this, we will use induction on the length $p$ of $w$: if $p = 1$ then this equation reduces to the identity $\mbraid_j(A_i(u)) = A_i(u)\xi_i(u)^{\delta_{ij}}$ that we established in Proposition \ref{P:tau-a}.
    Now suppose that this equation holds for $w$ of length $p$, and consider the element $w' = ws_{j_{p+1}}$ of length $p+1$.
    Then again using the identity of Proposition \ref{P:tau-a}, we have
    \begin{align*}
        \mbraid_{(w')^{-1}}(A_i(u)) &= \mbraid_{j_{p+1}}(\mbraid_w(A_i(u))) \\
        &= A_i(u)\xi_i(u)^{\delta_{i,j_{p+1}}}\prod_{\substack{1\leq r\leq p \\ j_r=i}}\mbraid_{j_{p+1}}(\mbraid_{w_r^{-1}}(\xi_i(u))) \\
        &= A_i(u)\xi_i(u)^{\delta_{i,j_{p+1}}}\prod_{\substack{1\leq r\leq p \\ j_r=i}}\mbraid_{(w'_r)^{-1}}(\xi_i(u)) \\
        &= A_i(u)\prod_{\substack{1\leq r\leq p+1 \\ j_r=i}}\mbraid_{(w'_r)^{-1}}(\xi_i(u))
    \end{align*}
    which completes the proof.
\end{proof}


\section{Cyclicity criteria for tensor products}

An important property of the poles of representations of $\Yhg$ is that they encode information about when the tensor product of two irreducible representations is cyclic or irreducible.
By \cite[Thm. 7.2]{gautam_poles_2023}, the representation $L(\uP)\otimes L(\uQ)$ is cyclic if for all $i\in\I$, none of the zeros of $Q_i(u+\hbar d_i)$ are $i$th poles of $L(\uP)$, i.e., if
\[\zeros(Q_i(u+\hbar d_i)) \subset \C\setminus\sigma_i(L(\uP)).\]
By \cite[Cor. 7.3]{gautam_poles_2023}, $L(\uP)\otimes L(\uQ)$ is irreducible if in addition to the above condition, none of the zeros of $P_i(u+\hbar d_i)$ are $i$th poles of $L(\uQ)$:
\[\zeros(P_i(u+\hbar d_i)) \subset \C\setminus\sigma_i(L(\uQ)).\]

\begin{example}\label{E:cyclicity}
    Consider the fundamental representations $V = L_{\fund_1}(a)$ and $W = L_{\fund_1}$ of $Y_\hbar(\mfsl_2)$, which we defined in Example \ref{E:dual-braid-sl3}.
    These representations have Drinfeld polynomials $P_1(u) = u-a$ and $Q_1(u) = u$, respectively.
    The longest element of the Weyl group is the simple reflection $s_1$, so using the formula of Theorem \ref{T:baxter}, we see that the Baxter polynomials associated to the lowest weights of these representations are just the Drinfeld polynomials: $\baxter_{1,V}^{\mfsl_2}(u) = u-a$ and $\baxter_{1,W}^{\mfsl_2} = u$.
    Looking at the zeros of these polynomials, the condition above for $V\otimes W$ to be cyclic is equivalent to $a\neq -\hbar$, and the additional condition for $V\otimes W$ to be irreducible is equivalent to $a\neq\hbar$.
\end{example}

Another sufficient condition for the cyclicity of a tensor product of irreducible representations was given in \cite[Thm. 4.8]{tan_braid_2015} using the action of the braid group on $(\C(u)^\times)^\I$, following \cite{chari_braid_2002}.
Let $w_0 = s_{j_1}s_{j_2}\cdots s_{j_p}$ be a reduced expression for the longest element $w_0\in\Wg$, and set $w_r := s_{j_{r+1}}\cdots s_{j_p}$ for each $1\leq r\leq p$ as in the previous section.
Then by \cite[Thm. 4.8]{tan_braid_2015} and the results of Section \ref{sec:extending}, $L(\uP)\otimes L(\uQ)$ is cyclic if for all $1\leq r\leq p$, the polynomials $Q_{j_r}(u+\hbar d_{j_r})$ and $\braid_{w_r}(\uP)_{j_r}$ have no roots in common:
\[\zeros(Q_{j_r}(u+\hbar d_{j_r})) \subset \C\setminus\zeros(\braid_{w_r}(\uP)_{j_r}).\]

It was conjectured in \cite[\S 7.5]{gautam_poles_2023} that the two sufficient conditions for cyclicity above are identical.
To prove this, recall from Section \ref{sec:baxter-poles} that the poles of a finite-dimensional highest-weight representation are exactly the zeros of the Baxter polynomial associated to the lowest weight.
Then using the factorization given in Theorem \ref{T:baxter}, we obtain the following corollary.

\begin{corollary}\label{C:baxter-poles}
    Let $\uP$ be a tuple of Drinfeld polynomials.
    For all $i\in\I$, the $i$th set of poles of $L(\uP)$ is given by
    \[\sigma_i(L(\uP)) = \bigcup_{r:j_r=i}\zeros(\braid_{w_r}(\uP)_i).\]
\end{corollary}

Combining this corollary with the cyclicity criteria of \cite{gautam_poles_2023} and \cite{tan_braid_2015} above, we see that they are identical.
The following corollary summarizes this result.

\begin{corollary}\label{C:cyclicity}
    Let $\uP = (P_i(u))_{i\in\I}$ and $\uQ = (Q_i(u))_{i\in\I}$ be tuples of Drinfeld polynomials.
    The following conditions are equivalent:
    \begin{enumerate}
        \item For all $i\in\I$, $\zeros(Q_i(u+\hbar d_i))\subset\C\setminus\sigma_i(L(\uP))$.
        \item For all $1\leq r\leq p$, $\zeros(Q_{j_r}(u+\hbar d_{j_r}))\subset\C\setminus\zeros(\braid_{w_r}(\uP)_{j_r})$.
    \end{enumerate}
    If either of these conditions hold, then the representation $L(\uP)\otimes L(\uQ)$ is cyclic.
\end{corollary}


\section{Fundamental representations of \texorpdfstring{$Y_\hbar(\mfsl_n$)}{Y(sln)}}\label{sec:baxter-example}

In this section we will expand upon Example \ref{E:cyclicity} of the previous section and explicitly compute Baxter polynomials associated to the lowest weights of fundamental representations in the case where $\g=\mfsl_n$.

For convenience, let $\baxter_{ij}(u)$ denote the $i$th Baxter polynomial for the $j$th fundamental representation: $\baxter_{i,L_{\fund_j}}^{\mfsl_n}(u)$.
As in \cite[\S 6]{chari_braid_2002}, if we let $\gamma_k := s_1s_2\cdots s_k\in\Wg$, then a reduced expression for the longest element is given by $w_0 = \gamma_{n-1}\gamma_{n-2}\cdots\gamma_1$.
Recall from Example \ref{E:dual-braid-sl3} that there is an automorphism of the Dynkin diagram induced by the action of $w_0$ on the simple roots; in this case, one can show that this automorphism is given by $i\mapsto n-i$.
From this it follows that $\baxter_{ij}(u) = \baxter_{n-i,n-j}(u)$, meaning that in order to compute all of the Baxter polynomials, we need only compute $\baxter_{ij}(u)$ where $i\geq j$.
Alternatively, this means we need only compute $\baxter_{ij}(u)$ where $j\leq\lceil n/2 \rceil$ since the Baxter polynomials for the $j$th fundamental representation are the same as those of the $(n-j)$th fundamental representation in reverse order.
Either way, making use of the diagram automorphism halves the amount of computation required.

To illustrate the computation of Baxter polynomials using the braid group action, we will consider the particular example of the second fundamental representation of $Y_\hbar(\mfsl_5)$.
This representation has Drinfeld polynomials $\uP = (1, u, 1, 1)$, and the reduced expression for the longest element of the Weyl group described above is
\[w_0 = \gamma_4\gamma_3\gamma_2\gamma_1 = s_1s_2s_3s_4s_1s_2s_3s_1s_2s_1.\]
Recall that the Cartan matrix for $\mfsl_n$ is given by
\[a_{ij} =
\begin{cases}
    2 & \text{if } i=j \\
    -1 & \text{if } \abs{i-j}=1 \\
    0 & \text{otherwise}
\end{cases}\]
and so the relevant formulas for the action of the braid group on $\uP$ are as follows:
\[\braid_i(\uP)_j =
\begin{cases}
    P_j(u-\hbar)^{-1} & \text{if } i=j \\
    P_j(u)P_i(u-\frac{\hbar}{2}) & \text{if } \abs{i-j}=1 \\
    P_j(u) & \text{otherwise}
\end{cases}\]
Now that we have this data, the formula of Theorem \ref{T:baxter} tells us that we can compute the Baxter polynomials for this representation by applying $\braid_{w_0}$ to $\uP$ one $\braid_i$ at a time, then picking out certain polynomials that we obtain along the way.
The table below shows the process of applying $\braid_{w_0}$ to $\uP$: each row represents one step, where the leftmost column gives the index of the braid group operator $\braid_i$ to be applied to the tuple of polynomials given in the remaining columns.
\[\begin{array}{c|cccc}
    \hphantom{m}i\hphantom{m} & P_1 & P_2 & P_3 & P_4 \\
    \hline
    1 & 1 & u & 1 & 1 \\
    2 & 1 & u & 1 & 1 \\
    1 & u-\frac{\hbar}{2} & (u-\hbar)^{-1} & u-\frac{\hbar}{2} & 1 \\
    3 & (u-\frac{3\hbar}{2})^{-1} & 1 & u-\frac{\hbar}{2} & 1 \\
    2 & (u-\frac{3\hbar}{2})^{-1} & u-\hbar & (u-\frac{3\hbar}{2})^{-1} & u-\hbar \\
    1 & 1 & (u-2\hbar)^{-1} & 1 & u-\hbar \\
    4 & 1 & (u-2\hbar)^{-1} & 1 & u-\hbar \\
    3 & 1 & (u-2\hbar)^{-1} & u-\frac{3\hbar}{2} & (u-2\hbar)^{-1} \\
    2 & 1 & 1 & (u-\frac{5\hbar}{2}) & 1 \\
    1 & 1 & 1 & (u-\frac{5\hbar}{2}) & 1 \\
    & 1 & 1 & (u-\frac{5\hbar}{2}) & 1
\end{array}\]
Then using the formula for the Baxter polynomials, we construct $\baxter_{ij}(u)$ by taking the product of the polynomials in the column labeled $P_i$ for each row with $i$ in the leftmost column.
Doing so for each $i$, we obtain:
\[\baxter_{12}(u) = u-\frac{\hbar}{2}, \qquad
\baxter_{22}(u) = u(u-\hbar), \qquad
\baxter_{32}(u) = (u-\frac{\hbar}{2})(u-\frac{3\hbar}{2}), \qquad
\baxter_{42}(u) = u-\hbar.\]
By the discussion at the beginning of this section, this also gives us the Baxter polynomials for the third fundamental representation of $Y_\hbar(\mfsl_5)$:
\[\baxter_{12} = \baxter_{43}, \qquad
\baxter_{22} = \baxter_{33}, \qquad
\baxter_{32} = \baxter_{23}, \qquad
\baxter_{42} = \baxter_{13}.\]

Something to note is that in our choice of reduced expression for $w_0$, the simple reflection $s_i$ occurs $n-i$ times for each $1\leq i\leq n-1$.
This number of occurrences is the number of factors in our formula for $Q_{ij}(u)$, but some of these factors may be $1$, so in particular this means that $Q_{ij}(u)$ has degree at most $n-i$.

Another interesting property of this choice of reduced expression for $w_0$ of $\mfsl_n$ is that it contains reduced expressions for $w_0$ of $\mfsl_m$ for all $m\leq n$: removing $\gamma_{n-1}$ from the left gives us $w_0=\gamma_{n-2}\cdots\gamma_1$ for $\mfsl_{n-1}$, and so on.
Because of the way we construct the Baxter polynomials using our formula as above, this means that the computation for the $j$th fundamental representation of $Y_\hbar(\mfsl_n)$ also encodes the computation for the $j$th fundamental representation of $Y_\hbar(\mfsl_m)$ with $m\leq n$, as long as $j<m$.
For example, we may obtain the Baxter polynomials for the second fundamental representation of $Y_\hbar(\mfsl_4)$ using the above table by considering only the rows before the first occurrence of $4$ in the leftmost column, and only the first three components of the tuple of polynomials:
\[\baxter_{12}(u) = u-\frac{\hbar}{2}, \qquad
\baxter_{22}(u) = u(u-\hbar), \qquad
\baxter_{32}(u) = u-\frac{\hbar}{2}.\]
Note that this time we do not get the Baxter polynomials for another fundamental representation, since the diagram automorphism for $\mfsl_4$ maps $2$ to $2$.
It follows from this discussion that  $\baxter_{ij}^{\mfsl_m}(u)$ divides $\baxter_{ij}^{\mfsl_n}(u)$ for all $m\leq n$.

Lastly, we note that there is another explicit formula for $\baxter_{ij}(u)$ in the case where $\g=\mfsl_n$ given in \cite[Cor. 5.5]{gautam_poles_2023}:
\[\baxter_{ij}(u) = \prod_{b\in\mathbf{J}_{ij}}\left(u-\hbar\left(\frac{i+j}{2}-b\right)\right),\]
where $\mathbf{J}_{ij}$ denotes the $\Z$-valued interval
\[\mathbf{J}_{ij} := [i+j+1-n, i] \cap [1, j].\]
This formula explains why in the above examples (and indeed in all cases), the roots of $\baxter_{ij}(u)$ differ by a factor of $\hbar$.
