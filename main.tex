\documentclass{uofsthesis-cs}
% main packages
\usepackage{amsmath,amssymb,amsthm}
\usepackage{mathtools}
\usepackage{hyperref}
\usepackage[shortlabels]{enumitem}

% fonts
\usepackage{mathrsfs}
\usepackage[scr=boondoxo]{mathalpha}
\usepackage{upgreek}

% theorem environments
\theoremstyle{theorem}
\newtheorem{theorem}{Theorem}[section]
\newtheorem{corollary}[theorem]{Corollary}
\newtheorem{lemma}[theorem]{Lemma}
\newtheorem{proposition}[theorem]{Proposition}

\theoremstyle{definition}
\newtheorem{definition}[theorem]{Definition}
\newtheorem{example}[theorem]{Example}
\newtheorem{remark}[theorem]{Remark}

% numbers
\newcommand{\C}{\mathbb{C}}
\newcommand{\N}{\mathbb{N}}
\newcommand{\Q}{\mathbb{Q}}
\newcommand{\R}{\mathbb{R}}
\newcommand{\Z}{\mathbb{Z}}

% Lie algebras
\newcommand{\g}{\mathfrak{g}}
\newcommand{\h}{\mathfrak{h}}
\newcommand{\n}{\mathfrak{n}}
\newcommand{\mfgl}{\mathfrak{gl}}
\newcommand{\mfsl}{\mathfrak{sl}}
\newcommand{\Ug}{U(\g)}
\newcommand{\I}{\mathbf{I}}

% braid groups and Yangians
\newcommand{\UqLg}[1][]{U_q^{#1}(L\g)}
\newcommand{\Yhg}[1][]{Y_\hbar^{#1}(\g)}
\newcommand{\ul}{\underline{\lambda}}
\newcommand{\um}{\underline{\mu}}
\newcommand{\uP}{\underline{P}}
\newcommand{\uQ}{\underline{Q}}
\newcommand{\Bg}{\mathscr{B}_\g}
\newcommand{\Wg}{\mathscr{W}_\g}
\newcommand{\braid}{\uptau}
\newcommand{\mbraid}{\mathsf{T}}
\newcommand{\rmat}{\mathcal{R}}
\newcommand{\q}{\mathsf{q}}
\newcommand{\baxterV}{\mathcal{Q}_{i,V}^{\g}}
\newcommand{\baxterP}{\mathcal{Q}_{i,\uP}^w}
\newcommand{\zeros}{\mathsf{Z}}

% arrows
\newcommand{\inj}{\hookrightarrow}
\newcommand{\surj}{\twoheadrightarrow}
\newcommand{\iso}{\xrightarrow{\sim}}

% operators
\DeclareMathOperator{\ad}{ad}
\DeclareMathOperator{\End}{End}
\DeclareMathOperator{\id}{id}
\DeclareMathOperator{\gr}{gr}
\DeclareMathOperator{\Hom}{Hom}

% delimiters
\DeclarePairedDelimiter{\abs}{\lvert}{\rvert}
\DeclarePairedDelimiter{\db}{[\![}{]\!]}
\DeclarePairedDelimiter{\dparen}{(\!(}{)\!)}

% arrays
\newcommand{\bmat}[2][c]{\begin{bmatrix*}[#1]#2\end{bmatrix*}}


\title{Braid groups and Baxter polynomials}
\author{Noah Friesen}
\degree{\MSc}
\defencedate{July 2024}
\department{Mathematics and Statistics}

\ptuaddress{Head of the Department of Mathematics and Statistics\\
University of Saskatchewan\\
142 McLean Hall, 106 Wiggins Road\\
Saskatoon, Saskatchewan S7N 5E6 Canada
}

\abstract{
It is well known that the braid group $\Bg$ of a simple Lie algebra $\g$ acts on integrable representations of $\g$ via products of exponentials of its Chevalley generators.
In particular, the Yangian $\Yhg$ is an integrable representation of $\g$, so there is an action of $\Bg$ on this space.
We show that modifying this action induces an action of $\Bg$ on a certain commutative subalgebra of $\Yhg$ by Hopf algebra automorphisms.
By dualizing this modified action, we recover an action of $\Bg$ on tuples of rational functions defined in the work of Y. Tan.
Using this dual action, we prove a conjecture of S. Gautam and C. Wendlandt that the two sufficient conditions for the tensor product of finite-dimensional irreducible representations of $\Yhg$ to be cyclic are identical.
One of these conditions involves the aforementioned action of $\Bg$ on rational functions, and the other involves roots of the Baxter polynomials, which have many interesting properties and ties to mathematical physics.
}

\acknowledgements{
I would like to thank my supervisors, Prof. Alex Weekes and Prof. Curtis Wendlandt, for their invaluable guidance over the past two years.
I am extremely grateful for their support and this thesis would not have been possible without their direction and feedback.
I would also like to thank Prof. Steven Rayan for his exceptional instruction during my coursework.

I gratefully acknowledge the financial support I have received from the Natural Sciences and Engineering Research Council of Canada (NSERC) through the Canada Graduate Scholarships (CGS M) program.

Finally, I thank my family for their love and support.
}


\begin{document}

\maketitle
\frontmatter

\chapter{Introduction}

\section{Braid groups and Yangians}

Given a simple Lie algebra $\g$ over $\C$, we may consider its associated braid group $\Bg$.
The generators of this group satisfy the same braid relations as those of the Weyl group $\Wg$, but the generators of $\Bg$ are all of infinite order.
When $\g = \mfsl_n$, the braid group $\Bg$ coincides with Artin's braid group on $n$ strands, and one can visualize elements of this group quite well using diagrams of physical braids.
In this case, imposing the additional relation that the generators of the group have order $2$ gives us the symmetric group $S_n$, which is the Weyl group of $\mfsl_n$.

Much like the Weyl group, the braid group plays an important role in the representation theory of $\g$.
More precisely, it is well known that the generator $\braid_i$ of $\Bg$ acts on integrable representations $V$ of $\g$ as
\[\exp(e_i)\exp(-f_i)\exp(e_i)\]
where $e_i$ and $f_i$ are the usual Chevalley generators of $\g$ and $i\in\I$ indexes the Cartan matrix (or equivalently, the nodes of the Dynkin diagram) of $\g$.
The integrability of $V$ is needed so that $e_i$ and $f_i$ act nilpotently on every vector of $V$, so that the exponentials above truncate after finitely many terms.
An important property of this braid group action is that it sends a vector in $V$ of weight $\lambda$ to one of weight $w(\lambda)$ for some $w$ in the Weyl group.
Moreover, this action actually preserves the dimension of the weight spaces, and so all of the weight spaces with weight of the form $w(\lambda)$ are isomorphic.
We make use of this property in Chapter \ref{chap:weights} to prove some key lemmas.

For our purposes, the most important integrable representation of $\g$ is an infinite-dimensional Hopf algebra called the \emph{Yangian}, denoted $\Yhg$.
This algebra is a type of affine quantum group, and is a deformation of the universal enveloping algebra of the current algebra $\g[t]$.
Yangians originally arose in the theory of integrable systems, and their representation theory has been studied by many since the seminal paper \cite{drinfeld_hopf_1985} of the Fields medalist V. Drinfeld.
They also have many applications to the areas of geometry and mathematical physics, and their name honours physicist C. N. Yang.

In Chapter \ref{chap:yangians}, after reviewing some basic structure and representation theory of $\g$, we recall a presentation for the Yangian known as \emph{Drinfeld's new presentation}.
The generators and relations therein are very reminiscent of the Chevalley--Serre presentation of $\g$ itself.
Indeed, using this presentation it is easy to see that there is a natural inclusion of $\g$ and its enveloping algebra into $\Yhg$ given by sending the generators of the former to the degree-zero generators of the latter.
Because of this inclusion, we get an adjoint action of $\g$ on $\Yhg$ which is integrable, and so there is an action of the braid group $\Bg$ on $\Yhg$.
This action has appeared in a number of contexts; see for example \cite{guay_coproduct_2018, kodera_braid_2019, weekes_highest_2016}.
Also note that because of this, we may regard any representation of $\Yhg$ as a representation of $\g$.

In a later section of Chapter \ref{chap:yangians}, we introduce an alternative set of generating series for the Yangian given by A. Gerasimov et al. in \cite{gerasimov_class_2005}.
These generating series play a crucial role in our results, as they allow us to compute explicit formulas for the action of the braid group on the Yangian.
We are then able to compute formulas for our modified braid group action described in the section below, and ultimately use these formulas to find a new factorization of the so-called \emph{Baxter polynomials}, giving us our main results.


\section{Modified braid group action}

One of the main motivations for this research was to develop a framework for better understanding an action of the braid group $\Bg$ on tuples of rational functions defined in the work \cite{tan_braid_2015} of Y. Tan.
Drawing inspiration from the results of V. Chari for quantum loop algebras given in \cite{chari_braid_2002}, Tan defined operators directly and proved that they indeed give an action of $\Bg$.
In Chapters \ref{chap:braidgroup} and \ref{chap:weights}, we recover Tan's action starting with the action of $\Bg$ on the Yangian $\Yhg$ described in the previous section, and in doing so prove that Tan's action computes the eigenvalues of certain generators of $\Yhg$ on the \emph{extremal weight spaces} of a representation of $\Yhg$---weight spaces that lie in the Weyl group orbit of the highest weight space.

We now describe the key results of Chapter \ref{chap:braidgroup}.
Much like the Lie algebra $\g$, the Yangian $\Yhg$ has a triangular decomposition: one can partition the generators into raising, lowering, and Cartan operators and consider the spaces generated by each of these.
We denote by $\Yhg[0]$ the commutative subalgebra of the Yangian generated by the Cartan generators $\xi_{i,r}$ for all $i\in\I$ and $r\in\N$.
Note that this subalgebra deforms the universal enveloping algebra of the current algebra $\h[t]$ where $\h$ is a Cartan subalgebra of $\g$.
We may package the generators of $\Yhg[0]$ into series $\xi_i(u)$ in a formal variable $u^{-1}$ so that the generator $\xi_{i,r}$ is the coefficient of $u^{-r-1}$; there is a unique Hopf algebra structure on $\Yhg[0]$ such that each of these series is grouplike.
Using properties of the triangular decomposition of $\Yhg$, we show that there is a natural linear projection
\[\Pi:\Yhg\to\Yhg[0]\]
and we use this projection to introduce what we call the \emph{modified braid group operators}:
\[\mbraid_i := \Pi\circ\braid_i\big|_{\Yhg[0]}.\]
By the weight-permuting property of the braid group action on representations of $\g$ described in the previous section, the braid group action maps elements of $\Yhg[0]$ to elements of $\Yhg$ which are sums of monomials containing the same number of raising and lowering operators.
Then applying the projection $\Pi$ discards terms that are not in $\Yhg[0]$, so the above operators are endomorphisms of $\Yhg[0]$.
In fact, they have many remarkable properties which we summarize in the theorem below.
This theorem constitutes the first main result of the thesis.

\begin{theorem}\label{T:intro-main1}
    The modified braid group operators $\mbraid_i$ have the following properties:
    \begin{enumerate}
        \item They are Hopf algebra automorphisms of $\Yhg[0]$ that satisfy the braid relations, i.e., they define an action of $\Bg$ on $\Yhg[0]$.
        \item They are uniquely determined by the following formulas, for each $j\in\I$:
        \[\mbraid_i(\xi_j(u)) = \xi_j(u)\prod_{k=0}^{\abs{a_{ij}}-1} \xi_i\left(u-\frac{\hbar d_i}{2}(\abs{a_{ij}}-2k)\right)^{(-1)^{\delta_{ij}}}\]
        where $a_{ij}$ and $d_i$ are the entries of the Cartan matrix and the symmetrizing integers of $\g$, respectively.
        \item The diagonal factor $\rmat^0(z)\in \Yhg[0]^{\otimes 2}\db{z^{-1}}$ of the universal $R$-matrix of $\Yhg$ is $\Bg$-invariant:
        \[(\mbraid_i\otimes\mbraid_i)(\rmat^0(z)) = \rmat^0(z).\]
    \end{enumerate}
\end{theorem}

Let us now turn to the results of Chapter \ref{chap:weights}, in which we recover Tan's braid group action.
Using the modified braid group operators above, we define an action of $\Bg$ on the linear dual of $\Yhg[0]$.
The group of algebra homomorphisms $\Yhg[0]\to\C$ is a subrepresentation, and this group is isomorphic to the group $(1+u^{-1}\C\db{u^{-1}})^\I$ of $\I$-tuples of formal series in $u^{-1}$ with constant coefficient $1$.
To see this, notice that we can define an algebra homomorphism $\Yhg[0]\to\C$ by choosing a complex number for each generator $\xi_{i,r}$ to be mapped to; applying this homomorphism to the generating series $\xi_i(u)$ gives the desired isomorphism.
The formula for this dual action of $\Bg$ on a tuple of series $(\lambda_i(u))_{i\in\I}$ is the same as the formula in the second part of the theorem above, replacing $\xi$ with $\lambda$.
It is also the same formula as the one for the action of $\Bg$ on the group $(\C(u)^\times)^\I$ of $\I$-tuples of rational functions defined by Tan.

In Section \ref{sec:extending}, we reconcile the underlying representation spaces of our dual braid group action and Tan's.
We do this by extending our dual action to the group $M^\I$ of $\I$-tuples of monic Laurent series using the theory of formal additive difference equations.
We then further extend to the group $(\C\dparen{u^{-1}}^\times)^\I$ of $\I$-tuples of arbitrary Laurent series by taking the product of $M^\I$ and the group $(\C^\times)^\I$ on which there is a natural action of the Weyl group $\Wg$.
The group of rational functions on which Tan's action is defined is then recovered as a subrepresentation of this largest space.

In Section \ref{sec:extremal-weights}, we show that the dual braid group action determines the weights of the \emph{extremal vectors} of a representation of the Yangian.
Finite-dimensional irreducible representations of $\Yhg$ are classified in a way very similar to those of $\g$ in that they are determined by highest weights: for such a representation $V$ there is a \emph{highest weight vector} $v\in V$ that is killed by the raising operators and is an eigenvector for the Cartan operators, say $\xi_{i,r}\cdot v = \lambda_{i,r}v$ for some $\lambda_{i,r}\in\C$.
Packaging these eigenvalues into series as we did for the generators themselves and collecting them into a tuple, we let $\ul = (\lambda_i(u))_{i\in\I}$ denote the highest weight of $V$.
Recall that we can regard $V$ as a representation of $\g$, and doing so we define the extremal vectors of $V$ to be those of $\g$-weight $w(\lambda)$ for $w\in\Wg$, where $\lambda$ is the $\g$-weight of the highest weight vector $v$.
Note that because of the aforementioned dimension-preserving property of the braid group action on weight spaces, each extremal weight space is one-dimensional.
The main result of this section tells us that the eigenvalue of the generating series $\xi_i(u)$ (i.e., the $\Yhg$-weight) on the extremal vector of $\g$-weight $w(\lambda)$ is given by the dual braid group action on the $\Yhg$-highest weight.
This eigenvalue is $\braid_w(\ul)_i$, where $\braid_w\in\Bg$ is defined by taking any reduced expression for $w\in\Wg$ and replacing each Weyl group generator with its corresponding braid group generator, and the subscript $i$ denotes taking the $i$th component of the tuple.

Something to note (and that is explained in detail in \cite[\S 6]{friesen_braid_2024}) is that the results of this section also apply to the parallel case of the quantum loop algebra $\UqLg$.
In this setting, there is an action of $\Bg$ on $\UqLg$ via operators given in \cite{lusztig_introduction_2010} by G. Lusztig, and we can modify these operators as above to obtain an action on a commutative subalgebra $\UqLg[0]$.
Dualizing this modified action, we recover an action of $\Bg$ studied by Chari in \cite{chari_braid_2002}.
We can then prove the analogue of Theorem \ref{T:intro-main1} in this setting, and further we can show that this modified action is compatible with the homomorphism from $\UqLg$ to a completion of $\Yhg$ that was given in \cite{gautam_yangians_2013}.
By this, we mean that modifying the action on $\UqLg$ then applying the homomorphism gives us the same action on $\Yhg$ that we obtain by modifying the action on $\Yhg$.


\section{Baxter polynomials and cyclicity}

The other main motivation for studying these braid group actions arises from the following fact: the tensor product of two finite-dimensional irreducible representations of the Yangian will almost always be cyclic, i.e., highest weight.
This has been studied extensively and has applications in many areas; see for example \cite{chari_yangians_1996, guay_local_2015, molev_yangians_2007, nazarov_irreducibility_2002, akasaka_finite_1997}.
In the literature, there are two sufficient conditions for this cyclicity property.
The first of these was established by Tan in \cite{tan_braid_2015} using a construction that appeared earlier for the quantum loop algebra in \cite{chari_braid_2002}, and involves the braid group action on rational functions described in the previous section.
The second condition was established by S. Gautam and C. Wendlandt in \cite{gautam_poles_2023} and involves the \emph{Baxter polynomials}.
It was conjectured in the latter paper that these two conditions are identical, and we prove in Chapter \ref{chap:baxter} by giving a new factorization of the Baxter polynomials that this is indeed the case.

Let us describe the first cyclicity condition of \cite{tan_braid_2015}.
Finite-dimensional irreducible representations of $\Yhg$ are parametrized by $\I$-tuples of monic polynomials $\uP=(P_i(u))_{i\in\I}$ called \emph{Drinfeld polynomials} which encode the highest weight.
We denote by $L(\uP)$ the representation of $\Yhg$ with Drinfeld polynomials $\uP$.
Choose any reduced expression $w_0 = s_{j_1}\cdots s_{j_p}$ for the longest element of the Weyl group $\Wg$, and let $w_r\in\Wg$ be the element of the Weyl group obtained by removing $r$ generators from the left of this reduced expression.
Then the representation $L(\uP)\otimes L(\uQ)$ is cyclic if
\[\zeros(Q_{j_r}(u+\hbar d_{j_r}))\subset\C\setminus\zeros(\braid_{w_r}(\uP)_{j_r})\]
for each $r$, where $\zeros(p(u))$ denotes the set of zeros of a polynomial $p(u)$.
I.e., the tensor product is cyclic as long as certain Drinfeld polynomials of the right tensor leg and certain polynomials arising from the braid group action on the Drinfeld polynomials of the left tensor leg do not have zeros in common.

Now we will describe the second cyclicity condition of \cite{gautam_poles_2023}.
Similar to how we defined the Cartan generating series $\xi_i(u)$ of $\Yhg$ in the previous section, we may do the same for the raising operators $x^+_i(u)$ and the lowering operators $x^-_i(u)$.
Each of these series acts on a representation $L(\uP)$ as the expansion at $u=\infty$ of some $\End(L(\uP))$-valued rational function in $u$.
If we denote the set of poles of these three rational functions by $\sigma_i(L(\uP))$, then the representation $L(\uP)\otimes L(\uQ)$ is cyclic if
\[\zeros(Q_i(u+\hbar d_i))\subset\C\setminus\sigma_i(L(\uP))\]
for all $i\in\I$.
I.e., the tensor product is cyclic as long as the zeros of the Drinfeld polynomials of the right tensor leg are not poles of the corresponding generating series of the left tensor leg.
Moreover, the tensor product is irreducible if the same condition also holds with $\uP$ and $\uQ$ interchanged.

It was proven in \cite{gautam_poles_2023} that the set of poles $\sigma_i(L(\uP))$ exactly coincides with the set of roots of a distinguished polynomial $\baxter_{i,L(\uP)}^\g(u)$ called the \emph{Baxter polynomial}.
These polynomials first appeared in generality in the work \cite{frenkel_baxters_2015} of E. Frenkel and D. Hernandez, though they have been studied in more specialized contexts for a long period of time.
Their name honours physicist R. J. Baxter, who first encountered polynomials of this type in his paper \cite{baxter_partition_1972}.
The motivation behind Baxter's original work, and much of the literature on Baxter polynomials, is based in mathematical physics, as these polynomials encode information about certain models describing physical phenomena.
For us, $\baxter_{i,L(\uP)}^\g(u)$ arises as the eigenvalue of a certain \emph{abelian transfer operator} on the lowest weight vector of $L(\uP)$.
% TODO add sentence about polynomiality of the transfer operator here, something like below:
% This operator $T_i(u)$ is the solution to a certain difference equation, and is remarkable in that its eigenvalues are polynomials; see Section 5.1.
In \cite{gautam_poles_2023}, a formula for this polynomial was given in terms of the Drinfeld polynomials $\uP$ and a matrix known as the \emph{quantum Cartan matrix} of $\g$.

In Section \ref{sec:baxter-extremal}, we use the results of Chapters \ref{chap:braidgroup} and \ref{chap:weights} to obtain a factorization of the Baxter polynomials in terms of polynomials arising from the action of the braid group on the Drinfeld polynomials.
Because of the discussion above---namely that the zeros of the Baxter polynomials are exactly the poles of the generating series---this factorization implies that the two cyclicity criteria for tensor products are identical.
The theorem below summarizes these findings and provides the second main result of the thesis.

\begin{theorem}\label{T:intro-main2}
    The Baxter polynomial $\baxter_{i,L(\uP)}^\g(u)$ admits the following factorization:
    \[\baxter_{i,L(\uP)}^\g(u) = \prod_{r:j_r=i}\braid_{w_r}(\uP)_i.\]
    Consequently, the sufficient conditions for the cyclicity of any tensor product $L(\uP)\otimes L(\uQ)$ obtained in \cite{tan_braid_2015} and \cite{gautam_poles_2023} are identical.
\end{theorem}

Something to note is that the above theorem actually provides a factorization for the Baxter polynomial associated to any extremal weight, not just the lowest weight.
Instead of the longest element $w_0$ in the above construction, we may take a reduced expression for any element $w\in\Wg$ and define the elements $w_r$ in the same way.

Lastly, in Section \ref{sec:baxter-example}, we explicitly compute Baxter polynomials for fundamental representations in the case where $\g=\mfsl_n$.
There are many symmetries which make the computation nicer in this case, and this section provides a good example of how one might make use of the main results of the thesis.

\chapter{Yangians}

\section{Simple Lie algebras}

We review the structure of a simple Lie algebra, and the classification of its irreducible representations by dominant integral weights.


\section{Yangians}

\subsection{Structure}

We give the generators and relations of the Yangian associated to a simple Lie algebra.

\subsection{Representation theory}

We review the classification of the irreducible representations of the Yangian by Drinfeld polynomials.

\chapter{Braid groups}

\section{Braid group action}

Here we review the action of the braid group of a simple Lie algebra on its integrable representations, and prove some important properties.

\section{Modified braid group action}

Here we define modified braid group operators that act on the Cartan part of the Yangian.

\section{Action on the universal \texorpdfstring{$R$}{R}-matrix}

Here we show how the braid group acts on the universal $R$-matrix and its $R^0$ part.

\chapter{Dual braid group action and weights}

\section{Dualizing the braid group action}

In this section, we dualize the modified action of the braid group $\Bg$ on the Cartan part $\Yhg[0]$ of the Yangian that we constructed in the previous chapter.
In doing so, we obtain an action of $\Bg$ by automorphisms on the group $(1+u^{-1}\C\db{u^{-1}})^\I$ of $\I$-tuples of formal series in $u^{-1}$ with constant coefficient $1$.

Let $\vartheta$ be the group antiautomorphism of $\Bg$ uniquely determined by $\vartheta(\braid_i)=\braid_i$ for all $i\in\I$.
Notice that if $w = s_{i_1}\cdots s_{i_\ell}$ is a reduced expression for $w\in\Wg$, then $w^{-1} = s_{i_\ell}\cdots s_{i_1}$, so it follows that $\vartheta(\braid_w)=\braid_{w^{-1}}$ for any $w\in\Wg$.
We can use this antiautomorphism to define an action of $\Bg$ on the linear dual $\Yhg[0]^*$ as follows: for $\sigma\in\Bg$, $y\in\Yhg[0]$, and $f\in\Yhg[0]^*$, let
\[\sigma(f)(y) = f(\vartheta(\sigma)\cdot y).\]
This action is uniquely determined by the requirement that $\braid_i$ operates as the transpose of $\mbraid_i$, i.e., for all $i\in\I$ and $f\in\Yhg[0]^*$, we have
\[\braid_i(f) = \mbraid_i^*(f) = f\circ\mbraid_i.\]

Recall the Drinfeld Hopf algebra structure on $\Yhg[0]$ from Section \ref{ssec:Y-Hopf}.
This induces a commutative algebra structure on $\Yhg[0]^*$ where the unit is given by the counit $\varepsilon_D$ and the product is given by $\Delta_D^*$, the transpose of the coproduct.
Since each modified braid group operator $\mbraid_i$ is a coalgebra homomorphism, $\Bg$ acts on $\Yhg[0]^*$ by algebra automorphisms.
Since each $\mbraid_i$ is an algebra automorphism, the space of algebra homomorphisms $\Hom_{\text{Alg}}(\Yhg[0],\C)$ is a subrepresentation of $\Yhg[0]^*$.
This space is a subgroup of the group of units in $\Yhg[0]^*$, and we have an isomorphism of groups
\[\Hom_{\text{Alg}}(\Yhg[0],\C)\to(1+u^{-1}\C\db{u^{-1}})^\I, \qquad f\mapsto(f(\xi_i(u)))_{i\in\I}.\]
The following proposition gives a formula for the action of $\Bg$ on this group.

\begin{proposition}\label{P:Tan-formula}
    If $\ul = (\lambda_i(u))_{i\in\I}\in (1+u^{-1}\C\db{u^{-1}})^\I$, then for all $i,j\in\I$, the $i$th component of $\braid_j(\ul)$ is given by
    \[\braid_j(\ul)_i = \lambda_i(u)\prod_{k=0}^{\abs{a_{ji}}-1}\lambda_j\left(u-\frac{\hbar d_j}{2}(\abs{a_{ji}}-2k)\right)^{(-1)^{\delta_{ij}}}.\]
\end{proposition}
\begin{proof}
    By the discussion above, we have
    \[\braid_j(\ul)_i = \braid_j(\ul)(\xi_i(u)) = \ul(\mbraid_j(\xi_i(u)))\]
    where we view $\ul$ as an algebra homomorphism in $\Yhg[0]^*$ via the above isomorphism.
    Then using the formula for $\mbraid_j(\xi_i(u))$ given in Corollary \ref{C:tau-xi}, we see that the claimed formula holds.
\end{proof}

The formula for the action of $\Bg$ on $(1+u^{-1}\C\db{u^{-1}})^\I$ given in the above proposition is the same as the formula for the action of $\Bg$ on the group $(\C(u)^\times)^\I$ of $\I$-tuples of rational functions defined in \cite[Prop. 3.1]{tan_braid_2015}.
This is why we have used the antiautomorphism $\vartheta$ to define the dual action rather than the usual antiautomorphism $\sigma\mapsto\sigma^{-1}$ of $\Bg$.


\section{Extending the representation space}\label{sec:extending}

In this section, we will extend our dual braid group action to a larger space in order to reconcile our action and Tan's (see the end of the previous section).

Let $M$ be the subgroup of $\C\dparen{u^{-1}}^\times$ consisting of monic Laurent series in $u^{-1}$:
\[M := \bigcup_{k\in\Z}u^{k}(1+u^{-1}\C\db{u^{-1}}).\]
Define the \emph{degree} of $\lambda(u)\in M$ to be the unique integer $k$ for which $u^{-k}\lambda(u)\in 1+u^{-1}\C\db{u^{-1}}$.

\begin{lemma}\label{L:diff-eqn}
    Let $\lambda(u) = 1+\hbar\sum_{r\geq 0}\lambda_ru^{-r-1}$ be a series in $1+u^{-1}\C\db{u^{-1}}$ and let $d\in\C^\times$.
    Then the formal difference equation
    \[\frac{\mu(u+\hbar d)}{\mu(u)} = \lambda(u)\]
    has a solution in $M$ if and only if $\lambda_0\in d\Z$.
    In this case, $\mu(u)$ is unique and has degree $\lambda_0/d$.
\end{lemma}
\begin{proof}
    Suppose $\mu(u)$ has degree $k$, and write $\mu(u)=u^k\dot{\mu}(u)$ where $\dot{\mu}\in 1+u^{-1}\C\db{u^{-1}}$.
    Then the difference equation becomes
    \[\left(1+\hbar\frac{d}{u}\right)^k \frac{\dot{\mu}(u+\hbar d)}{\dot{\mu}(u)} = \lambda(u).\]
    Since all series involved have constant coefficient $1$, we may make use of the formal series logarithm, which is defined for a series $f$ in $u^{-1}$ as follows:
    \[\log(f(u)) := \sum_{n\geq 1}\frac{(-1)^{n+1}}{n}(f(u)-1)^n.\]
    Taking the logarithm of both sides of our difference equation above, we obtain
    \[k\log(1+\hbar\frac{d}{u}) + b(u+\hbar d) - b(u) = \log(\lambda(u)),\]
    where
    \[b(u) = \sum_{r\geq 0}b_ru^{-r-1} := \log(\dot{\mu}(u)).\]
    Taking the coefficient of $u^{-1}$ on both sides, we see that $k\hbar d = \hbar\lambda_0$.
    Thus if a solution exists, we must have $\lambda_0\in d\Z$ and $k = \lambda_0/d$.

    Conversely, suppose that $\lambda_0\in d\Z$.
    Take $k = \lambda/d$, then we can solve the above equation for $b(u)$ recursively by expressing $b_{r}$ in terms of $b_s$ for $s<r$ and the coefficients of $k\log(1+\hbar\frac{d}{u})$ and $\log(\lambda(u))$.
    This allows us to construct the solution $\mu(u)$.
\end{proof}

Let $(\fund_i)_{i\in\I}$ denote the \emph{fundamental weights} of $\g$, which form a basis of $\h^*$ dual to the basis $(h_i)_{i\in\I}$ of $\h$, i.e., $\fund_i(h_j) = \delta_{ij}$ for all $i,j\in\I$.
Let $\Lambda := \bigoplus_{i\in\I}\Z\fund_i$ denote the \emph{weight lattice} of $\g$.
Recall from Section \ref{sec:braid} that there is an action of $\Wg$ on $\h^*$, so we may view $\Lambda$ as a representation of $\Bg$ through this action.
We define the \emph{degree} of an element of $M^\I$ via the following group homomorphism:
\[\deg:M^\I\to\Lambda, \qquad (\lambda_i(u))_{i\in\I}\mapsto\sum_{i\in\I}\deg(\lambda_i(u))\fund_i.\]
Next, for each $a\in\C$ let $\q^a$ be the group automorphism of $M$ given by the translation $\lambda(u)\mapsto\lambda(u+a\frac{\hbar}{2})$.
Similarly, given a diagonal matrix $A = (a_i)_{i\in\I}$, let $\q^A$ be the group automorphism of $M^\I$ given by
\[\q^A(\lambda(u))_{i\in\I} = (\q^{a_i}\lambda_i(u))_{i\in\I} = (\lambda_i(u+a_i\tfrac{\hbar}{2}))_{i\in\I}.\]
We will only be interested in the case where $A = 2D$, where $D = (d_i)_{i\in\I}$ is the diagonal matrix of symmetrizing integers for $\g$.
Hence the map $\q^{2D}$ is given by $\lambda_i(u)\mapsto\lambda_i(u+\hbar d_i)$ for each $i\in\I$.
Also note that for all $\um\in M^\I$, the element $(\q^{2D}\um)\um^{-1}$ belongs to $(1+u^{-1}\C\db{u^{-1}})^\I$, so we may apply our dual braid group action to it.

\begin{proposition}\label{P:Bg-action-M}
    For all $\sigma\in\Bg$ and $\um = (\mu_i(u))_{i\in\I} \in M^\I$, there exists a unique element $\sigma(\mu)\in M^\I$ satisfying
    \[\sigma\left(\frac{\q^{2D}\um}{\um}\right) = \frac{\q^{2D}\sigma(\um)}{\sigma(\um)}.\]
    This defines an action of $\Bg$ on $M^\I$ by group automorphisms for which $1+u^{-1}\C\db{u^{-1}}^\I$ is a subrepresentation and $\deg:M^\I\to\Lambda$ is a module homomorphism:
    \[\deg\sigma(\um) = \sigma(\deg\um).\]
\end{proposition}
\begin{proof}
    Let $\ul = (\lambda_i(u))_{i\in\I}$ denote the element $(\q^{2D}\um)\um^{-1}$, and write $\lambda_i(u) = 1+\hbar\sum_{r\geq 0}\lambda_{i,r}u^{-r-1}$.
    There is an isomorphism
    \[\h^*\to\C^\I, \qquad f\mapsto (f(h_i))_{i\in\I} = (f, \alpha_i)_{i\in\I},\]
    so we have an action of $\Bg$ on $\C^\I$.
    Then
    \[\sigma(\ul) = 1 + \hbar\sigma\cdot(\lambda_{i,0})u^{-1} + O(u^{-2}),\]
    and using the action of $\Bg$ on $\C^\I$, we have
    \[\sigma\cdot(\lambda_{i,0})_{i\in\I} = \left(\sum_{j\in\I}d_j^{-1}\lambda_{j,0}(\alpha_i,\sigma(\fund_j))\right)_{i\in\I}\]
    By the previous lemma, we have $\lambda_{j,0} = d_j\deg\mu_j(u)$ for all $j\in\I$, so the above is equal to $(\alpha_i,\sigma(\deg\um))_{i\in\I}$.
    Since the action of $\Bg$ on $\h^*$ preserves $\Lambda$, we have $(\alpha_i,\sigma(\deg\um))\in d_i\Z$ for all $i\in\I$.
    Thus by the previous lemma, $\sigma(\um)$ exists, is unique, and has degree $\sigma(\deg\um)$.
\end{proof}

By considering the unique simply connected Lie group with Lie algebra $\g$, one can show that there is an action of $\Wg$ by group automorphisms on $(\C^\times)^\I$ given explicitly for all $i,j\in\I$ and $\lambda_i\in\C^\times$ by
\[s_j(\lambda_i)_{i\in\I} = (\lambda_i\lambda_j^{-a_{ji}})_{i\in\I},\]
hence we get an action of $\Bg$ on this group as well; see \cite[Remark 4.5]{friesen_braid_2024} for details.
Then using the action given in the above proposition, we have an action of $\Bg$ on the group $(\C^\times)^\I\times M^\I$.
There is an isomorphism from this group to $(\C\dparen{u^{-1}}^\times)^\I$ given by component-wise multiplication:
\[((\lambda_i)_{i\in\I}, (\mu_i(u))_{i\in\I}) \mapsto (\lambda_i\mu_i(u))_{i\in\I},\]
hence $\Bg$ acts on $(\C\dparen{u^{-1}}^\times)^\I$ by group automorphisms via the formulas of Proposition \ref{P:Tan-formula}.
The space $(\C(u)^\times)^\I$ is a subrepresentation, so we have successfully extended our dual braid group action so that the underlying space agrees with that of \cite[Prop. 3.1]{tan_braid_2015}.


\section{Weights of extremal vectors}

In this section, we will show that the braid group action of the previous section determines the weights of the \emph{extremal vectors} of a representation of the Yangian.

Let $V$ be a finite-dimensional irreducible representation of $\Yhg$ with highest weight $\ul$, and let $\lambda\in\h^*$ be the highest weight of $V$ as a representation of $\g$.
The \emph{extremal vectors} are those which lie in the Weyl group orbit of the highest weight vector, i.e., they are elements of $V_{w(\lambda)}$ for $w\in\Wg$.
Recall that the highest weight space is one-dimensional, so by Proposition \ref{P:tau-wt-space}, each extremal weight space is also one-dimensional.
The following proposition is the main result of this section, and provides a strengthening of \cite[Prop. 4.5]{tan_braid_2015}.

\begin{proposition}\label{P:extremal-weight}
    Let $V$ be as above.
    For all $i\in\I$ and $w\in\Wg$, the eigenvalue of $\xi_i(u)$ on the extremal weight space $V_{w(\lambda)}$ is $\braid_w(\ul)_i$.
    In particular, if $V=L(\uP)$ then this eigenvalue is equal to
    \[\frac{\q^{2d_i}\braid_w(\uP)_i}{\braid_w(\uP)_i}.\]
\end{proposition}
\begin{proof}
    Let $v$ be a highest weight vector of $V$.
    Using the definition of our dual braid group action from the first section of this chapter, we have
    \[\braid_w(\ul)_i = \ul(\mbraid_{w^{-1}}(\xi_i(u))),\]
    so for the first part of the proposition, it suffices to prove that the eigenvalue of $\xi_i(u)$ on the extremal vector $\Omega_w := \braid_{w^{-1}}(v)$ is the same as that of $\mbraid_w(\xi_i(u))$ on the highest weight vector $v$.
    Let $\mu_{i,w}(u)$ denote this former eigenvalue, so that
    \[\xi_i(u)\cdot\Omega_w = \mu_{i,w}(u)\Omega_w.\]
    Applying the (unmodified) braid group operator $\braid^V_w$ to both sides of the above equation, for the right-hand side we obtain $\mu_{i,w}(u)v$.
    For the left-hand side, we have
    \[\braid^V_w(\xi_i(u)\cdot\Omega_w) = \braid^{\Yhg}_w(\xi_i(u))\cdot\braid^V_w(\Omega_w) = \braid^{\Yhg}_w(\xi_i(u))\cdot v\]
    where in the first equality we have used Corollary \ref{C:tau-alg}.
    Now $\braid^{\Yhg}_w$ sends $\xi_i(u)$ to something in the zero weight space $\Yhg_0$ but not necessarily something in $\Yhg[0]$.
    But any terms not in $\Yhg[0]$ have raising operators on the right and thus kill the highest weight vector $v$, so in fact we have
    \[\braid^{\Yhg}_w(\xi_i(u))\cdot v = \mbraid_w(\xi_i(u))\cdot v\]
    as desired, which completes the proof of the first part of the proposition.

    The second part follows from the first part, and that since $\ul = (\q^{2D}\uP)\uP^{-1}$, by Proposition \ref{P:Bg-action-M} we have $\braid_w(\ul) = (\q^{2D}\braid_w(\uP))\braid_w(\uP)^{-1}$.
\end{proof}

We now prove a corollary of the above proposition which further relates our work to that of Tan; cf. \cite[Prop. 4.5]{tan_braid_2015}.

\begin{corollary}\label{C:Tan-monic}
    Let $\uP = (P_i(u))_{i\in\I}$ be a tuple of Drinfeld polynomials and let $V = L(\uP)$.
    For all $i\in\I$ and $w\in\Wg$, define $P_{i,w}(u)\in M$ by
    \[P_{i,w}(u) :=
    \begin{cases}
        \braid_w(\uP)_i & \text{if } w^{-1}(\alpha_i)\in\Phi^+ \\
        \braid_w(\uP)_i^{-1} & \text{if } w^{-1}(\alpha_i)\notin\Phi^+
    \end{cases}\]
    Then $P_{i,w}(u)$ is a monic polynomial in $u$.
\end{corollary}
\begin{proof}
    By \cite[Remark 2.2]{chari_fundamental_1991}, there is a monic polynomial $P^+_{i,w}(u)$ such that the eigenvalue of $\xi_i(u)$ on the extremal weight space $V_{w(\lambda)}$ is given by
    \[\frac{P^+_{i,w}(u+\hbar d_i)^{f_i(w)}}{P^+_{i,w}(u)^{f_i(w)}},\]
    where $f_i(w)$ is $1$ if $w\in\Phi^+$ and $-1$ otherwise.
    By the above proposition and the uniqueness assertion of Lemma \ref{L:diff-eqn}, we have
    \[P^+_{i,w}(u)^{f_i(w)} = \braid_w(\uP)_i = P_{i,w}(u)^{f_i(w)}\]
    hence $P_{i,w}(u)$ coincides with the monic polynomial $P^+_{i,w}(u)$.
\end{proof}

Lastly, we give an explicit example of the dual braid group action defined in this chapter.

\begin{example}\label{E:dual-braid-sl3}
    Let $j\in\I$ and $a\in\C$.
    The representation $L(\uP)$ of $\Yhg$ with Drinfeld polynomials given by
    \[P_i(u) =
    \begin{cases}
        u-a & \text{if } i=j \\
        1 & \text{if } i\neq j
    \end{cases}\]
    is called a \emph{fundamental representation}, and is denoted $L_{\fund_j}(a)$ (or simply $L_{\fund_j}$ in the case where $a=0$) since the highest weight of $L(\uP)$ as a representation of $\g$ is the $j$th fundamental weight $\fund_j$.

    In this example, we will consider the fundamental representation $L_{\fund_1}(a)$ of $Y_\hbar(\mfsl_3)$, which has Drinfeld polynomials $P_1(u) = u-a$ and $P_2(u) = 1$.
    As a representation of $\mfsl_3$, $L_{\fund_1}(a)$ has three weights, which are the linear functionals that send $(h_1,h_2)$ to $(1,0)$, $(-1,1)$, and $(0,-1)$.
    All three of these weights are extremal, since they are equal to $\fund_1$, $s_1(\fund_1)$, and $s_2s_1(\fund_1)$, respectively.
    The last of these three weights is the lowest weight, since it is also equal to $s_1s_2s_1(\fund_1)$ and $w_0 = s_1s_2s_1$ is the longest element of the Weyl group of $\mfsl_3$.

    By Proposition \ref{P:Tan-formula}, we can use the formulas for the modified braid group action we found in Example \ref{E:sl3-mbraid-action} to get the formulas for the dual action on the Drinfeld polynomials.
    For $i,j\in\{1,2\}$ with $i\neq j$, we have
    \[\braid_i(P_i(u)) = \frac{1}{P_i(u-\hbar)}, \qquad \braid_i(P_j(u)) = P_j(u)P_i(u-\tfrac{\hbar}{2}).\]
    Now we can use Proposition \ref{P:extremal-weight} to compute the eigenvalues of $\xi_1(u)$ and $\xi_2(u)$ on the lowest weight space $V_{w_0(\fund_1)}$.
    By the above, we find that
    \begin{align*}
        \braid_{w_0}(\uP) & = \braid_1\braid_2\braid_1(u-a,\ 1) \\
        &= \braid_1\braid_2\left(\frac{1}{u-a-\hbar},\ u-a-\frac{\hbar}{2}\right) \\
        &= \braid_1\left(1,\ \frac{1}{u-a-\frac{3\hbar}{2}}\right) \\
        &= \left(1,\ \frac{1}{u-a-\frac{3\hbar}{2}}\right)
    \end{align*}
    so $\xi_1(u)$ acts on the lowest weight space by $1$, and $\xi_2(u)$ acts by
    \[\frac{u-a-\frac{3\hbar}{2}}{u-a-\frac{\hbar}{2}}.\]
    Note that this is exactly the lowest weight $(\mu_i(u))_{i\in\I}$ that we obtain using the following formula given in \cite[Prop. 3.5]{gautam_poles_2023}: for all $i\in\I$,
    \[\mu_i(u) = \frac{P_{i^*}(u-\hbar\kappa)}{P_{i^*}(u-\hbar\kappa+\hbar d_i)},\]
    where $i\mapsto i^*$ is the involution of the Dynkin diagram of $\g$ induced by $\alpha_{i^*} := -w_0(\alpha_i)$, and $\kappa$ is $1/4$ times the eigenvalue of the Casimir element $C\in\Ug$ on the adjoint representation.
    In this case, one can check that the involution is given by $1\mapsto 2$, and that we have $\kappa = 3/2$.
\end{example}

\chapter{Baxter polynomials and cyclicity}

A representation of $\Yhg$ is said to be \emph{cyclic} if it is generated by a single vector, i.e., if it is a highest-weight representation.
In this chapter, we prove a conjecture from \cite[\S 7.4]{gautam_poles_2023} which states that the two sufficient conditions for the cyclicity and irreducibility of any tensor product $L(\uP)\otimes L(\uQ)$ obtained in \cite{gautam_poles_2023} and \cite{tan_braid_2015} are identical.


\section{Baxter polynomials and poles}\label{sec:baxter-poles}

Let $V$ be a finite-dimensional highest-weight representation of $\Yhg$, and let $\lambda\in\h^*$ be the highest weight of $V$ as a representation of $\g$.
Recall the series $A_i(u)$ of Section \ref{ssec:alt-gen}; for each $i\in\I$ let $\lambda_i^A(u)\in 1+u^{-1}\C\db{u^{-1}}$ denote the eigenvalue of $A_i(u)$ on the highest weight space $V_\lambda$.
We then introduce the normalized operator
\[A_i^V(u) := \lambda_i^A(u)^{-1}A_i(u)\big|_V \in\End(V)\db{u^{-1}},\]
which acts by the identity on the highest weight space.
Then by \cite[Thm. 4.4]{gautam_poles_2023} (see also \cite[Cor. 4.7]{gautam_poles_2023} together with \cite[Prop. 5.7, 5.8]{hernandez_shifted_2022}), there is a unique monic polynomial $T_i(u)\in\End(V)[u]$ that satisfies
\begin{equation}\label{eqn:transfer-op}
    T_i(u+\hbar d_i) = A_i^V(u)T_i(u).
\end{equation}
We note that $T_i(u)$ can be recovered as $\lambda_i^T(u)^{-1}\mathscr{T}_i(u)$, where $\mathscr{T}_i(u)$ is the \emph{$i$th abelianized transfer operator} introduced in \cite[\S 4.3]{gautam_poles_2023} and $\lambda_i^T(u)$ is the eigenvalue of $\mathscr{T}_i(u)$ on the highest weight space; see \cite[Remark 5.1]{friesen_braid_2024} for details.
The eigenvalues of the operator $T_i(u)$ are called the \emph{(specialized) Baxter polynomials} associated to $V$.

The generating series $\xi_i(u)$ and $x^\pm_i(u)$ of $\Yhg$ introduced in Section \ref{ssec:Y-Hopf} operate on $V$ as the expansions at infinity of $\End(V)$-valued rational functions of $u$; see \cite[Prop. 3.6]{gautam_yangians_2016}.
We define the \emph{$i$th set of poles} of $V$ to be the joint set of poles of these operators:
\[\sigma_i(V) := \{\text{Poles of } \xi_i(u)\big|_V,\ x^\pm_i(u)\big|_V \in\End(V)(u)\} \subset\C.\]
The Baxter polynomials are related to the poles of $V$ in the following way: let $\zeros(P(u))$ denote the zeros of any polynomial $P(u)$, let $\mathcal{Z}_i(V)$ denote the zeros of all eigenvalues of $T_i(u)$, and let $\baxterV(u)$ denote the eigenvalue of $T_i(u)$ on the lowest weight space $V_{w_0(\lambda)}$ where $w_0\in\Wg$ is the longest element.
Then by \cite[Thm. 4.4]{gautam_poles_2023}, for all $i\in\I$ we have
\[\sigma_i(V) = \mathcal{Z}_i(V) = \zeros(\baxterV(u)).\]

In the case where $V$ is irreducible, the polynomials $\baxterV(u)$ (and hence the poles of $V$) were computed explicitly in \cite[Thm. 5.2]{gautam_poles_2023}: for all $i\in\I$,
\[\baxterV(u) = \prod_{j\in\I}\prod_{b=d_i}^{2\kappa-d_i} P_j\left(u-(b-d_j)\frac{\hbar}{2}\right)^{v_{ij}^{(b)}}\]
where $\uP=(P_j(u))_{j\in\I}$ is the tuple of Drinfeld polynomials associated to $V\cong L(\uP)$, $\kappa$ is $1/4$ times the eigenvalue of the Casimir element $C\in\Ug$ on the adjoint representation of $\g$ as in Example \ref{E:dual-braid-sl3}, and the integers $v_{ij}^{(b)}$ are obtained from the \emph{quantum Cartan matrix} $(v_{ij}(z))_{i,j\in\I}$; see \cite[\S 5.2]{friesen_braid_2024} for details.


\section{Baxter polynomials associated to extremal weights}

Now let $V$ be irreducible, so $V=L(\uP)$ for some Drinfeld polynomials $\uP=(P_i(u))_{i\in\I}$, and the highest weight of $V$ as a representation of $\g$ is $\lambda = \sum_{i\in\I}\deg(P_i)\omega_i$.
For any $w\in\Wg$, let $\baxterP(u)\in\C[u]$ denote the eigenvalue of $T_i(u)$ on the extremal weight space $V_{w(\lambda)}$.
In particular, if $w$ is the longest element $w_0$, then $\baxterP(u) = \baxterV(u)$.

The following theorem is the main result of this chapter.
It gives a factorization of the Baxter polynomial $\baxterP(u)$ in terms of the braid group action defined in the previous chapter on the Drinfeld polynomials.

\begin{theorem}\label{T:baxter}
    Let $w = s_{j_1}s_{j_2}\cdots s_{j_p}$ be a reduced expression for $w\in\Wg$.
    For each $1\leq r\leq p$, set $w_r := s_{j_{r+1}}\cdots s_{j_p}$, where $w_p = {\id}$.
    Then
    \[\baxterP(u) = \prod_{r:j_r=i} \braid_{w_r}(\uP)_i.\]
    Moreover, $\braid_{w_r}(\uP)_{j_r}$ is a monic polynomial in $u$ for each $r$.
\end{theorem}
\begin{proof}
    First, since $s_{j_p}\cdots s_{j_{r+1}}s_{j_r}$ is a reduced expression, it follows that $w_r^{-1}(\alpha_{j_r}) = s_{j_p}\cdots s_{j_{r+1}}(\alpha_{j_r}) \in\Phi^+$.
    Then by Corollary \ref{C:Tan-monic}, $\braid_{w_r}(\uP)_{j_r}$ is a monic polynomial in $u$ for each $r$.

    By the defining equation \ref{eqn:transfer-op} of $T_i(u)$, the eigenvalue of the normalized operator $A_i^V(u) = \lambda_i^A(u)^{-1}A_i(u)$ on the extremal weight space $V_{w(\lambda)}$ is given by
    \[\frac{\baxterP(u+\hbar d_i)}{\baxterP(u)}\]
    hence the eigenvalue of $A_i(u)$ on $V_{w(\lambda)}$ is $\lambda_i^A(u)$ times the above.
    On the other hand, by Proposition \ref{P:extremal-weight} the eigenvalue of $A_i(u)$ on $V_{w(\lambda)}$ is $\braid_w(\ul)(A_i(u))$, where $\ul$ is the highest weight of $V$.
    Then by the uniqueness assertion of Lemma \ref{L:diff-eqn}, it suffices to prove that
    \[\braid_w(\ul)(A_i(u))
    = \lambda_i^A(u)\prod_{r:j_r=i}\braid_{w_r}(\ul)_i
    = \lambda_i^A(u)\prod_{r:j_r=i}\frac{\q^{2d_i}\braid_{w_r}(\uP)_i}{\braid_{w_r}(\uP)_i},\]
    where the second equality follows from the fact that the action of $\braid_{w_r}$ is an automorphism and $\ul = (\q^{2D}\uP)\uP^{-1}$.
    Using the definition of the dual braid group action from the beginning of the previous chapter, the first equality is equivalent to
    \[\mbraid_{w^{-1}}(A_i(u)) = A_i(u)\prod_{r:j_r=i}\mbraid_{w_r^{-1}}(\xi_i(u)).\]
    To prove this, we will use induction on the length $p$ of $w$: if $p = 1$ then this equation reduces to the identity $\mbraid_j(A_i(u)) = A_i(u)\xi_i(u)^{\delta_{ij}}$ that we established in Proposition \ref{P:tau-a}.
    Now suppose that this equation holds for $w$ of length $p$, and consider the element $w' = ws_{j_{p+1}}$ of length $p+1$.
    Then again using the identity of Proposition \ref{P:tau-a}, we have
    \begin{align*}
        \mbraid_{(w')^{-1}}(A_i(u)) &= \mbraid_{j_{p+1}}(\mbraid_w(A_i(u))) \\
        &= A_i(u)\xi_i(u)^{\delta_{i,j_{p+1}}}\prod_{\substack{1\leq r\leq p \\ j_r=i}}\mbraid_{j_{p+1}}(\mbraid_{w_r^{-1}}(\xi_i(u))) \\
        &= A_i(u)\xi_i(u)^{\delta_{i,j_{p+1}}}\prod_{\substack{1\leq r\leq p \\ j_r=i}}\mbraid_{(w'_r)^{-1}}(\xi_i(u)) \\
        &= A_i(u)\prod_{\substack{1\leq r\leq p+1 \\ j_r=i}}\mbraid_{(w'_r)^{-1}}(\xi_i(u))
    \end{align*}
    which completes the proof.
\end{proof}


\section{Cyclicity criteria for tensor products}

An important property of the poles of representations of $\Yhg$ is that they encode information about when the tensor product of two irreducible representations is cyclic or irreducible.
By \cite[Thm. 7.2]{gautam_poles_2023}, the representation $L(\uP)\otimes L(\uQ)$ is cyclic if for all $i\in\I$, none of the zeros of $Q_i(u+\hbar d_i)$ are poles of $L(\uP)$, i.e., if
\[\zeros(Q_i(u+\hbar d_i)) \subset \C\setminus\sigma_i(L(\uP)).\]
By \cite[Cor. 7.3]{gautam_poles_2023}, $L(\uP)\otimes L(\uQ)$ is irreducible if in addition to the above condition, none of the zeros of $P_i(u+\hbar d_i)$ are poles of $L(\uQ)$:
\[\zeros(P_i(u+\hbar d_i)) \subset \C\setminus\sigma_i(L(\uQ)).\]

Another sufficient condition for the cyclicity of a tensor product of irreducible representations was given in \cite[Thm. 4.8]{tan_braid_2015} using the action of the braid group on $(\C(u)^\times)^\I$, following \cite{chari_braid_2002}.
Let $w_0 = s_{j_1}s_{j_2}\cdots s_{j_p}$ be a reduced expression for the longest element $w_0\in\Wg$, and set $w_r := s_{j_{r+1}}\cdots s_{j_p}$ for each $1\leq r\leq p$ as in the previous section.
Then by \cite[Thm. 4.8]{tan_braid_2015} and the results of Section \ref{sec:extending}, $L(\uP)\otimes L(\uQ)$ is cyclic if for all $1\leq r\leq p$, the polynomials $Q_{j_r}(u+\hbar d_{j_r})$ and $\braid_{w_r}(\uP)_{j_r}$ have no roots in common:
\[\zeros(Q_{j_r}(u+\hbar d_{j_r})) \subset \C\setminus\zeros(\braid_{w_r}(\uP)_{j_r}).\]

It was conjectured in \cite[\S 7.5]{gautam_poles_2023} that the two sufficient conditions for cyclicity above are identical.
To prove this, recall from Section \ref{sec:baxter-poles} that the poles of a finite-dimensional highest-weight representation are exactly the zeros of the Baxter polynomial associated to the lowest weight.
Then using the factorization given in Theorem \ref{T:baxter}, we obtain the following corollary.

\begin{corollary}\label{C:baxter-poles}
    Let $\uP$ be a tuple of Drinfeld polynomials.
    For all $i\in\I$, the $i$th set of poles of $L(\uP)$ is given by
    \[\sigma_i(L(\uP)) = \bigcup_{r:j_r=i}\zeros(\braid_{w_r}(\uP)_i).\]
\end{corollary}

Combining this corollary with the cyclicity criteria of \cite{gautam_poles_2023} and \cite{tan_braid_2015} above, we see that they are identical.
The following corollary summarizes this result.

\begin{corollary}\label{C:cyclicity}
    Let $\uP = (P_i(u))_{i\in\I}$ and $\uQ = (Q_i(u))_{i\in\I}$ be tuples of Drinfeld polynomials.
    The following conditions are equivalent:
    \begin{enumerate}
        \item For all $i\in\I$, $\zeros(Q_i(u+\hbar d_i))\subset\C\setminus\sigma_i(L(\uP))$.
        \item For all $1\leq r\leq p$, $\zeros(Q_{j_r}(u+\hbar d_{j_r}))\subset\C\setminus\zeros(\braid_{w_r}(\uP)_{j_r})$.
    \end{enumerate}
    If either of these conditions hold, then the representation $L(\uP)\otimes L(\uQ)$ is cyclic.
\end{corollary}


% \nocite{*}
\uofsbibliography[alpha]{refs}

% \uofsappendix
% \begin{appendices}
%     \input{chapters/appendix}
% \end{appendices}

\end{document}
