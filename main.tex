\documentclass{uofsthesis-cs}
% main packages
\usepackage{amsmath,amssymb,amsthm}
\usepackage{mathtools}
\usepackage{hyperref}
\usepackage[shortlabels]{enumitem}

% fonts
\usepackage{mathrsfs}
\usepackage[scr=boondoxo]{mathalpha}
\usepackage{upgreek}

% theorem environments
\theoremstyle{theorem}
\newtheorem{theorem}{Theorem}[section]
\newtheorem{corollary}[theorem]{Corollary}
\newtheorem{lemma}[theorem]{Lemma}
\newtheorem{proposition}[theorem]{Proposition}

\theoremstyle{definition}
\newtheorem{definition}[theorem]{Definition}
\newtheorem{example}[theorem]{Example}
\newtheorem{remark}[theorem]{Remark}

% numbers
\newcommand{\C}{\mathbb{C}}
\newcommand{\N}{\mathbb{N}}
\newcommand{\Q}{\mathbb{Q}}
\newcommand{\R}{\mathbb{R}}
\newcommand{\Z}{\mathbb{Z}}

% Lie algebras
\newcommand{\g}{\mathfrak{g}}
\newcommand{\h}{\mathfrak{h}}
\newcommand{\n}{\mathfrak{n}}
\newcommand{\mfgl}{\mathfrak{gl}}
\newcommand{\mfsl}{\mathfrak{sl}}
\newcommand{\Ug}{U(\g)}
\newcommand{\I}{\mathbf{I}}

% braid groups and Yangians
\newcommand{\UqLg}[1][]{U_q^{#1}(L\g)}
\newcommand{\Yhg}[1][]{Y_\hbar^{#1}(\g)}
\newcommand{\ul}{\underline{\lambda}}
\newcommand{\Bg}{\mathscr{B}_\g}
\newcommand{\Wg}{\mathscr{W}_\g}
\newcommand{\braid}{\uptau}
\newcommand{\mbraid}{\mathsf{T}}
\newcommand{\rmat}{\mathcal{R}}

% arrows
\newcommand{\inj}{\hookrightarrow}
\newcommand{\surj}{\twoheadrightarrow}
\newcommand{\iso}{\xrightarrow{\sim}}

% operators
\DeclareMathOperator{\ad}{ad}
\DeclareMathOperator{\End}{End}
\DeclareMathOperator{\id}{id}
\DeclareMathOperator{\gr}{gr}

% delimiters
\DeclarePairedDelimiter{\abs}{\lvert}{\rvert}
\DeclarePairedDelimiter{\db}{[\![}{]\!]}
\DeclarePairedDelimiter{\dparen}{(\!(}{)\!)}


\title{Braid groups and Baxter polynomials}
\author{Noah Friesen}
\degree{\MSc}
\defencedate{July 2024}
\department{Mathematics and Statistics}

\ptuaddress{Head of the Department of Mathematics and Statistics\\
University of Saskatchewan\\
142 McLean Hall, 106 Wiggins Road\\
Saskatoon, Saskatchewan S7N 5E6 Canada
}

\abstract{
It is well known that the braid group $\Bg$ of a simple Lie algebra $\g$ acts on integrable representations of $g$ via products of exponentials of its Chevalley generators.
In particular, the Yangian $\Yhg$ is an integrable representation of $\g$, so there is an action of $\Bg$ on this space.
We show that modifying this action induces an action of $\Bg$ on a certain commutative subalgebra of $\Yhg$ by Hopf algebra automorphisms.
By dualizing this modified action, we recover an action of $\Bg$ on tuples of rational functions defined in the works of Y. Tan and V. Chari.
Using this dual action, we prove a conjecture of S. Gautam and C. Wendlandt that the two sufficient conditions for the tensor product of finite-dimensional irreducible representations of $\Yhg$ to be cyclic are identical.
One of these conditions involves the aforementioned action of $\Bg$ on rational functions, and the other involves roots of the Baxter polynomials, which have many interesting properties and ties to mathematical physics.
}

\acknowledgements{
I would like to thank my supervisors, Prof. Alex Weekes and Prof. Curtis Wendlandt, for their invaluable guidance over the past two years.
I am extremely grateful for their support and this thesis would not have been possible without their direction and feedback.
I would also like to thank Prof. Steven Rayan for his exceptional instruction during my coursework.

I would like to acknowledge the financial support I have received from the Natural Sciences and Engineering Research Council of Canada (NSERC) through the Canada Graduate Scholarships (CGS M) program.

Finally, I thank my family for their love and support.
}


\begin{document}

\maketitle
\frontmatter

\chapter{Introduction}

\section{Braid groups and Yangians}

Given a simple Lie algebra $\g$ over $\C$, we may consider its associated braid group $\Bg$.
The generators of this group satisfy the same braid relations as those of the Weyl group $\Wg$, but the generators of $\Bg$ are all of infinite order.
When $\g = \mfsl_n$, the braid group $\Bg$ coincides with Artin's braid group on $n$ strands, and one can visualize elements of this group quite well using diagrams of physical braids.
In this case, imposing the additional relation that the generators of the group have order $2$ gives us the symmetric group $S_n$, which is the Weyl group of $\mfsl_n$.

Much like the Weyl group, the braid group plays an important role in the representation theory of $\g$.
More precisely, it is well known that the generator $\braid_i$ of $\Bg$ acts on integrable representations $V$ of $\g$ as
\[\exp(e_i)\exp(-f_i)\exp(e_i)\]
where $e_i$ and $f_i$ are the usual Chevalley generators of $\g$ and $i\in\I$ indexes the Cartan matrix (or equivalently, the nodes of the Dynkin diagram) of $\g$.
The integrability of $V$ is needed so that $e_i$ and $f_i$ act nilpotently on every vector of $V$, so that the exponentials above truncate after finitely many terms.
An important property of this braid group action is that it sends a vector in $V$ of weight $\lambda$ to one of weight $w(\lambda)$ for some $w$ in the Weyl group.
Moreover, this action actually preserves the dimension of the weight spaces, and so all of the weight spaces with weight of the form $w(\lambda)$ are isomorphic.
We make use of this property in Chapter \ref{chap:weights} to prove some key lemmas.

For our purposes, the most important integrable representation of $\g$ is an infinite-dimensional Hopf algebra called the Yangian $\Yhg$.
This algebra is a type of affine quantum group, and is a deformation of the universal enveloping algebra of the current algebra $\g[t]$.
Yangians originally arose in the theory of integrable systems, and their representation theory has been studied by many since the seminal paper \cite{drinfeld_hopf_1985} of the Fields medalist V. Drinfeld.
They also have many applications to the areas of geometry and mathematical physics, and their name honours physicist C. N. Yang.

In Chapter \ref{chap:yangians}, after recalling some basic structure and representation theory of $\g$, we give a presentation for the Yangian known as Drinfeld's new presentation.
The generators and relations therein are very reminiscent of the Chevalley--Serre presentation of $\g$ itself.
Indeed, using this presentation it is easy to see that there is a natural inclusion of $\g$ and its enveloping algebra into $\Yhg$ given by sending the generators of the former to the degree-zero generators of the latter.
Because of this inclusion, we get an adjoint action of $\g$ on $\Yhg$ which is integrable, and so there is an action of the braid group $\Bg$ on $\Yhg$.
This action has appeared in a number of contexts; see for example \cite{guay_coproduct_2018, kodera_braid_2019, weekes_highest_2016}.
Also note that because of this, we may regard any representation of $\Yhg$ as a representation of $\g$.

In a later section of Chapter \ref{chap:yangians}, we introduce an alternative set of generating series for the Yangian given by Gerasimov et al. in \cite{gerasimov_class_2005}.
These generating series play a crucial role in our results, as they allow us to compute explicit formulas for the action of the braid group $\Bg$ on the Yangian $\Yhg$.
We are then able to compute formulas for our modified braid group action described in the section below, and ultimately use these formulas to find a new factorization of the so-called Baxter polynomials which gives us our main result.


\section{Modified braid group action}

One of the main motivations for this research was to develop a framework for better understanding an action of the braid group $\Bg$ on tuples of rational functions defined in the work \cite{tan_braid_2015} of Y. Tan.
Drawing inspiration from the results of V. Chari for quantum loop algebras given in \cite{chari_braid_2002}, Tan defined operators directly and proved that they indeed give an action of $\Bg$.
In Chapters \ref{chap:braidgroup} and \ref{chap:weights}, we recover Tan's action starting with the action of $\Bg$ on the Yangian $\Yhg$ described in the previous section, and in doing so prove that Tan's action computes the eigenvalues of certain generators of $\Yhg$ on the \emph{extremal weight spaces} of a representation of $\Yhg$ --- weight spaces that lie in the Weyl group orbit of the highest weight space.

We now describe the key results of Chapter \ref{chap:braidgroup}.
Much like the Lie algebra $\g$, the Yangian $\Yhg$ has a triangular decomposition: one can partition the generators into raising, lowering, and Cartan operators and consider the spaces generated by each of these.
We denote by $\Yhg[0]$ the commutative subalgebra of the Yangian generated by the Cartan generators $\xi_{i,r}$ for $i\in\I$ and $r\in\N$.
Note that this subalgebra deforms the universal enveloping algebra of the current algebra $\h[t]$ where $\h$ is a Cartan subalgebra of $\g$.
We may package the generators of $\Yhg[0]$ into formal series $\xi_i(u)$ in $u^{-1}$ so that the generator $\xi_{i,r}$ is the coefficient of $u^{-r-1}$; there is a unique Hopf algebra structure on $\Yhg[0]$ such that each of these series is grouplike.
Using properties of the triangular decomposition of $\Yhg$, we show that there is a natural linear projection
\[\Pi:\Yhg\to\Yhg[0]\]
and we use this projection to introduce what we call the \emph{modified braid group operators}:
\[\mbraid_i := \Pi\circ\braid_i\big|_{\Yhg[0]}.\]
By the weight-permuting property of the braid group action on representations of $\g$ described in the previous section, the braid group action maps elements of $\Yhg[0]$ to elements of $\Yhg[0]$ which are sums of monomials containing the same number of raising and lowering operators.
Then applying the projection $\Pi$ discards terms that are not in $\Yhg[0]$, so the above operators are endomorphisms of $\Yhg[0]$.
In fact, they have many remarkable properties which we summarize in the below theorem.
This theorem constitutes the first main result of the thesis.

\begin{theorem}\label{T:intro-main1}
    The modified braid group operators $\mbraid_i$ have the following properties:
    \begin{enumerate}
        \item They are Hopf algebra automorphisms of $\Yhg[0]$ that satisfy the braid relations, i.e., they define an action of $\Bg$ on $\Yhg[0]$.
        \item They are uniquely determined by the following formulas, for each $j\in\I$:
        \[\mbraid_i(\xi_j(u)) = \xi_j(u)\prod_{k=0}^{\abs{a_{ij}}-1} \xi_i\left(u-\frac{\hbar d_i}{2}(\abs{a_{ij}}-2k)\right)^{(-1)^{\delta_{ij}}}\]
        where $a_{ij}$ and $d_i$ are the entries of the Cartan matrix and the symmetrizing integers of $\g$, respectively.
        \item The diagonal factor $\rmat^0(z)\in \Yhg[0]^{\otimes 2}\db{z^{-1}}$ of the universal $R$-matrix of $\Yhg$ is $\Bg$-invariant:
        \[(\mbraid_i\otimes\mbraid_i)(\rmat^0(z)) = \rmat^0(z).\]
    \end{enumerate}
\end{theorem}

Let us now turn to the results of Chapter \ref{chap:weights}, in which we recover Tan's braid group action.
Using the modified braid group operators above, we define an action of $\Bg$ on the linear dual of $\Yhg[0]$.
The group of algebra homomorphisms $\Yhg[0]\to\C$ is a subrepresentation, and this group is isomorphic to the group $(1+u^{-1}\C\db{u^{-1}})^\I$ of $\I$-tuples of formal series in $u^{-1}$ with constant coefficient $1$.
To see this, notice that we can define an algebra homomorphism $\Yhg[0]\to\C$ by choosing a complex number for each generator $\xi_{i,r}$ to be mapped to; applying this homomorphism to the generating series $\xi_i(u)$ gives the desired isomorphism.
The formula for this dual action of $\Bg$ on a tuple of series $(\lambda_i(u))_{i\in\I}$ is the same as the formula in the second part of the theorem above, replacing $\xi$ with $\lambda$.
It is also the same formula as the one for the action of $\Bg$ on the group $(\C(u)^\times)^\I$ of $\I$-tuples of rational functions defined by Tan.

In Section \ref{sec:extending}, we reconcile the underlying representation spaces of our dual braid group action and Tan's.
We do this by extending our dual action to the group $M^\I$ of $\I$-tuples of monic Laurent series using the theory of formal additive difference equations.
We then further extend to the group $(\C\dparen{u^{-1}}^\times)^\I$ of $\I$-tuples of arbitrary Laurent series by taking the product of $M^\I$ and the group $(\C^\times)^\I$ on which there is a natural action of the Weyl group $\Wg$.
The group of rational functions on which Tan's action is defined is then recovered as a subrepresentation of this largest space.

In Section \ref{sec:extremal-weights}, we show that the dual braid group action determines the weights of the \emph{extremal vectors} of a representation of the Yangian.
Finite-dimensional irreducible representations of $\Yhg$ are classified in a way very similar to those of $\g$ in that they are determined by highest weights: for such a representation $V$ there is a highest weight vector $v\in V$ that is killed by the raising operators and is an eigenvector for the Cartan operators, say $\xi_{i,r}\cdot v = \lambda_{i,r}v$ for some $\lambda_{i,r}\in\C$.
Packaging these eigenvalues into series as we did for the generators themselves and collecting them into a tuple, we let $\ul = (\lambda_i(u))_{i\in\I}$ denote the highest weight of $V$.
Recall that we can regard $V$ as a representation of $\g$, and doing so we define the extremal vectors of $V$ to be those of $\g$-weight $w(\lambda)$ for $w\in\Wg$, where $\lambda$ is the $\g$-weight of the highest weight vector $v$.
Note that because of the aforementioned dimension-preserving property of the braid group action on weight spaces, each extremal weight space is one-dimensional.
The main result of this section tells us that the eigenvalue of the generating series $\xi_i(u)$ (i.e., the $\Yhg$-weight) on the extremal vector of $\g$-weight $w(\lambda)$ is given by the dual action on the $\Yhg$-highest weight. This eigenvalue is $\braid_w(\ul)_i$, where $\braid_w\in\Bg$ is defined by taking any reduced expression for $w\in\Wg$ and replacing each Weyl group generator with its corresponding braid group generator, and the subscript $i$ denotes taking the $i$th component of the tuple.


\section{Baxter polynomials and cyclicity}

The other main motivation for studying these braid group actions arises from the following fact: the tensor product of two finite-dimensional irreducible representations of the Yangian $\Yhg$ will almost always be cyclic, i.e., highest weight.
This has been studied extensively and has applications in many areas; see for example \cite{chari_yangians_1996, guay_local_2015, molev_yangians_2007, nazarov_irreducibility_2002, akasaka_finite_1997}.
In the literature, there are two sufficient conditions for this cyclicity property.
The first of these was established by Tan in \cite{tan_braid_2015} using a construction that appeared earlier for the quantum loop algebra in \cite{chari_braid_2002}, and involves the braid group action on rational functions described in the previous section.
The second condition was established by S. Gautam and C. Wendlandt in \cite{gautam_poles_2023} and involves the so-called \emph{Baxter polynomials}.
It was conjectured in the latter paper that these two conditions are identical, and we prove in Chapter \ref{chap:baxter} by giving a new factorization of the Baxter polynomials that this is indeed the case.

Let us describe the first cyclicity condition of \cite{tan_braid_2015}.
Finite-dimensional irreducible representations of $\Yhg$ are parametrized by $\I$-tuples of monic polynomials $\uP=(P_i(u))_{i\in\I}$ called \emph{Drinfeld polynomials} which encode the highest weight.
We denote by $L(\uP)$ the representation of $\Yhg$ with Drinfeld polynomials $\uP$.
Choose any reduced expression $w_0 = s_{j_1}\cdots s_{j_p}$ for the longest element of the Weyl group $\Wg$, and let $w_r\in\Wg$ be the element of the Weyl group obtained by removing $r$ generators from the left of this reduced expression.
Then the representation $L(\uP)\otimes L(\uQ)$ is cyclic if
\[\zeros(Q_{j_r}(u+\hbar d_{j_r}))\subset\C\setminus\zeros(\braid_{w_r}(\uP)_{j_r})\]
for each $r$, where $\zeros(p(u))$ denotes the set of zeros of a polynomial $p(u)$.
I.e., the tensor product is cyclic as long as certain Drinfeld polynomials of the right tensor leg and certain polynomials arising from the braid group action on the Drinfeld polynomials of the left tensor leg do not have zeros in common.

Now we will describe the second cyclicity condition of \cite{gautam_poles_2023}.
Similar to how we defined the Cartan generating series $\xi_i(u)$ of $\Yhg$ in the previous section, we may do the same for the raising operators $x^+_i(u)$ and the lowering operators $x^-_i(u)$.
Each of these series acts on a representation $L(\uP)$ as the expansion at $u=\infty$ of some $\End(L(\uP))$-valued rational function in $u$.
If we denote the set of poles of these three rational functions by $\sigma_i(L(\uP))$, then the representation $L(\uP)\otimes L(\uQ)$ is cyclic if
\[\zeros(Q_i(u+\hbar d_i))\subset\C\setminus\sigma_i(L(\uP))\]
for all $i\in\I$.
I.e., the tensor product is cyclic as long as the zeros of the Drinfeld polynomials of the right tensor leg are not poles of the corresponding generating series of the left tensor leg.
Moreover, the tensor product is irreducible if the same condition also holds with $\uP$ and $\uQ$ interchanged.

It was proven in \cite{gautam_poles_2023} that the set of poles $\sigma_i(L(\uP))$ exactly coincides with the set of roots of a distinguished polynomial $\baxter_{i,L(\uP)}^\g(u)$ called the \emph{Baxter polynomial}.
These polynomials first appeared in generality in the work \cite{frenkel_baxters_2015} of E. Frenkel and D. Hernandez, though they have been studied in more specialized contexts for a long period of time.
Their name honours physicist R. J. Baxter, who first encountered polynomials of this type in his paper \cite{baxter_partition_1972}.
The motivation behind Baxter's original work, and much of the literature on Baxter polynomials, is based in mathematical physics, as these polynomials encode information about certain models describing physical phenomena.
For us, $\baxter_{i,L(\uP)}^\g(u)$ arises as the eigenvalue of a certain \emph{abelian transfer operator} on the lowest weight vector of $L(\uP)$.
In \cite{gautam_poles_2023}, a formula for this polynomial was given in terms of the Drinfeld polynomials $\uP$ and a matrix known as the \emph{quantum Cartan matrix} of $\g$.

In Section \ref{sec:baxter-extremal}, we use the results of Chapters \ref{chap:braidgroup} and \ref{chap:weights} to obtain a factorization of the Baxter polynomials in terms of polynomials arising from the action of the braid group on the Drinfeld polynomials.
Because of the discussion above --- namely that the zeros of the Baxter polynomials are exactly the poles of the generating series --- this factorization implies that the two cyclicity criteria for tensor products are identical.
The theorem below summarizes these findings and provides the second main result of the thesis.

\begin{theorem}\label{T:intro-main2}
    The Baxter polynomial $\baxter_{i,L(\uP)}^\g(u)$ admits the following factorization:
    \[\baxter_{i,L(\uP)}^\g(u) = \prod_{r:j_r=i}\braid_{w_r}(\uP)_i.\]
    Consequently, the sufficient conditions for the cyclicity of any tensor product $L(\uP)\otimes L(\uQ)$ obtained in \cite{tan_braid_2015} and \cite{gautam_poles_2023} are identical.
\end{theorem}

Something to note is that the above theorem actually provides a factorization for the Baxter polynomial associated to any extremal weight, not just the lowest weight.
Instead of the longest element $w_0$ in the above construction, we may take a reduced expression for any element $w\in\Wg$ and define the elements $w_r$ in the same way.

Lastly, in Section \ref{sec:baxter-example}, we explicitly compute Baxter polynomials for fundamental representations in the case where $\g=\mfsl_n$.
There are many symmetries which make the computation nicer in this case, and this section provides a good example of how one might make use of the main results of the thesis.

\chapter{Yangians}

\section{Simple Lie algebras}

\subsection{Structure}

To begin, we will review some basic structure and representation theory of a simple Lie algebra, following \cite{humphreys_introduction_1972}.
This will allow us to more easily see the many similarities that arise in the Yangian setting, detailed in the next section.

Let $\g$ be a finite-dimensional simple Lie algebra over $\C$, with Cartan matrix $(a_{ij})_{i,j\in\I}$.
Fix minimal symmetrizing integers $d_i\in\{1,2,3\}$ so that $d_ia_{ij}=d_ja_{ji}$.
Let $\{\alpha_i\}_{i\in\I}$ be a basis of simple roots relative to a Cartan subalgebra $\h\subset\g$, and define a nondegenerate symmetric bilinear form on $\h^{*}$ by $(\alpha_i,\alpha_j)=d_ia_{ij}$.

The following theorem gives a presentation of $\g$ in terms of so-called \emph{Chevalley--Serre} generators and relations.
Note that $\delta_{ij}$ denotes the Kronecker delta function, which is $1$ if $i=j$ and $0$ otherwise, and $\ad:\g\to\End(\g)$ is defined by $\ad(x)(y)=[x,y]$ for $x,y\in\g$.

\begin{theorem}[Serre]\label{T:Serre}
    $\g$ is generated by the elements $\{e_i,f_i,h_i\}_{i\in\I}$ subject to the following relations:
    \begin{enumerate}
        \item $[h_i,h_j] = 0$,
        \item $[e_i,f_j] = \delta_{ij}h_i$,
        \item $[h_i,e_j] = a_{ij}e_j, \quad [h_i,f_j] = -a_{ij}f_j$,
        \item $\ad(e_i)^{1-a_{ij}}(e_j) = \ad(f_i)^{1-a_{ij}}(f_j) = 0 \quad\text{if } i\neq j$.
    \end{enumerate}
\end{theorem}

This presentation is especially useful for studying the representation theory of $\g$.
Before discussing representations, we recall the \emph{universal enveloping algebra} $\Ug$, which is the associative algebra generated by elements of $\g$ in which the commutator corresponds to the Lie bracket of $\g$.
More precisely, $\Ug$ is the quotient of the tensor algebra
\[T(\g) := \C \oplus \g \oplus (\g\otimes\g) \oplus \cdots\]
by the two-sided ideal generated by the elements
\[x\otimes y - y\otimes x - [x,y]\]
for all $x,y\in\g$.
The following theorem gives us a description of $\Ug$ in terms of $\g$ itself, which has many useful consequences.

\begin{theorem}[Poincar\'e--Birkhoff--Witt]\label{T:PBW-g}
    If $x_1,\cdots,x_n$ is an ordered basis for $\g$, then $\{x_1^{k_1}\cdots x_n^{k_n} : k_i\in\N\}$ is a basis for $\Ug$.
\end{theorem}

Recall that $\g$ has a \emph{triangular decomposition}:
\[\g \cong \n^-\oplus\h\oplus\n^+,\]
where $\n^-$ and $\n^+$ are the subalgebras of $\g$ generated by the elements $f_i$ and $e_i$ for all $i\in\I$, respectively.
A consequence of the above theorem is that $\Ug$ also has a triangular decomposition: there is a vector space isomorphism
\[\Ug \cong U(\n^-)\otimes U(\h) \otimes U(\n^+).\]


\subsection{Representation theory}

A \emph{representation} of $\g$ is a vector space $V$ together with a linear map $\phi:\g\to\End(V)$ satisfying
\[\phi([x,y]) = \phi(x)\phi(y)-\phi(y)\phi(x)\]
for all $x,y\in\g$.
In other words, the map $\phi$ is a Lie algebra homomorphism from $\g$ to $\mfgl(V)$, where the latter is the endomorphism space $\End(V)$ with Lie bracket given by the commutator of endomorphisms.
Equivalently, a representation of $\g$ is a vector space $V$ together with a bilinear ``action'' map $\cdot:\g\times V\to V$ satisfying
\[[x,y]\cdot v = x\cdot(y\cdot v) - y\cdot(x\cdot v)\]
for all $x,y\in\g$ and $v\in V$.
From this definition, we see that representations of $\g$ correspond to modules over $\Ug$ and vice versa:
\[(xy)\cdot v = x\cdot (y\cdot v).\]

We would like to classify the finite-dimensional irreducible representations of $\g$.
In order to do so, we first consider a class of representations known as \emph{highest-weight} representations.

\begin{definition}\label{D:hw-g}
    $V$ is a \emph{highest-weight representation} with \emph{highest weight} $\lambda\in\h^*$ if there exists $v\in V$ such that for all $i\in\I$, the following hold:
    \begin{enumerate}
        \item $e_i\cdot v = 0$,
        \item $h_i\cdot v = \lambda(h_i)v$,
        \item $\Ug\cdot v = V$.
    \end{enumerate}
\end{definition}

The second condition in the above definition is equivalent to saying that the highest weight vector $v$ belongs to the \emph{weight space} of weight $\lambda$, which is defined as follows:
\[V_\lambda := \{v\in V : h\cdot v = \lambda(h)v \quad \forall h\in\h\}.\]
The sum of weight spaces $V_\mu$ over all $\mu\in\h^*$ is always a direct sum, and is always a subrepresentation of $V$.
In the case where $V$ is finite-dimensional, this subrepresentation is equal to $V$.

Another important thing to note is that the action of $\g$ permutes the weight spaces of $V$: for all $\mu\in\h^*$ and $i\in\I$, the generators $e_i$ and $f_i$ map $V_\mu$ into $V_{\mu+\alpha_i}$ and $V_{\mu-\alpha_i}$, respectively.
Thus we can think of the generators $e_i$ and $f_i$ as ``raising operators'' and ``lowering operators,'' respectively.
From this, along with the PBW theorem (Theorem \ref{T:PBW-g}), it follows that the weights of $V$ (that is, the linear functionals $\mu\in\h^*$ for which $V_\mu$ is nonempty) are all of the form
\[\mu = \lambda - \sum_{i\in\I} k_i\alpha_i\]
for some $k_i\in\N$.
This provides motivation for using the term ``highest weight'' in the above definition.

Next, we recall the definition of the \emph{Verma module} $M(\lambda)$ of highest weight $\lambda\in\h^*$, which is in some sense the ``largest'' highest-weight representation of $\g$ of a given highest weight.
To construct the Verma module, we make $\Ug$ into a highest-weight representation of $\g$ by taking a quotient in order to impose only the conditions required by Definition \ref{D:hw-g}.
Hence $M(\lambda)$ is the quotient of $\Ug$ by the left ideal generated by the elements $e_i$ and $h_i-\lambda(h_i)1$ for all $i\in\I$.
This representation has a unique maximal subrepresentation, and therefore a unique irreducible quotient representation that we will denote $L(\lambda)$.

The following theorem (known as the \emph{theorem of the highest weight}) completely classifies the finite-dimensional irreducible representations of $\g$ up to isomorphism.

\begin{theorem}\label{T:hw-g}
    \begin{enumerate}
        \item Every finite-dimensional irreducible representation of $\g$ is isomorphic to $L(\lambda)$ for some $\lambda\in\h^*$.
        \item $L(\lambda)$ is finite-dimensional if and only if $\lambda(h_i)\in\N$ for all $i\in\I$.
    \end{enumerate}
\end{theorem}

Finally, we give a couple of examples to better illustrate some of the concepts above.

\begin{example}\label{E:sl2-ad}
    Consider the special linear Lie algebra $\mfsl_2$, which has $1\times 1$ Cartan matrix $[2]$.
    By Theorem \ref{T:Serre}, $\mfsl_2$ is generated by $\{e,f,h\}$ subject to the relations
    \[[e,f]=h, \qquad [h,e]=2e, \qquad [h,f]=-2f.\]
    The universal enveloping algebra $U(\mfsl_2)$ is thus the associative algebra generated by the symbols $\{e,f,h\}$ subject to the relations
    \[ef-fe=h, \qquad he-eh=2e, \qquad hf-fh=-2f.\]
    Since $(f,h,e)$ is an ordered basis of $\mfsl_2$, it follows that a basis for $U(\mfsl_2)$ is given by the monomials $f^{k_1}h^{k_2}e^{k_3}$ where each $k_i\in\N$.

    Consider the \emph{adjoint representation} where $V=\mfsl_2$ and $\phi(x)=\ad(x)$ for all $x\in\mfsl_2$, i.e., where the action of $\mfsl_2$ on itself is given by $x\cdot v=[x,v]$ for all $x,v\in\mfsl_2$.
    This is a highest-weight representation where the highest weight $\lambda\in\h^*$ is given by $\lambda(h)=2$:
    \begin{enumerate}
        \item $e\cdot e = 0$ by definition,
        \item $h\cdot e = 2e$ by the Chevalley--Serre relations,
        \item $e$, $f\cdot e = -h$, and $f^2\cdot e = -2f$ span $\mfsl_2$, hence $e$ generates $\mfsl_2$ as a representation.
    \end{enumerate}
    There are three weights of this representation: the linear functionals that send $h$ to $2$, $0$, and $-2$.
    This representation is finite-dimensional and irreducible, and hence isomorphic to the quotient module $L(\lambda)$ of the Verma module $M(\lambda)$ by Theorem \ref{T:hw-g}.
    We also have that $\lambda(h)\in\N$, as we would expect due to the second part of Theorem \ref{T:hw-g}.
\end{example}

Abusing notation, we will use $L(n)$ to denote the unique irreducible representation $L(\lambda)$ of $\mfsl_2$ where $\lambda(h)=n$.
Then the adjoint representation in the example above is $L(2)$.

\begin{example}\label{E:sl2-C2}
    Consider now the \emph{natural representation} of $\mfsl_2$ as the space of $2\times 2$ trace-zero matrices with basis
    \[e = \bmat{0 & 1 \\ 0 & 0}, \qquad f=\bmat{0 & 0 \\ 1 & 0}, \qquad h=\bmat{1 & 0 \\ 0 & -1}.\]
    This gives an action of $\mfsl_2$ on $\C^2$, and it is not hard to see that $v=(1,0)$ is a highest weight vector of weight $1$.
    Further, $f\cdot v = (0,1)$ is a vector of weight $-1$, and these two vectors $v$ and $f\cdot v$ span $\C^2$, so we see that this is the unique irreducible representation $L(1)$.
\end{example}


\section{Yangians}

\subsection{Algebra structure}

We now review the definition of the Yangian associated to a simple Lie algebra $\g$, and some of its properties.

\begin{definition}\label{D:Y}
    Fix some nonzero complex number $\hbar\in\C^\times$.
    The \emph{Yangian} $\Yhg$ is the unital associative algebra over $\C$ with generators $\{\xi_{i,r}, x^\pm_{i,r}\}_{i\in\I,r\in\N}$ subject to the following relations:
    \begin{enumerate}
        \item $[\xi_{i,r},\xi_{j,s}] = 0$,
        \item $[x^+_{i,r},x^-_{j,s}] = \delta_{ij}\xi_{i,r+s}$,
        \item $[\xi_{i,0},x^\pm_{j,s}] = \pm d_ia_{ij}x^\pm_{j,s}$,
        \item $[\xi_{i,r+1},x^\pm_{j,s}]-[\xi_{i,r},x^\pm_{j,s+1}] = \pm\hbar\dfrac{d_ia_{ij}}{2}(\xi_{i,r}x^\pm_{j,s}+x^\pm_{j,s}\xi_{i,r})$,
        \item $[x^\pm_{i,r+1},x^\pm_{j,s}]-[x^\pm_{i,r},x^\pm_{j,s+1}] = \pm\hbar\dfrac{d_ia_{ij}}{2}(x^\pm_{i,r}x^\pm_{j,s}+x^\pm_{j,s}x^\pm_{i,r})$,
        \item $\displaystyle\sum\limits_{\pi\in S_m}[x^\pm_{i,r_{\pi(1)}}, [x^\pm_{i,r_{\pi(2)}}, [\cdots[x^\pm_{i,r_{\pi(m)}}, x^\pm_{j,s}]\cdots]]]=0 \quad\text{if } i\neq j$.
    \end{enumerate}
    where in the final relation $m=1-a_{ij}$ and $S_m$ denotes the symmetric group of degree $m$.
\end{definition}

Comparing these relations to the Chevalley--Serre relations of Theorem \ref{T:Serre}, we see that the first three are very similar.
Indeed, the generators $\xi_{i,r}$ behave similarly to the Cartan generators $h_i$ of $\g$, and the generators $x^+_{i,r}$ and $x^-_{i,r}$ can be though of as raising and lowering operators like the generators $e_i$ and $f_i$ of $\g$, respectively.
There is an analogue of the PBW theorem (Theorem \ref{T:PBW-g}) for the Yangian, which states that monomials in these generators (in any order) give a basis for $\Yhg$.
One consequence of this theorem is that the Yangian has a triangular decomposition: there is a vector space isomorphism
\[\Yhg \cong \Yhg[-]\otimes\Yhg[0]\otimes\Yhg[+]\]
where $\Yhg[\pm]$ and $\Yhg[0]$ are the subalgebras generated by the elements $x^\pm_{i,r}$ and $\xi_{i,r}$ for all $i\in\I$ and $r\in\N$, respectively.
As another consequence, there is an injective algebra homomorphism $\Ug\inj\Yhg$ given by
\[e_i\mapsto d_i^{-1/2}x^+_{i,0}, \qquad f_i\mapsto d_i^{-1/2}x^-_{i,0}, \qquad h_i\mapsto d_i^{-1/2}\xi_{i,0}\]
thus we can view $\Ug$ (and also $\g$ itself) as being a subalgebra of $\Yhg$.

The Yangian also has the structure of a filtered algebra: let the generators $x^\pm_{i,r}$ and $\xi_{i,r}$ have degree $r$ for all $i\in\I$ and $r\in\N$.
This gives us a sequence of vector subspaces
\[\{0\}\subset F_0\subseteq F_1\subseteq\cdots\subseteq \Yhg\]
whose union is $\Yhg$ and that have the property that $F_m\cdot F_n\subseteq F_{m+n}$ for all $m,n\in\N$.
As a generalization of the embedding of $\Ug$ into $\Yhg$, there is an isomorphism of graded algebras
\[U(\g[t]) \cong \gr(\Yhg) := \bigoplus_{r\in\N}F_r/F_{r-1}\]
given by
\[e_it^r\mapsto d_i^{-1/2}\overline{x^+_{i,r}}, \qquad f_it^r\mapsto d_i^{-1/2}\overline{x^-_{i,r}}, \qquad h_it^r\mapsto d_i^{-1/2}\overline{\xi_{i,r}}\]
where for each generator $y\in\{x^\pm_i,\xi_i\}$, $\overline{y_r}$ denotes the image of $y_r$ in the quotient $F_r/F_{r-1}$, and $\g[t] := \g\otimes\C[t]$ is the \emph{current algebra} with Lie bracket given by
\[[x\otimes t^m, y\otimes t^n] = [x,y]\otimes t^{m+n}\]
for all $x,y\in\g$ and $m,n\in\N$.


\subsection{Hopf algebra structure}

The Yangian is not only a unital associative algebra, but also a Hopf algebra.
In order to write down the Hopf algebra structure, we first note that $\Yhg$ is generated by the elements $x^\pm_{i,0}$, $\xi_{i,0}$, and $t_{i,1} := \xi_{i,1}-\frac{\hbar}{2}\xi_{i,0}^2$, for all $i\in\I$.
Indeed, we can recover the generators $x^\pm_{i,r}$ and $\xi_{i,r}$ for all $r\in\N$ inductively using the defining relations of the Yangian.
The counit $\varepsilon:\Yhg\to\C$, coproduct $\Delta:\Yhg\to\Yhg\otimes\Yhg$, and antipode $S:\Yhg\to\Yhg$ are thus determined by the Hopf algebra structure on $\Ug$: for all $y\in\{x^\pm_{i,0},\xi_{i,0} : i\in\I\}$, we have
\begin{gather*}
    \varepsilon(y) = 0, \\
    \Delta(y) = y\otimes 1 + 1\otimes y, \\
    S(y) = -y.
\end{gather*}
For the remaining generators $t_{i,1}$, the coproduct and antipode have an additional term: choose any $x^\pm_\alpha\in\g_{\pm\alpha}$ satisfying $(x^+_\alpha,x^-_\alpha)=1$, then for all $i\in\I$, we have
\begin{gather*}
    \varepsilon(t_{i,1}) = 0, \\
    \Delta(t_{i,1}) = t_{i,1}\otimes 1 + 1\otimes t_{i,1} - \hbar\sum_{\alpha\in\Phi^+}(\alpha_i,\alpha) x^-_\alpha \otimes x^+_\alpha, \\
    S(t_{i,1}) = -t_{i,1} - \hbar\sum_{\alpha\in\Phi^+}(\alpha_i,\alpha) x^-_\alpha x^+_\alpha,
\end{gather*}
where $\Phi^+$ is the set of positive roots of $\g$.

We can package the generators of $\Yhg$ into \emph{generating series} in a formal variable $u^{-1}$ as follows:
\begin{equation}\label{eqn:gen-series}
    \xi_i(u) := 1+\hbar\sum_{r\geq 0}\xi_{i,r}u^{-r-1}, \qquad x^\pm_i(u) := \hbar\sum_{r\geq 0}x^\pm_{i,r}u^{-r-1}
\end{equation}
for all $i\in\I$.
For each $a\in\C$, there is a Hopf algebra automorphism $\tau_a$ of $\Yhg$, called a \emph{shift automorphism}, uniquely determined by
\[\tau_a(y(u)) = y(u-a)\]
for all $y(u)\in\{\xi_i(u),x^\pm_i(u) : i\in\I\}$.
Replacing $a$ with a formal variable $z$ gives us an injective algebra homomorphism $\tau_z:\Yhg\inj\Yhg[][z]$.
This homomorphism was used by Drinfeld in \cite{drinfeld_hopf_1985} to define the \emph{universal $R$-matrix} $\rmat(z)$ of the Yangian, whose properties are captured in the following theorem.

\begin{theorem}\label{T:R}
    There is a unique element $\rmat(z)\in 1+z^{-1}\Yhg^{\otimes 2}\db{z^{-1}}$ satisfying the following identity in $\Yhg^{\otimes 2}\dparen{z^{-1}}$ for all $y\in\Yhg$:
    \[(\tau_z\otimes{\id})\Delta^{\mathrm{op}}(y) = \rmat(z) \cdot (\tau_z\otimes{\id})\Delta(y) \cdot \rmat(z)^{-1}\]
    and the following identities in $\Yhg^{\otimes 3}\db{z^{-1}}$:
    \begin{gather*}
        (\Delta\otimes{\id})(\rmat(z)) = \rmat_{13}(z)\rmat_{23}(z), \\
        ({\id}\otimes\Delta)(\rmat(z)) = \rmat_{13}(z)\rmat_{12}(z).
    \end{gather*}
    Moreover, $\rmat(z)^{-1} = \rmat_{21}(-z)$, and for all $a,b\in\C$:
    \[(\tau_a\otimes\tau_b)\rmat(z) = \rmat(z+a-b).\]
\end{theorem}

A proof of the above theorem was recently given in \cite{gautam_meromorphic_2021}, in which the universal $R$-matrix was reconstructed from its \emph{Gauss decomposition}:
\[\rmat(z) = \rmat^+(z)\rmat^0(z)\rmat^-(z)\]
where $\rmat^+(z) := \rmat^-_{21}(-z)^{-1}$, and $\rmat^-(z)\in(\Yhg[-]\otimes\Yhg[+])\db{z^{-1}}$ is of the form
\[\rmat^-(z) = \sum_{\beta\in Q^+}\rmat_\beta^-(z) \qquad\text{with}\qquad \rmat_\beta^-(z) \in(\Yhg[-]_{-\beta}\otimes\Yhg[+]_\beta)\db{z^{-1}}\]
where $Q^+$ is the positive cone of the root lattice of $\g$.
The components $\rmat_\beta^-(z)$ were constructed recursively in the height of the root $\beta$, and $\rmat_0^-(z)=1$; see \cite[\S 4.2]{gautam_meromorphic_2021}.
The ``diagonal factor'' $\rmat^0(z)$ is defined as the unique series in $1+z^{-1}\Yhg[0]^{\otimes 2}\db{z^{-1}}$ satisfying a certain formal difference equation; see \cite[\S6]{gautam_meromorphic_2021}.


\subsection{Alternate generators}\label{ssec:alt-gen}

Recall the generating series $\xi_i(u)$ and $x^\pm_i(u)$ defined above (\ref{eqn:gen-series}).
In \cite{gerasimov_class_2005}, Gerasimov et al. used these series to define another set of generating series for the Yangian which will prove useful to us in later sections.
In particular, there is a unique tuple of formal series $(A_j(u))_{j\in\I}\in (1+u^{-1}\Yhg[0]\db{u^{-1}})$ that satisfy the following relation for all $i\in\I$:
\[\xi_i(u) = \frac{\prod\limits_{j\neq i}\prod\limits_{r=1}^{-a_{ji}}A_j\bigl(u-\frac{\hbar d_j}{2}(2r-a_{ji})\bigr)}{A_i(u)A_i(u-\hbar d_i)},\]
and the coefficients of these series generate the subalgebra $\Yhg[0]$.

Using the series above, we can now define the following set of series:
\begin{gather*}
    B_i(u) = d_i^{1/2}A_i(u)x^+_i(u), \\
    C_i(u) = d_i^{1/2}x^-_i(u)A_i(u), \\
    D_i(u) = A_i(u)\xi_i(u) + C_i(u)A_i(u)^{-1}B_i(u).
\end{gather*}
The coefficients of $A_i(u)$, $B_i(u)$, and $C_i(u)$ generate $\Yhg$ as an algebra.
The following proposition \cite[Prop. 2.1]{gerasimov_class_2005} contains the commutation relations for these series that we will need.

\begin{proposition}\label{P:GKLO}
    The series $A_i(u)$, $B_i(u)$, $C_i(u)$, and $D_i(u)$ satisfy the following relations:
    \begin{enumerate}
        \item For all $i,j\in\I$, we have
        \[[A_i(u),A_j(v)] = [B_i(u),B_i(v)] = [C_i(u),C_i(v)] = 0.\]
        \item For all $i,j\in\I$ with $i\neq j$, we have
        \[[A_i(u),B_j(v)] = [A_i(u),C_j(v)] = [B_i(u),C_j(v)] = 0.\]
        \item For all $i,j\in\I$, we have
        \begin{align*}
            (u-v)[A_i(u),B_i(v)] &= d_i\hbar(B_i(u)A_i(v) - B_i(v)A_i(u)), \\
            (u-v)[A_i(u),C_i(v)] &= d_i\hbar(A_i(u)C_i(v) - A_i(v)C_i(u)), \\
            (u-v)[A_i(u),D_i(v)] &= d_i\hbar(B_i(u)C_i(v) - B_i(v)C_i(u)), \\
            (u-v)[B_i(u),C_i(v)] &= d_i\hbar(A_i(u)D_i(v) - A_i(v)D_i(u)), \\
            (u-v)[C_i(u),D_i(v)] &= d_i\hbar(D_i(u)C_i(v) - D_i(v)C_i(u)).
        \end{align*}
    \end{enumerate}
\end{proposition}

Now notice that the generator $e_i$ of $\g$, via the embedding of $\Ug$ into $\Yhg$, can be seen as the coefficient of $u^{-1}$ in the series $B_i(u)$.
Similarly, the generator $f_i$ is the coefficient of $u^{-1}$ in $C_i(u)$.
We can take the coefficient of $u^{-1}$ on both sides of the relations in the above proposition in order to obtain commutation relations of these generators with the other generating series.
Thus the above proposition encodes information about the adjoint action of $\g$ on $\Yhg$, which is captured in the following corollary.

\begin{corollary}\label{C:GKLO}
    The series $A_i(u)$, $B_i(u)$, $C_i(u)$, and $D_i(u)$ satisfy the following relations:
    \begin{enumerate}
        \item For all $i,j\in\I$ with $i\neq j$, we have
        \[[e_i,A_j(u)] = [f_i,A_j(u)] = 0.\]
        \item For all $i\in\I$, we have
        \begin{gather*}
            [e_i,A_i(u)] = B_i(u), \qquad [e_i,B_i(u)] = 0, \\
            [e_i,C_i(u)] = D_i(u)-A_i(u), \qquad [e_i,D_i(u)] = -B_i(u).
        \end{gather*}
        \item For all $i\in\I$, we have
        \begin{gather*}
            [f_i,A_i(u)] = -C_i(u), \qquad [f_i,B_i(u)] = A_i(u)-D_i(u), \\
            [f_i,C_i(u)] = 0, \qquad [f_i,D_i(u)] = C_i(u).
        \end{gather*}
    \end{enumerate}
\end{corollary}


\subsection{Representation theory}

We now review some basic representation theory of the Yangian, following \cite{chari_guide_1995}.
A \emph{representation} of $\Yhg$ is a module for $\Yhg$.
Similar to the case of a simple Lie algebra, we can define highest-weight representations for $\Yhg$ as follows.

\begin{definition}\label{D:hw-Y}
    Let $\ul=\{\lambda_{i,r}\in\C\}_{i\in\I,r\in\N}$ be a collection of complex numbers.
    A representation $V$ of $\Yhg$ is a \emph{highest-weight representation} with \emph{highest weight} $\ul$ if there exists $v\in V$ such that for all $i\in\I$ and $r\in\N$, the following hold:
    \begin{enumerate}
        \item $x^+_{i,r}\cdot v = 0$,
        \item $\xi_{i,r}\cdot v = \lambda_{i,r}v$,
        \item $\Yhg\cdot v = V$.
    \end{enumerate}
\end{definition}

We can define the \emph{Verma module} $M(\ul)$ for the Yangian by taking the quotient of $\Yhg$ by the left ideal generated by the elements $x^+_{i,r}$ and $\xi_{i,r}-\lambda_{i,r}1$ for all $i\in\I$ and $r\in\N$.
Again the Verma module has a unique maximal submodule and therefore a unique irreducible quotient module that we will denote $L(\lambda)$.
The following theorem provides an analogue of the theorem of the highest weight (Theorem \ref{T:hw-g}) for the Yangian.

\begin{theorem}\label{T:hw-Y}
    \begin{enumerate}
        \item Every finite-dimensional irreducible representation of $\Yhg$ is isomorphic to $L(\ul)$ for some $\ul$.
        \item $L(\ul)$ is finite-dimensional if and only if there exist a tuple of monic polynomials (called \emph{Drinfeld polynomials}) $P_i(u)\in\C[u]$ such that
        \[\frac{P_i(u+d_i\hbar)}{P_i(u)} = 1+\hbar\sum_{r\geq 0}\lambda_{i,r}u^{-r-1}\]
        for all $i\in\I$, where we take the Laurent expansion of the left-hand side of the above equation about $u=\infty$.
    \end{enumerate}
\end{theorem}

Finally, we give an example to illustrate the above theorem.

\begin{example}\label{E:Y(sl2)}
    Consider $Y_\hbar(\mfsl_2)$.
    This algebra is generated by $\{\xi_r,x^\pm_r\}_{r\in\N}$, where we have dropped the first subscript $i$ since in this case $\I$ is a singleton set.
    Following \cite[\S 12.1]{chari_guide_1995}, we can obtain representations of $Y_\hbar(\mfsl_2)$ by extending the finite-dimensional irreducible representations $L(n)$ of $\mfsl_2$.
    Every such representation has a basis $\{v_0,\dots,v_n\}$ on which the action of $\mfsl_2$ is given by
    \[e\cdot v_m = (n-m+1)v_{m-1}, \qquad f\cdot v_m = (m+1)v_{m+1}, \qquad h\cdot v_m = (n-2m)v_m\]
    for each $m=0,1,\dots,n$, where we set $v_{-1}=v_{n+1}=0$.
    One can prove that for every $a\in\C$, we get a representation $L(n)_a$ of $Y_\hbar(\mfsl_2)$ (called an \emph{evaluation representation}) where the action on the same basis as above is given by
    \begin{align*}
        x^+_r\cdot v_m &= \left(a+\frac{1}{2}n-m+\frac{1}{2}\right)^r(n-m+1)v_{m-1}, \\
        x^-_r\cdot v_m &= \left(a+\frac{1}{2}n-m-\frac{1}{2}\right)^r(m+1)v_{m+1}, \\
        \xi_r\cdot v_m &= \left(\left(a+\frac{1}{2}n-m-\frac{1}{2}\right)^r(n-m)(m+1) -\left(a+\frac{1}{2}n-m+\frac{1}{2}\right)^r(n-m+1)m\right) v_m.
    \end{align*}

    Recall the natural representation $L(1)=\C^2$ of Example \ref{E:sl2-C2}, with basis $v_0=(1,0)$ and $v_1=(0,1)$.
    Setting $n=1$ and $m=0,1$ in the formulas above, we get a representation $L(1)_a$ of $Y_\hbar(\mfsl_2)$ where
    \begin{align*}
        \begin{split}
            x^+_r\cdot v_0 &= 0, \\
            x^-_r\cdot v_0 &= a^rv_1, \\
            \xi_r\cdot v_0 &= a^rv_0,
        \end{split}
        \begin{split}
            x^+_r\cdot v_1 &= a^rv_0, \\
            x^-_r\cdot v_1 &= 0, \\
            \xi_r\cdot v_1 &= -a^rv_1.
        \end{split}
    \end{align*}
    hence the representation map on each Yangian generator is given explicitly by:
    \[x^+_r\mapsto\bmat{0 & a^r \\ 0 & 0}, \qquad x^-_r\mapsto\bmat{0 & 0 \\ a^r & 0}, \qquad \xi_r\mapsto\bmat{a^r & 0 \\ 0 & -a^r}.\]
    and we see that in the case of $r=0$ we recover the action of $\mfsl_2$ on $\C^2$ that we started with.
    Here, $v_0$ is again a highest weight vector and so by the above theorem $L(1)_a$ must be the unique finite-dimensional irreducible representation $L(\ul)$ where $\ul$ is defined by $\lambda_r=a^r$ for all $r\in\N$.
    By the second part of the theorem, this representation has a Drinfeld polynomial since it is finite-dimensional.
    We will show that $P(u)=u-a$ is the polynomial we are looking for: we have
    \[\frac{P(u+\hbar)}{P(u)} = \frac{u+\hbar-a}{u-a} = 1+\hbar\frac{1}{u-a}.\]
    Now to expand $\frac{1}{u-a}$ about $u=\infty$, we replace $u$ with another variable $1/x$ and expand about $x=0$:
    \[\frac{1}{\frac{1}{x}-a} = \frac{x}{1-ax} = x\sum_{r\geq 0}(ax)^r = \sum_{r\geq 0}a^rx^{r+1} = \sum_{r\geq 0}a^ru^{-r-1}.\]
    Thus $P(u)$ satisfies the equation in the theorem.
\end{example}

\chapter{Braid group actions}\label{chap:braidgroup}

\section{Braid group operators}\label{sec:braid}

In this section, we review the action of the braid group of a simple Lie algebra $\g$ on its integrable representations, and prove some important properties.

The \emph{braid group} of $\g$, denoted $\Bg$, is the group with generators $\braid_i$ for $i\in\I$ subject to the following defining relations, called the \emph{braid relations}: for all $i,j\in\I$ with $i\neq j$, we have
\[\underbrace{\braid_i\braid_j\braid_i\cdots}_{m_{ij}\text{ factors}} = \underbrace{\braid_j\braid_i\braid_j\cdots}_{m_{ij}\text{ factors}}\]
where the number of factors $m_{ij}=m_{ji}$ on each side of the above equality is defined according to the entries of the Cartan matrix:
\[\begin{array}{c|cccc}
    a_{ij}a_{ji} & 0 & 1 & 2 & 3 \\
    \hline
    m_{ij} & 2 & 3 & 4 & 6
\end{array}\]

Recall that the Weyl group $\Wg$ associated to $\g$ is generated by the simple reflections $s_i$ for $i\in\I$, which satisfy the same relations as those of the braid group generators as well as the relation $s_i^2={\id}$ for all $i\in\I$.
Thus there is a surjective map $\Bg\surj\Wg$ defined by $\braid_i\mapsto s_i$, with kernel generated by the elements $\braid_i^2$ for all $i\in\I$.
This surjection has a section (i.e., a right inverse) $\Wg\to\Bg$ where $w\mapsto\braid_w$, defined by taking a reduced expression $w=s_{i_1}\cdots s_{i_\ell}$ and setting $\braid_w:=\braid_{i_1}\cdots\braid_{i_\ell}$.
It turns out that this is independent of the choice of reduced expression for $w$, and these elements satisfy the property $\braid_{vw}=\braid_v\braid_w$ whenever the lengths of the reduced expressions for $v$ and $w$ add: $\ell(vw)=\ell(v)+\ell(w)$; see \cite[\S2.1.2]{lusztig_introduction_2010}.

As mentioned above, there is an action of the braid group on integrable representations of $\g$, which are defined below.

\begin{definition}\label{D:integrable}
    A representation $(V,\phi)$ of $\g$ is \emph{integrable} if the following two conditions hold:
    \begin{enumerate}
        \item $V$ is a direct sum of weight spaces $V_\mu$ where $\mu(h_i)\in\Z$ for all $i\in\I$,
        \item $\phi(e_i)$ and $\phi(f_i)$ are \emph{locally nilpotent} endomorphisms of $V$ for all $i\in\I$: for each $v\in V$ there exists some $n\in\N$ such that $\phi(e_i)^n(v)=\phi(f_i)^n(v)=0$.
    \end{enumerate}
\end{definition}

Given such a representation $(V,\phi)$, we can define the following operators for each $i\in\I$:
\[\braid_i^V := \exp(\phi(e_i))\exp(\phi(-f_i))\exp(\phi(e_i)).\]
Since $\phi(e_i)$ and $\phi(-f_i)$ are locally nilpotent endomorphisms of $V$, it follows by the argument in \cite[\S21.2]{humphreys_introduction_1972} that each $\braid_i^V$ is an automorphism of $V$.
One can check that these operators satisfy the braid relations, so the assignment $\braid_i\mapsto\braid_i^V$ gives us an action of the braid group on $V$.
% TODO: prove this, or give an example (maybe adjoint rep of sl3)

Recall that there is an action of the Weyl group on $\h^*$ given by
\[s_i(\alpha_j) = \alpha_j - a_{ij}\alpha_i\]
for all $i,j\in\I$.
The following proposition shows that the braid group action on $V$ intertwines the weight spaces of $V$ via the Weyl group action on the weights:

\begin{proposition}\label{P:tau-wt-space}
    For all $i\in\I$ and $\mu\in\h^*$, we have $\braid_i^V(V_\mu)=V_{s_i(\mu)}$.
\end{proposition}
\begin{proof}
    We use the argument shown in \cite[\S1.3]{kumar_kac-moody_2002}.
    Fix $v\in V_\mu$, then for any $h\in\h$ such that $\alpha_i(h)=0$, we have
    \[h\cdot \braid_i^V(v) = \mu(h)\braid_i^V(v).\]
    By the Serre relations and some properties of nilpotent derivations (namely using the Leibniz rule), we have
    \[(\braid_i^V)^{-1}\phi(h_i)\braid_i^V = -\phi(h_i)\]
    as endomorphisms of $V$.
    From this, it follows that
    \begin{align*}
        h_i\cdot\braid_i^V(v) &= -\braid_i^V(h_i\cdot v) \\
        &= -\braid_i^V(\mu(h_i)v) \\
        &= -\mu(h_i)\braid_i^V(v)
    \end{align*}
    which is sufficient to show that $\braid_i^V(v)$ is a vector of weight $s_i(\mu)$.
    One can similarly show that $(\braid_i^V)^{-1}(v)$ is also a vector of weight $s_i(\mu)$, and since $s_i^2={\id}$, this completes the proof.
\end{proof}

The following proposition shows another important property of these braid group operators.

\begin{proposition}\label{P:tau-tensor}
    If $(V,\phi_V)$ and $(W,\phi_W)$ are both integrable representations of $\g$, then $\braid_i^{V\otimes W} = \braid_i^V\otimes\braid_i^W$ for all $i\in\I$.
\end{proposition}
\begin{proof}
    Let $\phi=\phi_V\otimes\phi_W$, so for all $x\in\g$ we have
    \[\phi(x) = \phi_V(x)\otimes{\id} + {\id}\otimes\phi_W(x).\]
    Notice that the two terms of the right-hand side of the above equation commute, since
    \[(\phi_V(x)\otimes{\id})(\id\otimes\phi_W(x)) = ({\id}\otimes\phi_W(x))(\phi_V(x)\otimes{\id}) = \phi_V(x)\otimes\phi_W(x).\]
    So we can take advantage of the fact that $\exp(A+B)=\exp(A)\exp(B)$ for any two commuting operators $A$ and $B$.
    Also using the fact that $\exp(A\otimes{\id})=\exp(A)\otimes{\id}$ for any operator $A$, we have:
    \begin{align*}
        \exp(\phi(x)) &= \exp(\phi_V(x)\otimes{\id}+{\id}\otimes\phi_W(x)) \\
        &= \exp(\phi_V(x)\otimes{\id})\exp({\id}\otimes\phi_W(x)) \\
        &= (\exp(\phi_V(x))\otimes{\id})({\id}\otimes\exp(\phi_W(x))) \\
        &= \exp(\phi_V(x))\otimes\exp(\phi_W(x)).
    \end{align*}
    Now applying the definition of the braid group operators as products of exponentials of $e_i$ and $-f_i$ completes the proof.
\end{proof}

\begin{corollary}\label{C:tau-alg}
    Let $A$ be an associative algebra such that there is an injective algebra homomorphism $\Ug\inj A$, and let $V$ be a module for $A$.
    If the actions of $\g$ on $A$ and $V$ are both integrable, then
    \[\braid_i^V(a\cdot v) = \braid_i^A(a)\cdot\braid_i^V(v)\]
    for all $a\in A$ and $v\in V$.
\end{corollary}

This corollary is especially useful because the Yangian is such an associative algebra $A$.


\section{Modified braid group operators}\label{sec:mbraid}

In this section, we will define an action of the braid group on the Cartan part $\Yhg[0]$ of the Yangian by modifying the operators defined in the previous section.

Recall the triangular decomposition of the Yangian:
\[\Yhg \cong \Yhg[-]\otimes\Yhg[0]\otimes\Yhg[+],\]
and the fact that the Yangian is a Hopf algebra with counit $\varepsilon$ that maps all generators to zero.
Now define a map $\varepsilon^+:={\id}\otimes{\id}\otimes\varepsilon$.
Choosing a PBW basis for the Yangian that corresponds to the above triangular decomposition, we see that this map sends any basis monomial containing raising operators (i.e., elements of $\Yhg[+]$) to zero, and acts as the identity on the other basis monomials.
As mentioned at the end of the previous section, it is well-known that the adjoint action of $\g$ on $\Yhg$ defines an integrable representation of $\g$.
For simplicity, we will denote by $\braid_i$ the algebra automorphisms $\braid_i^{\Yhg}$ defined in the previous section, for $i\in\I$.

\begin{definition}\label{D:mbraid}
    The \emph{modified braid group operators} are the maps defined as follows, for $i\in\I$:
    \[\mbraid_i = \varepsilon^+ \circ \braid_i\Big|_{\Yhg[0]}.\]
\end{definition}

\begin{lemma}\label{L:mbraid-hom}
    Each $\mbraid_i$ is an algebra endomorphism of $\Yhg[0]$.
\end{lemma}
\begin{proof}
    Let $\Yhg_0$ be the zero weight space of $\Yhg$ as a representation of $\g$.
    Elements of $\Yhg$ belong to this weight space, but so do elements of with the same number of raising and lowering operators such as $x^-_{i,r}\xi_{i,r}x^+_{i,r}$, which means that $\Yhg[0]\subset\Yhg_0$.
    By Proposition \ref{P:tau-wt-space}, this means that $\braid_i$ maps $\Yhg[0]$ into $\Yhg_0$.
    Next, $\varepsilon^+$ maps monomials with operators to zero, so it maps $\Yhg_0$ into $\Yhg[0]$ and hence $\mbraid_i:\Yhg[0]\to\Yhg[0]$.

    For all $a\in\Yhg[0]$, let
    \[\braid_i(a) = a_0+a_+\]
    where $a_0\in\Yhg[0]$ and $a_+\in\Yhg_0\setminus\Yhg[0]$.
    Then for $a,b\in\Yhg[0]$, we have
    \begin{equation}\label{eqn:TaTb}
        \braid_i(a)\braid_i(b) = a_0b_0 + a_0b_+ + a_+b_0 + a_+b_+.
    \end{equation}
    Immediately, we see that applying $\varepsilon^+$ to the above will map the second and fourth terms of the right-hand side to zero, since they have raising operators on the right.
    Using the defining relations of the Yangian will allow us to write the third term of the right-hand side above as an element of $\Yhg[0]\otimes\Yhg[+]$ as well: recall that
    \[[\xi_{i,r+1},x^+_{j,s}]-[\xi_{i,r},x^+_{j,s+1}]-\hbar\frac{d_ia_{ij}}{2}(\xi_{i,r}x^+_{j,s}+x^+_{j,s}\xi_{i,r})\]
    for all $i,j\in\I$ and $r,s\in\N$.
    Expanding, we have
    \[x^+_{j,s}\xi_{i,r+1}=\xi_{i,r+1}x^+_{j,s} -\xi_{i,r}x^+_{j,s+1} -\hbar\frac{d_ia_{ij}}{2}\xi_{i,r}x^+_{j,s} +x^+_{j,s+1}\xi_{i,r} -\hbar\frac{d_ia_{ij}}{2}x^+_{j,s}\xi_{i,r}.\]
    The first three terms of the right-hand side above are in $\Yhg[0]\otimes\Yhg[+]$, and the other two terms are still in $\Yhg[+]\otimes\Yhg[0]$ but have a lower subscript $r$, which means that we can repeatedly apply this relation to these terms and this process will terminate.
    Thus all but the first term of the right-hand side of equation (\ref{eqn:TaTb}) will be mapped to zero by $\varepsilon^+$, so we have
    \[\mbraid_i(ab) = \varepsilon^+(\braid_i(ab)) = \varepsilon^+(\braid_i(a)\braid_i(b)) = a_0b_0 = \mbraid_i(a)\mbraid_i(b)\]
    which completes the proof.
\end{proof}

Notice that by a similar argument as in the proof above, we would obtain the same operators by modifying the braid group operators using $\varepsilon$ applied to the first tensor leg of the triangular decomposition rather than the third.
In other words, we obtain the same operators if we replace $\varepsilon^+$ in Definition \ref{D:mbraid} with $\varepsilon^- := \varepsilon\otimes{\id}\otimes{\id}$, or even with $\varepsilon\otimes{\id}\otimes\varepsilon$.
To make this more precise, considering the subsets
\[N^\pm := \{x^\pm_{i,r} : i\in\I, r\in\N\},\]
we have that $\Yhg N^\pm$ are left ideals and $N^\pm \Yhg$ are right ideals, and it follows from the PBW theorem that
\[\Yhg = \Yhg[0] \oplus (N^-\Yhg + \Yhg N^+).\]
Let $\Pi:\Yhg\to\Yhg[0]$ be the projection onto $\Yhg[0]$ in the direct sum above, then if $\beta$ is any nonzero weight of $\Yhg$ as a representation of $\g$, the corresponding weight space $\Yhg_\beta$ is in the kernel of $\Pi$.
The restriction of $\Pi$ to the zero weight space $\Yhg_0$ is the projection in the following direct sum:
\[\Yhg_0 = \Yhg[0] \oplus (\Yhg[0] \cap N^-\Yhg \cap \Yhg N^+).\]
From this, we can see that the restrictions of $\Pi$ and $\varepsilon^+$ to the zero weight space are in fact the same map, so we may replace $\varepsilon^+$ in Definition \ref{D:mbraid} with $\Pi$.
This allows us to use whichever of these maps is more convenient for our purposes at a given time.
For example, the proof of Lemma \ref{L:mbraid-hom} becomes much simpler using $\Pi$: projections are algebra homomorphisms and the rightmost direct summand above is an ideal in the zero weight space, so the restriction of $\Pi$ is an algebra homomorphism, hence $\mbraid_i$ must be as well.
Further, using properties of graded algebras, one can show that the operators $\mbraid_i$ are invertible and are coalgebra homomorphisms that commute with the antipode, and are therefore Hopf algebra automorphisms of $\Yhg[0]$; see \cite[Lemma 3.5]{friesen_braid_2024} for details.

Now recall that every $w\in\Wg$ has a lift $\braid_w\in\Bg$, and therefore a corresponding automorphism of $\Yhg$ which we will also denote by $\braid_w$.
The following lemma was proved in \cite[Lemma 3.6]{friesen_braid_2024} using the same method as in \cite[Thm. 5.3.19]{weekes_highest_2016}, and gives an analogue of this lifting for our modified operators.

\begin{lemma}\label{L:Tw}
    Let $w=s_{i_1}\cdots s_{i_\ell}$ be a reduced expression. Then
    \[\mbraid_w := \varepsilon^+\circ\braid_w\Big|_{\Yhg[0]} = \mbraid_{i_1}\cdots\mbraid_{i_\ell}.\]
\end{lemma}

As a consequence of this lemma, it follows that the operators $\mbraid_i$ satisfy the braid relations, and therefore define an action of the braid group on $\Yhg[0]$.
The results of this section are summarized in the following theorem.

\begin{theorem}\label{T:mbraid-action}
    The assignment $\braid_i\mapsto\mbraid_i$ for all $i\in\I$ defines an action of the braid group $\Bg$ on $\Yhg[0]$ by Hopf algebra automorphisms.
\end{theorem}


\section{Action on generating series of \texorpdfstring{$\Yhg[0]$}{Y0}}

In this section, we will compute the action of the modified braid group operators $\mbraid_i$ on $\Yhg[0]$ explicitly, using the generating series $(A_j(u))_{j\in\I}$ from \cite{gerasimov_class_2005}.
For the definition of these series and the other series used below, see Section \ref{ssec:alt-gen}.
The following proposition gives us explicit formulas for both the regular and modified braid group actions.
As a remark, the first equation of the following proposition was established for simply laced (i.e., type A, D, and E) Lie algebras in \cite[Lem. 5.3.16]{weekes_highest_2016}.

\begin{proposition}\label{P:tau-a}
    For all $i,j\in\I$, we have
    \[\braid_i(A_j(u)) =
    \begin{cases}
        D_i(u) & \text{if}\quad i=j, \\
        A_j(u) & \text{if}\quad i\neq j.
    \end{cases}\]
    Consequently, we have
    \[\mbraid_i(A_j(u)) = A_j(u)\xi_i(u)^{\delta_{ij}}.\]
\end{proposition}
\begin{proof}
    By the relations in the first part of Corollary \ref{C:GKLO}, if $i\neq j$ then $\mbraid_i(A_j(u)) = A_j(u)$.
    Using the relations in the second and third part of the same corollary, along with the fact that exponentials of linear maps are themselves linear maps, we obtain
    \begin{align*}
        \braid_i(A_i(u)) &= \exp(\ad(e_i))\exp(-\ad(f_i))\exp(\ad(e_i))(A_i(u)) \\
        &= \exp(\ad(e_i))\exp(-\ad(f_i))(A_i(u) + B_i(u)) \\
        &= \exp(\ad(e_i))(A_i(u) + B_i(u) - (A_i(u) - C_i(u) - D_i(u)) - C_i(u)) \\
        &= \exp(\ad(e_i))(B_i(u) + D_i(u)) \\
        &= B_i(u) + D_i(u) - B_i(u) \\
        &= D_i(u),
    \end{align*}
    which proves the first equality of the proposition.
    Now using the definition of $D_i(u)$, we have
    \[\varepsilon^+(D_i(u)) = \varepsilon^+(A_i(u)\xi_i(u) + C_i(u)A_i(u)^{-1}B_i(u)) = A_i(u)\xi_i(u)\]
    and so the second equality of the proposition follows from the definition of $\mbraid_i$.
\end{proof}

Using the defining relation of the generating series $A_j(u)$, along with the fact that the series $A_j(u)$ and $\xi_j(u)$ commute for all $j\in\I$, we obtain the following corollary, giving us an explicit formula for the modified braid group action on the more familiar generating series $\xi_j(u)$.

\begin{corollary}\label{C:tau-xi}
    For all $i,j\in\I$, we have
    \[\mbraid_i(\xi_j(u)) = \xi_j(u)\prod_{k=0}^{\abs{a_{ij}}-1} \xi_i\left(u-\frac{\hbar d_i}{2}(\abs{a_{ij}}-2k)\right)^{(-1)^{\delta_{ij}}}.\]
\end{corollary}

The formula of the above corollary simplifies to the following cases based on the Cartan matrix:
\[\begin{array}{c|l}
    a_{ij}a_{ji} & \mbraid_i(\xi_j(u)) \\
    \hline
    0 & \xi_j(u) \\
    1 & \xi_j(u)\xi_i(u-\frac{\hbar}{2}) \\
    2 & \xi_j(u)\xi_i(u)\xi_i(u-\hbar) \\
    3 & \xi_j(u)\xi_i(u+\frac{\hbar}{2})\xi_i(u-\frac{\hbar}{2})\xi_i(u-\frac{3\hbar}{2}) \\
    4 & \xi_j(u-\hbar)^{-1}
\end{array}\]
and we note that the last case occurs when $i=j$.
Finally, we give an example to illustrate these explicit formulas more concretely.

\begin{example}\label{E:sl3-mbraid-action}
    Consider the Lie algebra $\mfsl_3$, which has $2\times 2$ Cartan matrix given by $a_{11}=a_{22}=2$ and $a_{12}=a_{21}=-1$.
    Notice that the Cartan matrix is symmetric, so the symmetrizing integers are $d_1=d_2=1$.
    By definition, for $i\neq j$, we have
    \[\xi_i(u) = \frac{A_i(u-\frac{\hbar}{2})}{A_j(u)A_j(u-\frac{\hbar}{2})}.\]
    Using Corollary \ref{C:tau-xi} above, we find that the modified braid group action is given by
    \[\mbraid_i(\xi_i(u)) = \frac{1}{\xi_i(u-\hbar)}, \qquad
    \mbraid_i(\xi_j(u)) = \xi_j(u)\xi_i(u-\tfrac{\hbar}{2}),\]
    for all $i\neq j$.
    Using these formulas, it is easy to check that these braid group operators satisfy the braid relation $\mbraid_1\mbraid_2\mbraid_1 = \mbraid_2\mbraid_1\mbraid_2$ for $\mfsl_3$: we have
    \begin{align*}
        \mbraid_1\mbraid_2\mbraid_1(\xi_1(u)) &= \mbraid_1\mbraid_2\left(\frac{1}{\xi_1(u-\hbar)}\right) \\
        &= \mbraid_1\left(\frac{1}{\xi_1(u-\hbar)\xi_2(u-\frac{3\hbar}{2})}\right) \\
        &= \frac{1}{\xi_2(u-\frac{3\hbar}{2}),}
    \end{align*}
    and on the other hand,
    \begin{align*}
        \mbraid_2\mbraid_1\mbraid_2(\xi_1(u)) &= \mbraid_2\mbraid_1\left(\xi_1(u)\xi_2(u-\frac{\hbar}{2})\right) \\
        &= \mbraid_2\left(\xi_2(u-\frac{\hbar}{2})\right) \\
        &= \frac{1}{\xi_2(u-\frac{3\hbar}{2})}.
    \end{align*}
    The action on $\xi_2(u)$ is computed similarly, showing that these operators are the same.
\end{example}


\section{Action on the universal \texorpdfstring{$R$}{R}-matrix}

In this section, we show how the braid group operators and their modified versions interact with the universal $R$-matrix of the Yangian introduced in Section \ref{ssec:Y-Hopf}.

Recall the shift homomorphism $\tau_z:\Yhg\inj\Yhg[][z]$ used in the definition of the universal $R$-matrix.
Since this map restricts to the identity map on $\Ug$, by the intertwiner equation (\ref{eqn:R-intertwiner}) and the coproduct formula given in Section \ref{ssec:Y-Hopf}, for all $x\in\g$ we have:
\[[\Delta(x), \rmat(z)] = \ad(x\otimes 1 + 1\otimes x)(\rmat(z)) = 0.\]
From this, it follows that $\rmat(z)$ is fixed by $\exp(\ad(x))\otimes\exp(\ad(x))$, and thus by $\braid_i\otimes\braid_i$ for all $i\in\I$.
This proves the first part of the following proposition.

\begin{proposition}\label{P:braid-R}
    For all $i\in\I$, we have:
    \[(\braid_i\otimes\braid_i)(\rmat(z)) = \rmat(z) \qquad\text{and}\qquad (\mbraid_i\otimes\mbraid_i)(\rmat^0(z)) = \rmat^0(z).\]
\end{proposition}
\begin{proof}
    It remains to prove the second part of the proposition, which we will do by applying $\Pi\otimes\Pi$ to both sides of the identity in the first part of the proposition.
    By the same reasoning we used when we introduced the projection $\Pi$ in Section \ref{sec:mbraid}, we have that the restriction of $\Pi\otimes\Pi$ to the zero weight space $(\Yhg\otimes\Yhg)_0$ is an algebra homomorphism whose image is $\Yhg[0]\otimes\Yhg[0]$.
    The universal $R$-matrix and its components $\rmat^\pm(z)$ and $\rmat^0(z)$ have weight zero, and we have
    \[(\Pi\otimes\Pi)(\rmat^\pm(z)) = 1 \qquad\text{and}\qquad (\Pi\otimes\Pi)(\rmat^0(z)) = \rmat^0(z).\]
    From this, it follows that
    \[(\Pi\otimes\Pi)(\rmat(z)) = (\Pi\otimes\Pi)(\rmat^+(z)) \cdot (\Pi\otimes\Pi)(\rmat^0(z)) \cdot (\Pi\otimes\Pi)(\rmat^-(z)) = \rmat^0(z),\]
    so the right-hand side is as desired.
    For the left-hand side, again using the fact that the restriction of $\Pi\otimes\Pi$ is a homomorphism, we obtain
    \[(\Pi\circ\braid_i)^{\otimes 2}(\rmat(z)) = (\Pi\circ\braid_i)^{\otimes 2}(\rmat^+(z)) \cdot (\mbraid_i\otimes\mbraid_i)(\rmat^0(z)) \cdot (\Pi\circ\braid_i)^{\otimes 2}(\rmat^-(z)),\]
    and now it suffices to show that $(\Pi\circ\braid_i)^{\otimes 2}(\rmat^\pm(z))=1$.
    To see this, recall that $\rmat^\pm(z)$ are defined as sums of components labeled by weights, so we have
    \[(\braid_i\otimes\braid_i)(\rmat^\pm(z)) = \sum_{\beta\in Q^+}(\braid_i\otimes\braid_i)(\rmat_\beta^\pm(z))\]
    where the coefficients of the summand belong to $\Yhg_{\pm s_i(\beta)}\otimes\Yhg_{\mp s_i(\beta)}$.
    Thus each summand is in the kernel of $\Pi\otimes\Pi$ except in the case where $\beta=0$, since $\rmat_0^\pm(z)=1$.
    This completes the proof.
\end{proof}

\chapter{Weights of extremal vectors}

Here we show how the braid group acts on so-called ``extremal vectors,'' leading us to a formula for the Baxter polynomials.

\chapter{Baxter polynomials and cyclicity}

A representation of $\Yhg$ is said to be \emph{cyclic} if it is generated by a single vector, i.e., if it is a highest-weight representation.
In this chapter, we prove a conjecture from \cite[\S 7.4]{gautam_poles_2023} which states that the two sufficient conditions for the cyclicity and irreducibility of any tensor product $L(\uP)\otimes L(\uQ)$ obtained in \cite{gautam_poles_2023} and \cite{tan_braid_2015} are identical.


\section{Baxter polynomials and poles}\label{sec:baxter-poles}

Let $V$ be a finite-dimensional highest-weight representation of $\Yhg$, and let $\lambda\in\h^*$ be the highest weight of $V$ as a representation of $\g$.
Recall the series $A_i(u)$ of Section \ref{ssec:alt-gen}; for each $i\in\I$ let $\lambda_i^A(u)\in 1+u^{-1}\C\db{u^{-1}}$ denote the eigenvalue of $A_i(u)$ on the highest weight space $V_\lambda$.
We then introduce the normalized operator
\[A_i^V(u) := \lambda_i^A(u)^{-1}A_i(u)\big|_V \in\End(V)\db{u^{-1}},\]
which acts by the identity on the highest weight space.
Then by \cite[Thm. 4.4]{gautam_poles_2023} (see also \cite[Cor. 4.7]{gautam_poles_2023} together with \cite[Prop. 5.7, 5.8]{hernandez_shifted_2022}), there is a unique monic polynomial $T_i(u)\in\End(V)[u]$ that satisfies
\begin{equation}\label{eqn:transfer-op}
    T_i(u+\hbar d_i) = A_i^V(u)T_i(u).
\end{equation}
We note that $T_i(u)$ can be recovered as $\lambda_i^T(u)^{-1}\mathscr{T}_i(u)$, where $\mathscr{T}_i(u)$ is the \emph{$i$th abelianized transfer operator} introduced in \cite[\S 4.3]{gautam_poles_2023} and $\lambda_i^T(u)$ is the eigenvalue of $\mathscr{T}_i(u)$ on the highest weight space; see \cite[Remark 5.1]{friesen_braid_2024} for details.
The eigenvalues of the operator $T_i(u)$ are called the \emph{(specialized) Baxter polynomials} associated to $V$, and were first introduced in generality in \cite{frenkel_baxters_2015}.

The generating series $\xi_i(u)$ and $x^\pm_i(u)$ of $\Yhg$ introduced in Section \ref{ssec:Y-Hopf} operate on $V$ as the expansions at infinity of $\End(V)$-valued rational functions of $u$; see \cite[Prop. 3.6]{gautam_yangians_2016}.
We define the \emph{$i$th set of poles} of $V$ to be the joint set of poles of these operators:
\[\sigma_i(V) := \{\text{Poles of } \xi_i(u)\big|_V,\ x^\pm_i(u)\big|_V \in\End(V)(u)\} \subset\C.\]
The Baxter polynomials are related to the poles of $V$ in the following way: let $\zeros(P(u))$ denote the zeros of any polynomial $P(u)$, let $\mathcal{Z}_i(V)$ denote the zeros of all eigenvalues of $T_i(u)$, and let $\baxterV(u)$ denote the eigenvalue of $T_i(u)$ on the lowest weight space $V_{w_0(\lambda)}$ where $w_0\in\Wg$ is the longest element.
Then by \cite[Thm. 4.4]{gautam_poles_2023}, for all $i\in\I$ we have
\[\sigma_i(V) = \mathcal{Z}_i(V) = \zeros(\baxterV(u)).\]

In the case where $V$ is irreducible, the polynomials $\baxterV(u)$ (and hence the poles of $V$) were computed explicitly in \cite[Thm. 5.2]{gautam_poles_2023}: for all $i\in\I$,
\[\baxterV(u) = \prod_{j\in\I}\prod_{b=d_i}^{2\kappa-d_i} P_j\left(u-(b-d_j)\frac{\hbar}{2}\right)^{v_{ij}^{(b)}}\]
where $\uP=(P_j(u))_{j\in\I}$ is the tuple of Drinfeld polynomials associated to $V\cong L(\uP)$, $\kappa$ is $1/4$ times the eigenvalue of the Casimir element $C\in\Ug$ on the adjoint representation of $\g$ as in Example \ref{E:dual-braid-sl3}, and the integers $v_{ij}^{(b)}$ are obtained from the \emph{quantum Cartan matrix} $(v_{ij}(z))_{i,j\in\I}$; see \cite[\S 5.2]{friesen_braid_2024} for details.


\section{Baxter polynomials associated to extremal weights}

Now let $V$ be irreducible, so $V=L(\uP)$ for some Drinfeld polynomials $\uP=(P_i(u))_{i\in\I}$, and the highest weight of $V$ as a representation of $\g$ is $\lambda = \sum_{i\in\I}\deg(P_i)\fund_i$.
For any $w\in\Wg$, let $\baxterP(u)\in\C[u]$ denote the eigenvalue of $T_i(u)$ on the extremal weight space $V_{w(\lambda)}$.
In particular, if $w$ is the longest element $w_0$, then $\baxterP(u) = \baxterV(u)$.

The following theorem is the main result of this chapter.
It gives a factorization of the Baxter polynomial $\baxterP(u)$ in terms of the braid group action defined in the previous chapter on the Drinfeld polynomials.

\begin{theorem}\label{T:baxter}
    Let $w = s_{j_1}s_{j_2}\cdots s_{j_p}$ be a reduced expression for $w\in\Wg$.
    For each $1\leq r\leq p$, set $w_r := s_{j_{r+1}}\cdots s_{j_p}$, where $w_p = {\id}$.
    Then
    \[\baxterP(u) = \prod_{r:j_r=i} \braid_{w_r}(\uP)_i.\]
    Moreover, $\braid_{w_r}(\uP)_{j_r}$ is a monic polynomial in $u$ for each $r$.
\end{theorem}
\begin{proof}
    First, since $s_{j_p}\cdots s_{j_{r+1}}s_{j_r}$ is a reduced expression, it follows that $w_r^{-1}(\alpha_{j_r}) = s_{j_p}\cdots s_{j_{r+1}}(\alpha_{j_r}) \in\Phi^+$.
    Then by Corollary \ref{C:Tan-monic}, $\braid_{w_r}(\uP)_{j_r}$ is a monic polynomial in $u$ for each $r$.

    By the defining equation \ref{eqn:transfer-op} of $T_i(u)$, the eigenvalue of the normalized operator $A_i^V(u) = \lambda_i^A(u)^{-1}A_i(u)$ on the extremal weight space $V_{w(\lambda)}$ is given by
    \[\frac{\baxterP(u+\hbar d_i)}{\baxterP(u)}\]
    hence the eigenvalue of $A_i(u)$ on $V_{w(\lambda)}$ is $\lambda_i^A(u)$ times the above.
    On the other hand, by Proposition \ref{P:extremal-weight} the eigenvalue of $A_i(u)$ on $V_{w(\lambda)}$ is $\braid_w(\ul)(A_i(u))$, where $\ul$ is the highest weight of $V$.
    Then by the uniqueness assertion of Lemma \ref{L:diff-eqn}, it suffices to prove that
    \[\braid_w(\ul)(A_i(u))
    = \lambda_i^A(u)\prod_{r:j_r=i}\braid_{w_r}(\ul)_i
    = \lambda_i^A(u)\prod_{r:j_r=i}\frac{\q^{2d_i}\braid_{w_r}(\uP)_i}{\braid_{w_r}(\uP)_i},\]
    where the second equality follows from the fact that the action of $\braid_{w_r}$ is an automorphism and $\ul = (\q^{2D}\uP)\uP^{-1}$.
    Using the definition of the dual braid group action from the beginning of the previous chapter, the first equality is equivalent to
    \[\mbraid_{w^{-1}}(A_i(u)) = A_i(u)\prod_{r:j_r=i}\mbraid_{w_r^{-1}}(\xi_i(u)).\]
    To prove this, we will use induction on the length $p$ of $w$: if $p = 1$ then this equation reduces to the identity $\mbraid_j(A_i(u)) = A_i(u)\xi_i(u)^{\delta_{ij}}$ that we established in Proposition \ref{P:tau-a}.
    Now suppose that this equation holds for $w$ of length $p$, and consider the element $w' = ws_{j_{p+1}}$ of length $p+1$.
    Then again using the identity of Proposition \ref{P:tau-a}, we have
    \begin{align*}
        \mbraid_{(w')^{-1}}(A_i(u)) &= \mbraid_{j_{p+1}}(\mbraid_w(A_i(u))) \\
        &= A_i(u)\xi_i(u)^{\delta_{i,j_{p+1}}}\prod_{\substack{1\leq r\leq p \\ j_r=i}}\mbraid_{j_{p+1}}(\mbraid_{w_r^{-1}}(\xi_i(u))) \\
        &= A_i(u)\xi_i(u)^{\delta_{i,j_{p+1}}}\prod_{\substack{1\leq r\leq p \\ j_r=i}}\mbraid_{(w'_r)^{-1}}(\xi_i(u)) \\
        &= A_i(u)\prod_{\substack{1\leq r\leq p+1 \\ j_r=i}}\mbraid_{(w'_r)^{-1}}(\xi_i(u))
    \end{align*}
    which completes the proof.
\end{proof}


\section{Cyclicity criteria for tensor products}

An important property of the poles of representations of $\Yhg$ is that they encode information about when the tensor product of two irreducible representations is cyclic or irreducible.
By \cite[Thm. 7.2]{gautam_poles_2023}, the representation $L(\uP)\otimes L(\uQ)$ is cyclic if for all $i\in\I$, none of the zeros of $Q_i(u+\hbar d_i)$ are $i$th poles of $L(\uP)$, i.e., if
\[\zeros(Q_i(u+\hbar d_i)) \subset \C\setminus\sigma_i(L(\uP)).\]
By \cite[Cor. 7.3]{gautam_poles_2023}, $L(\uP)\otimes L(\uQ)$ is irreducible if in addition to the above condition, none of the zeros of $P_i(u+\hbar d_i)$ are $i$th poles of $L(\uQ)$:
\[\zeros(P_i(u+\hbar d_i)) \subset \C\setminus\sigma_i(L(\uQ)).\]

\begin{example}\label{E:cyclicity}
    Consider the fundamental representations $V = L_{\fund_1}(a)$ and $W = L_{\fund_1}$ of $Y_\hbar(\mfsl_2)$, which we defined in Example \ref{E:dual-braid-sl3}.
    These representations have Drinfeld polynomials $P_1(u) = u-a$ and $Q_1(u) = u$, respectively.
    The longest element of the Weyl group is the simple reflection $s_1$, so using the formula of Theorem \ref{T:baxter}, we see that the Baxter polynomials associated to the lowest weights of these representations are just the Drinfeld polynomials: $\baxter_{1,V}^{\mfsl_2}(u) = u-a$ and $\baxter_{1,W}^{\mfsl_2} = u$.
    Looking at the zeros of these polynomials, the condition above for $V\otimes W$ to be cyclic is equivalent to $a\neq -\hbar$, and the additional condition for $V\otimes W$ to be irreducible is equivalent to $a\neq\hbar$.
\end{example}

Another sufficient condition for the cyclicity of a tensor product of irreducible representations was given in \cite[Thm. 4.8]{tan_braid_2015} using the action of the braid group on $(\C(u)^\times)^\I$, following \cite{chari_braid_2002}.
Let $w_0 = s_{j_1}s_{j_2}\cdots s_{j_p}$ be a reduced expression for the longest element $w_0\in\Wg$, and set $w_r := s_{j_{r+1}}\cdots s_{j_p}$ for each $1\leq r\leq p$ as in the previous section.
Then by \cite[Thm. 4.8]{tan_braid_2015} and the results of Section \ref{sec:extending}, $L(\uP)\otimes L(\uQ)$ is cyclic if for all $1\leq r\leq p$, the polynomials $Q_{j_r}(u+\hbar d_{j_r})$ and $\braid_{w_r}(\uP)_{j_r}$ have no roots in common:
\[\zeros(Q_{j_r}(u+\hbar d_{j_r})) \subset \C\setminus\zeros(\braid_{w_r}(\uP)_{j_r}).\]

It was conjectured in \cite[\S 7.5]{gautam_poles_2023} that the two sufficient conditions for cyclicity above are identical.
To prove this, recall from Section \ref{sec:baxter-poles} that the poles of a finite-dimensional highest-weight representation are exactly the zeros of the Baxter polynomial associated to the lowest weight.
Then using the factorization given in Theorem \ref{T:baxter}, we obtain the following corollary.

\begin{corollary}\label{C:baxter-poles}
    Let $\uP$ be a tuple of Drinfeld polynomials.
    For all $i\in\I$, the $i$th set of poles of $L(\uP)$ is given by
    \[\sigma_i(L(\uP)) = \bigcup_{r:j_r=i}\zeros(\braid_{w_r}(\uP)_i).\]
\end{corollary}

Combining this corollary with the cyclicity criteria of \cite{gautam_poles_2023} and \cite{tan_braid_2015} above, we see that they are identical.
The following corollary summarizes this result.

\begin{corollary}\label{C:cyclicity}
    Let $\uP = (P_i(u))_{i\in\I}$ and $\uQ = (Q_i(u))_{i\in\I}$ be tuples of Drinfeld polynomials.
    The following conditions are equivalent:
    \begin{enumerate}
        \item For all $i\in\I$, $\zeros(Q_i(u+\hbar d_i))\subset\C\setminus\sigma_i(L(\uP))$.
        \item For all $1\leq r\leq p$, $\zeros(Q_{j_r}(u+\hbar d_{j_r}))\subset\C\setminus\zeros(\braid_{w_r}(\uP)_{j_r})$.
    \end{enumerate}
    If either of these conditions hold, then the representation $L(\uP)\otimes L(\uQ)$ is cyclic.
\end{corollary}


\section{Fundamental representations of \texorpdfstring{$Y_\hbar(\mfsl_n$)}{Y(sln)}}

In this section we will expand upon Example \ref{E:cyclicity} of the previous section and explicitly compute Baxter polynomials associated to the lowest weights of fundamental representations in the case where $\g=\mfsl_n$.

For convenience, let $\baxter_{ij}(u)$ denote the $i$th Baxter polynomial for the $j$th fundamental representation: $\baxter_{i,L_{\fund_j}}^{\mfsl_n}(u)$.
As in \cite[\S 6]{chari_braid_2002}, if we let $\gamma_k := s_1s_2\cdots s_k\in\Wg$, then a reduced expression for the longest element is given by $w_0 = \gamma_{n-1}\gamma_{n-2}\cdots\gamma_1$.
Recall from Example \ref{E:dual-braid-sl3} that there is an automorphism of the Dynkin diagram induced by the action of $w_0$ on the simple roots; in this case, one can show that this automorphism is given by $i\mapsto n-i$.
From this it follows that $\baxter_{ij}(u) = \baxter_{n-i,n-j}(u)$, meaning that in order to compute all of the Baxter polynomials, we need only compute $\baxter_{ij}(u)$ where $i\geq j$.
Alternatively, this means we need only compute $\baxter_{ij}(u)$ where $j\leq\lceil n/2 \rceil$ since the Baxter polynomials for the $j$th fundamental representation are the same as those of the $(n-j)$th fundamental representation in reverse order.
Either way, making use of the diagram automorphism halves the amount of computation required.

To illustrate the computation of Baxter polynomials using the braid group action, we will consider the particular example of the second fundamental representation of $Y_\hbar(\mfsl_5)$.
This representation has Drinfeld polynomials $\uP = (1, u, 1, 1)$, and the reduced expression for the longest element of the Weyl group described above is
\[w_0 = \gamma_4\gamma_3\gamma_2\gamma_1 = s_1s_2s_3s_4s_1s_2s_3s_1s_2s_1.\]
Recall that the Cartan matrix for $\mfsl_n$ is given by
\[a_{ij} =
\begin{cases}
    2 & \text{if } i=j \\
    -1 & \text{if } \abs{i-j}=1 \\
    0 & \text{otherwise}
\end{cases}\]
and so the relevant formulas for the action of the braid group on $\uP$ are as follows:
\[\braid_i(\uP)_j =
\begin{cases}
    P_j(u-\hbar)^{-1} & \text{if } i=j \\
    P_j(u)P_i(u-\frac{\hbar}{2}) & \text{if } \abs{i-j}=1 \\
    P_j(u) & \text{otherwise}
\end{cases}\]
Now that we have this data, the formula of Theorem \ref{T:baxter} tells us that we can compute the Baxter polynomials for this representation by applying $\braid_{w_0}$ to $\uP$ one $\braid_i$ at a time, then picking out certain polynomials that we obtain along the way.
The table below shows the process of applying $\braid_{w_0}$ to $\uP$: each row represents one step, where the leftmost column gives the index of the braid group operator $\braid_i$ to be applied to the tuple of polynomials given in the remaining columns.
\[\begin{array}{c|cccc}
    \hphantom{m}i\hphantom{m} & P_1 & P_2 & P_3 & P_4 \\
    \hline
    1 & 1 & u & 1 & 1 \\
    2 & 1 & u & 1 & 1 \\
    1 & u-\frac{\hbar}{2} & (u-\hbar)^{-1} & u-\frac{\hbar}{2} & 1 \\
    3 & (u-\frac{3\hbar}{2})^{-1} & 1 & u-\frac{\hbar}{2} & 1 \\
    2 & (u-\frac{3\hbar}{2})^{-1} & u-\hbar & (u-\frac{3\hbar}{2})^{-1} & u-\hbar \\
    1 & 1 & (u-2\hbar)^{-1} & 1 & u-\hbar \\
    4 & 1 & (u-2\hbar)^{-1} & 1 & u-\hbar \\
    3 & 1 & (u-2\hbar)^{-1} & u-\frac{3\hbar}{2} & (u-2\hbar)^{-1} \\
    2 & 1 & 1 & (u-\frac{5\hbar}{2}) & 1 \\
    1 & 1 & 1 & (u-\frac{5\hbar}{2}) & 1 \\
    & 1 & 1 & (u-\frac{5\hbar}{2}) & 1
\end{array}\]
Then using the formula for the Baxter polynomials, we construct $\baxter_{ij}(u)$ by taking the product of the polynomials in the column labeled $P_i$ for each row with $i$ in the leftmost column.
Doing so for each $i$, we obtain:
\[\baxter_{12}(u) = u-\frac{\hbar}{2}, \qquad
\baxter_{22}(u) = u(u-\hbar), \qquad
\baxter_{32}(u) = (u-\frac{\hbar}{2})(u-\frac{3\hbar}{2}), \qquad
\baxter_{42}(u) = u-\hbar.\]
By the discussion at the beginning of this section, this also gives us the Baxter polynomials for the third fundamental representation of $Y_\hbar(\mfsl_5)$:
\[\baxter_{12} = \baxter_{43}, \qquad
\baxter_{22} = \baxter_{33}, \qquad
\baxter_{32} = \baxter_{23}, \qquad
\baxter_{42} = \baxter_{13}.\]

Something to note is that in our choice of reduced expression for $w_0$, the simple reflection $s_i$ occurs $n-i$ times for each $1\leq i\leq n-1$.
This number of occurrences is the number of factors in our formula for $Q_{ij}(u)$, but some of these factors may be $1$, so in particular this means that $Q_{ij}(u)$ has degree at most $n-i$.

Another interesting property of this choice of reduced expression for $w_0$ of $\mfsl_n$ is that it contains reduced expressions for $w_0$ of $\mfsl_m$ for all $m\leq n$: removing $\gamma_{n-1}$ from the left gives us $w_0=\gamma_{n-2}\cdots\gamma_1$ for $\mfsl_{n-1}$, and so on.
Because of the way we construct the Baxter polynomials using our formula as above, this means that the computation for the $j$th fundamental representation of $Y_\hbar(\mfsl_n)$ also encodes the computation for the $j$th fundamental representation of $Y_\hbar(\mfsl_m)$ with $m\leq n$, as long as $j<m$.
For example, we may obtain the Baxter polynomials for the second fundamental representation of $Y_\hbar(\mfsl_4)$ using the above table by considering only the rows before the first occurrence of $4$ in the leftmost column, and only the first three components of the tuple of polynomials:
\[\baxter_{12}(u) = u-\frac{\hbar}{2}, \qquad
\baxter_{22}(u) = u(u-\hbar), \qquad
\baxter_{32}(u) = u-\frac{\hbar}{2}.\]
Note that this time we do not get the Baxter polynomials for another fundamental representation, since the diagram automorphism for $\mfsl_4$ maps $2$ to $2$.
It follows from this discussion that  $\baxter_{ij}^{\mfsl_m}(u)$ divides $\baxter_{ij}^{\mfsl_n}(u)$ for all $m\leq n$.

Lastly, we note that there is another explicit formula for $\baxter_{ij}(u)$ in the case where $\g=\mfsl_n$ given in \cite[Cor. 5.5]{gautam_poles_2023}:
\[\baxter_{ij}(u) = \prod_{b\in\mathbf{J}_{ij}}\left(u-\hbar\left(\frac{i+j}{2}-b\right)\right),\]
where $\mathbf{J}_{ij}$ denotes the $\Z$-valued interval
\[\mathbf{J}_{ij} := [i+j+1-n, i] \cap [1, j].\]
This formula explains why in the above examples (and indeed in all cases), the roots of $\baxter_{ij}(u)$ differ by a factor of $\hbar$.


\nocite{*}
\uofsbibliography[alpha]{refs}

% \uofsappendix
% \begin{appendices}
%     \input{chapters/appendix}
% \end{appendices}

\end{document}
